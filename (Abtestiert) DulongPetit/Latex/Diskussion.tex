In diesem Abschnitt werden die während des Versuchs gemessenen Werte und
die aus diesen berechneten Größen noch einmal genauer betrachtet und 
abschließend diskutiert, wobei auch auf den Versuchsaufbau und die
Versuchsdurchführung Bezug genommen wird.\\


Die mit Hilfe des Thermoelements gemessenen Temperaturen, sind sowohl bei der 
Messung zur Bestimmung der Wärmekapazität des Kalorimeters als auch bei der Messung
zur Bestimmung des spezifischen Wärmekapazität der beiden Metalle scheinen plausibel zu sein.

Bei den berechneten Wärmekapazitäten, zeigt sich jedoch eine große Abweichung, im Vergleich zu den Literaturwerten. Bis auf die letzten beiden Messungen mit Aluminium weisen alle 
berechneten spezifischen Wärmekapazitäten eine deutliche Abweichung, durchschnittlich
ca. 60\% nach oben auf. Dies spräche für einen systematischen Fehler der Versuchsdurchführung.
Da die Messungen an Aluminium und Kupfer jedoch abwechselnd durchgeführt wurden, die letzten beiden Messungen an Aluminium dabei die vierte und sechste Messung waren, müssen die Ursachen dieses systematischen Fehlers auch eine Erklärung für die vereinzelt auftretenden Abweichungen nach unten geben.
Eine mögliche Begründung dieses Fehlers, stellt die starke Temperaturabhängigkeit der 
spezifischen Wärmekapazität \eqref{eq:C_Metall} dar. Wie aus \autoref{tab:DataI_Al} zu erkennen ist unterscheiden sich die letzten beiden Messungen nur wenig von der ersten und von einander und dennoch führen diese geringen Abweichung  bei 
den Messwerten zu den großen Abweichungen bei der spezifischen
Wärmekapazität von Aluminium. Damit lässt sich nicht nur die 
Abweichung der beiden letzten Messungen an Aluminium erklären, sondern
auch der systematische Fehler, der anderen spezifischen
Wärmekapazitäten.

Ein andere Begründung ist die Annahme, es ginge in 
dem Kalorimeter keine Wärme in die Umgebung über, was nicht  
gänzlich realisierbar ist und damit die jeweilige Mischtemperatur 
geringer ausfällt als theoretisch bestimmt.
Diese Temperaturdifferenz gegenüber dem idealisierten Fall führt 
mit der oben angeführten ersten Begründung zu einer umso größeren 
Differenz der spezifischen Wärmekapazitäten.\\
Der Vergleich zwischen dem vom Dulong-Petitschen-Gesetz
vorhergesagtem Wert für die Molwärme $3R$ und den aus den 
Messwerten berechneten Molwärmen, zeigt ebenfalls den oben 
diskutierten systematischen Fehler. 
Jedoch zeigt sich, dass das Gesetz zumindest die richtige 
Größenordnung vorhersagt. Vergleicht man die $3R$ mit den
Literaturwerten $C_{v} \approx 24$ so gibt das Gesetz sogar eine gute 
Näherung der ersten beiden Stellen.\\            

Es zeigt sich demnach, dass das Dulong-Petitsche-Gesetz für den allgemeinen 
Fall eine gute Näherung der Molwärme von Festkörpern hergibt, der genaue Wert 
jedoch einen Grenzwert darstellt, wie es aus der quantenmechanischen Betrachtung folgt.    
 
