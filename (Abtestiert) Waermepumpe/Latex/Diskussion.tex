Wie schon durch die Fragestellung in der Anleitung suggeriert weisen die realen 
Güteziffern einen großen Unterschied zu den idealen Güteziffern auf.
An dem in \autoref{tab:Güte} aufgeführten Verhältnis dieser beiden Güteziffern
ist zusehen, dass die reale Güteziffer durchschnittlich nur ca. $20\%$ der idealen Güteziffer
entspricht.
Eine Begründung für diese große Abweichung ist der hohe Verlust an Energie durch Reibung und Widerstand,
dies ist gut an der Mechanischen Leistung in \autopageref{tab:Masse} zu sehen. Das Verhältnis von mechanischer
zu aufgewandter elektrischer Leistung $P = \SI{125(5)}{W} $ liegt im Mittel ($\mean{P_{mech}} = \SI{16,53}{W}$)
bei ca. $13\%$. Das heißt, dass nur etwas mehr als ein Zehntel der elektrischen Leistung 
in der realen Situation tatsächlich genutzt werden kann.\\

Auch ist zu bemerken, dass, obwohl beide Güteziffern mit voranschreitender Zeit abnehmen, das Verhältnis
mit der Zeit zunimmt. Der Grund dafür ist auch aus \autoref{tab:Güte} zu entnehmen. Während die reale 
Güteziffer bei der letzten gezeigten Zeit noch etwa $80\%$ des ersten Wertes hat, hat sich der Wert der 
idealen Güteziffer mehr als halbiert. Dies lässt sich durch den schon in der Anleitung herausgestellten 
Zusammenhang des idealen Wirkungsgrads und der Temperaturdifferenz zwischen den beiden Reservoirs (siehe \eqref{eq:vid} begründen,
da die Temperatur in  dem einen Reservoir größer und in dem anderen kleiner wird fällt die ideale Güteziffer schneller als die reale.
Im Gegensatz dazu hängt die reale Güteziffer nur von der Änderung der Temperatur in Reservoir 1 
ab und verringert sich somit nicht auch noch bei abnehmender Temperatur in Reservoir 2, dies führt zu einem langsameren abfallen der Güteziffer.\\

Zusammenfassend ist somit zusagen, dass die Effizienz thermodynamische Maschinen weit unter der theoretisch möglichen liegen,
da immer Verlust in Form von nicht konservativen Kräften auftreten, die die Leistung einer solchen Maschine im Vergleich zum idealisierten Fall
erheblich verringern.     
   
 
 