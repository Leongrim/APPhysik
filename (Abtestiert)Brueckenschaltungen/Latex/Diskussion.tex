In diesem Abschnitt werden die in \cref{sec:Auswertung} berechneten 
Größen abschließend diskutiert und auf deren Genauigkeit beziehungsweise
Plausibilität hin untersucht, wobei auch auf die verwendeten
Versuchsaufbauten Bezug genommen wird.\\

Die mit Hilfe der Messbrücken in \cref{sec:Auswertung_Wheatstone} und \ref{sec:Auswertung_Kapazitaet} 
berechneten Größen des ohmschen Widerstandes, der ideale Kapazität und der Kapazität mit Wirkwiderstand,
sind aufgrund der relativ geringen Abweichungen vom Mittelwert als plausible anzusehen. 
Die vorliegenden Abweichungen  sind zum einen durch die Toleranzen der verwendeten Bauteile 
und zum anderen dadurch zu begründen, dass es aufgrund der in \cref{sec:Auswertung_Klirrfaktor} untersuchten
\enquote{Unreinheit} der vom Generator erzeugten Frequenzen nicht möglich ist, die Brückenspannung exakt 
auf Null abzustimmen.  
Hier zeigt sich wiederum der Vorteil der Brückenschaltungen, da nur eine Spannung gemessen werden muss
bleiben die Abweichung die durch die Messinstrumente verursacht werden gering und man erhält schon  
mit 3 Messungen ein Ergebnis mit hoher Genauigkeit.\\
  
Für die Messung der Induktivität mit Wirkwiderstand in \cref{sec:Auswertung_Induktivität_Messbrücke}
gelten die zu vor angesprochen Vorteile der Brückenschaltung im gleichen Maße, auch wenn die
Abweichung bei dieser Messung größer ausfallen als bei den vorherigen Messungen. Ein Grund dafür ist  
die zwar nur um einen, dennoch geringer Anzahl an Messungen, da \enquote{Ausreißer} in den Messwerten,
bei weniger Messungen eine größere Gewichtung erhalten.  
Die Messung in \cref{sec:Auswertung_Induktivität_Maxwell} mit Hilfe der Maxwell-Brücke ist jedoch
ein sehr ungenau und liefert ein unplausibles Ergebnis. Dieser Eindruck entsteht nicht durch die Abweichungen der
erhaltenen Werte, denn auch diese sind eher gering, sondern durch eine Vergleich mit den in
\cref{sec:Auswertung_Induktivität_Messbrücke} erhaltenen Werte. So unterscheiden sich die berechneten Induktivitäten
$\mean{L_{x}}$ in \cref{eq:LxRx_Brücke} und $L_{x}$ in \cref{eq:LxRx_Maxwell} um einen Faktor $> 10$, was in Anbetracht der 
Tatsache, dass in beiden Fällen die selbe Spule vermessen wurde ein sehr unrealistisches Ergebnis darstellt.
Ein Grund für diese große Abweichung ist wiederum die einmalige Durchführung der Messung, da durch eine mehrmalige
Messung neben der schon angesprochene Genauigkeit, auch das Vermögen stiege eine Aussage über den Ursprung dieses Fehlers zu 
machen. Bei der einen vorliegenden großen Abweichung, ist es jedoch unmöglich zu bestimmen ob es sich bei
dem Grund für diese Abweichung um einen zufälligen, einen systematischen oder einen groben Fehler handelt.\\
\pagebreak

Zu der Bestimmung der Frequenzabhängigkeit der Wien-Robinson-Brücke \cref{sec:Auswertung_Frequenz} ist zusagen, dass
der gemessene Wert der Frequenz bei der die Brückenspannung minimal wird mit einer relativen Abweichung 
von $\tfrac{\envert{\nu_{0,theo} - \nu_{0}}}{\nu_{0, theo}} = \SI{2.3}{\percent}$ zur theoretisch bestimmten 
Frequenz ein genaues und plausibles Ergebnis darstellt.
Dies Genauigkeit, spiegelt sich auch in der \cref{fig:WienRobinson} wider, so folgen die Messwerte dem 
theoretischen Verlauf mit großer Übereinstimmung für Frequenzverhältnisse $\frac{\nu}{\nu_{0}} < 2$. 
Der veränderte Verlauf der Messwerte für größere Frequenzen und vor allem das erneute Abfallen der Messwerte 
für die höchsten gemessenen Frequenzen, sind auf den im Versuchsaufbau verwendeten Tiefpass zurückzuführen,
der vor den Eingang des Oszilloskops geschaltet wurde.\\
Der Berechnete Klirrfaktor des Frequenzgenerators \cref{eq:Klirrfaktor_Wert} liegt mit 
ungefähr $\SI{5}{\percent}$ in einem, für 
relative Werte dieser Art, erwartbaren Bereich und ist damit plausibel. \\

Abschließend lässt sich sagen, das Brückenschaltungen, bis auf \enquote{Ausreißer}, eine sehr genaue und gute Methode
darstellen, den Widerstand eines elektrischen Bauteils zu bestimmen. 