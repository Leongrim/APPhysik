\begin{table}[!h]
	\centering
	\begin{tabular}{|c|c|c|}
		\hline
		Gang & Differenzfrequenz hin & Differenzfrequenz zurück\\
		$g$ & $\Delta \nu_{h}\,[\si{\hertz}]$\cref{std:Frequenzänderung} & $\Delta \nu_{r}\,[\si{\hertz}]$\cref{std:Frequenzänderung}\\\hline\hline
		\num{6}  & \num{10(7)}  & \num{4(7)} \\
		\num{12}  & \num{13(7)}  & \num{2(7)} \\
		\num{18}  & \num{17(7)}  & \num{-1(7)} \\
		\num{24}  & \num{19(7)}  & \num{-4(7)} \\
		\num{30}  & \num{22(7)}  & \num{-7(7)} \\
		\num{36}  & \num{25(7)}  & \num{-10(7)} \\
		\num{42}  & \num{28(7)}  & \num{-13(7)} \\
		\num{48}  & \num{31(7)}  & \num{-15(7)} \\
		\num{54}  & \num{33(7)}  & \num{-19(7)} \\
		\num{60}  & \num{37(7)}  & \num{-22(7)} \\
		\hline
	\end{tabular}
	\caption{Frequenzänderungen des Wagens nach der Direkten Methode\\ \hspace*{1.95cm} in den verschiedenen Gängen \label{tab:Auswertung_Frequenzänderung_Direkt}}
\end{table}