
Mit \autoref{fig:Aufbau} wird der prinzipielle Aufbau der Wärmepumpe dargestellt. Der Kompressor \emph{K} erzeugt einen ständigen Kreislauf des Systems, indem er den Aggregatzustand des Transportgases (\ce{Cl2F2C}) beeinflusst von flüssig zu gasförmig und umgekehrt. \\
Durch das Drosselventil \emph{D} wird ein Druckunterschied erzeugt, unter dem Druck $p_a$ und bei Temperatur $T_2$ ist das Wasser gasförmig und unter $p_b$ und $T_1$ flüssig. \\

\begin{wrapfigure}{r}{0.6\textwidth}
	\includegraphics*[scale=0.45]{Grafiken/Aufbau.jpg}
	\caption{Prinzipieller Aufbau einer Wärmepumpe ($p_a < p_b$ ; $T_2 < T_1$)}
	\label{fig:Aufbau}
\end{wrapfigure}

Im kalten Reservoir 2 verdampft das Transportgas unter Wärmeaufnahme und wird im Kompressor komprimiert, dadurch erhöhen sich Druck und Temperatur, danach gibt es bei der Kondensation die Wärme ab und so erhöht sich die Temperatur in Reservoir 1, während sie in Reservoir 2 sinkt.
Bei jedem Messgang werden Temperaturen und Drücke von $T_1$ und $T_2$, bzw. von $p_a$ und $p_b$ notiert. Abgelesen wird im
90 Sekunden Takt bis Reservoir 1 maximal 50 Grad Celsius erreicht hat.
