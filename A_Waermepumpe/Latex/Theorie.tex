Aus dem ersten Hauptsatz der Thermodynamik geht hervor, dass die innere Energie, die Summe 
Wärmeänderung und der verrichteten Arbeit ist.
\begin{empheq}{equation}
	dU = \delta Q + \delta W
\end{empheq}
Sobald dem kälteren Reservoir die Wärmemenge $Q_2$ abgezogen wird, wird die Wärmemenge $Q_1$ 
dem wärmeren Reservoir zu geführt, dies geschieht nur mit der verrichteten Arbeit $A$:
\begin{empheq}{equation}	
	Q_1 = Q_2 + A
\end{empheq}
Die Güteziffer $\nu$ gibt an wie \enquote{gut}, also wie gering der Verlust ist. 
Sie sagt aus, dass bei kleiner Temperaturdifferenz 
weniger Arbeitsaufwand geleistet werden muss.
\begin{empheq}{equation}
	\nu_{ideal} = \dfrac {Q_1}{A} = \frac{T_1}{T_1-T_2}
	\label{eq:vid}
\end{empheq}
Da das System nicht vollständig isoliert ist, gibt es Abweichungen von $\nu_{ideal}$ und der Vorgang ist somit auch nicht reversibel unter realen Voraussetzungen, daher folgt:
\begin{empheq}{equation}	
	\nu_{real} < \dfrac{T_1}{T_1-T_2}
\end{empheq}
Berechnen lässt sich die reale Güteziffer durch die aufgenommene elektrische Leistung $P$ und der Änderung der Wärmemenge 
$Q_{1}$, welche sich wiederum aus der Änderung der Temperatur $T_{1}$ bestimmen lässt. Man bedarf demnach der zwei Gleichungen
\begin{empheq}{align}
	\label{eq:dQ1}
	\dfrac{\Delta Q_{1}}{\Delta t} &= (m_{1}c_{w} + m_{k}c_{k}) \dfrac{\Delta T_{1}}{\Delta t} \\
	\label{eq:vreal}
	\dfrac{1}{P} \dfrac{\Delta Q_{1}}{\Delta t} &= \nu_{real}. 
\end{empheq}  

Der Massendurchsatz des Transportgases, jene Größe die beschreibt wie viel Masse des Transportgases pro Zeiteinheit
durch das Drosselventil \emph{D} strömen. \\
Dieser lässt sich vergleichbar zur realen Güteziffer mit den Gleichungen 
\begin{empheq}{align}
	\label{eq:dQ2}
	\dfrac{\Delta Q_{2}}{\Delta t} &= (m_{2}c_{w} + m_{k}c_{k}) \dfrac{\Delta T_{2}}{\Delta t} \\
	\label{eq:dM}
	\dfrac{1}{L} \dfrac{\Delta Q_{2}}{\Delta t}  &= \dfrac{\Delta m}{\Delta t} 
\end{empheq}  
bestimmen, wobei $L$ die Verdampfungswärme des verwendeten Gases darstellt.

Mit Hilfe des Massendurchsatzes ist nun die mechanische Leistung des Kompressor durch
\begin{empheq}{equation}
	P_{mech} = \dfrac{1}{\kappa - 1}\left(p_{b}\sqrt[\kappa]{\dfrac{p_{a}}{p_{b}}} - p_{a} \right) \dfrac{1}{\rho} 
	\dfrac{\Delta m}{\Delta t}
	\label{eq:Pmech}
\end{empheq}
 berechnen. Dabei ist $\kappa$ der Adiabatenexponent und $\rho$ die Dichte des verwendeten Gases. 