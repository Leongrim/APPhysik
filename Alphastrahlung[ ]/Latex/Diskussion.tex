Im Folgenden werden die erhaltenen Versuchsergebnisse noch einmal
abschließend diskutiert und dabei auf ihre Plausibilität hin untersucht.
Dabei werden auch Versuchsaufbau und Durchführung mit in Betracht gezogen.\\

Die in den ersten beiden Messreihen erhaltenen Ergebnisse für die 
Energie der Alphastrahlung (\cref{eq:Messergebnis_I_E} und \cref{eq:Messergebnis_II_E})
weisen zwar eine Abweichung von $\Delta E_{\alpha} = \SI{0.295}{\mega\eV}$ auf liegen
jedoch beide unter der in der Anleitung \cite{V701} genannten Grenze von 
$E_{\alpha} = \SI{2.5}{\mega\eV}$.
Ein Vergleich der Messdaten zeigt, dass die im Abstand $x_{0} = \SI{25}{\milli\meter}$ 
gemessenen Daten wesentlich weniger Abweichungen enthalten als die des kürzeren Abstandes,
wodurch die in \cref{sec:Messung_II}  berechnete Energie \cref{eq:Messergebnis_II_E} 
der Alphastrahlung plausibler zu seien scheint. \\

Der Vergleich zwischen dem in \cref{fig:Messdaten_III} dargestellten Histogramm und 
den Poissonverteilung in \cref{fig:Messdaten_III_Poisson} und der Gaußverteilung in
\cref{fig:Messdaten_III_Gauss} zeigen, dass die Statistik des radioaktiven Zerfalls
keiner der beiden Verteilungen entspricht. Da es sich bei der Gaußverteilung um eine 
kontinuierliche Verteilung handelt, ist diese für die Darstellung von diskreten Messwerte
ohnehin schon ungeeignet.\\
Aber auch zu der Poissonverteilung zeigen die Messwerte einige Unterschiede auf,
wie zum Beispiel die unsymmetrische Verteilung und die daraus resultierende,
vom Mittelwert abweichende Stelle des Peaks. Jedoch ist der Verlauf der Messdaten für
Zerfallsraten $A > A_{max}$ dem der Poissonverteilung  für $A > \mean{A}$ sehr ähnlich, 
sodass diese zur Näherungsweisen Beschreibung des Verlaufs verwendet werden kann. 