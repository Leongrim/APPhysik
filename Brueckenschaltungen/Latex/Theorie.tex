In der Elektrotechnik sind die Kirchhoff'schen Gesetze essentiell.
\begin{empheq}{equation}
\sum_k I_k = 0
\end{empheq}
lautet die Knotenregel und sagt aus, dass der Betrag aller Ströme, die in einen Punkt laufen gleich dem Betrag der Herauslaufenden sein muss. \\
\\ Die Maschenregel
\begin{empheq}{equation}
\sum_k E_k = \sum_k I_kR_k
\end{empheq}
sagt aus, dass die Summe der Energie, innerhalb einer Masche, gleich der Summe aus Strömen und Widerständen ist.
Weiterhin wichtig für die Wheatstone'sche-, \\ 
die Kapazitätsmess-, die Induktivitätsmess- und die Maxwell-Brücke ist die Gleichung:
\begin{empheq}{equation}
R_x = R_2\dfrac{R_3}{R_4}
\label{eq:Rx}
\end{empheq}
um den Betrag des unbekannten Widerstandes zu ermitteln.\\
Speziell für die Kapazitätsbrücke, um den Wert des unbekannten Kondensators zu berechnen:
\begin{empheq}{equation}
C_x = C_2\dfrac{R_4}{R_3}
\label{eq:Kapazitaet_C}
\end{empheq}
Analog dazu für die Induktivitätsmessbrücke:
\begin{empheq}{equation}
L_x = L_2\dfrac{R_3}{R_4}
\label{eq:Induktivitaet_L}
\end{empheq}
Da die Maxwell-Brücke nicht analog zu den vorherigen Brücken aufgebaut ist, verändert sich dem entsprechend auch die Gleichung zur Bestimmung der Unbekannten.
\begin{empheq}{equation}
L_x = R_2R_3C_4
\label{eq:Induktivitaet_Maxwell_L}
\end{empheq}
Bei der Wien-Robinson-Brücke sind nun alle Teile bekannt und die Frequenz ist veränderlich so ergibt sich für die Brückenspannung $\mathfrak{u}_{Br}$ und die Eingangsspannung $\mathfrak{u}_{S}$:
\begin{empheq}{equation}
\envert{\dfrac{\mathfrak{u}_{Br}}{\mathfrak{u}_{S}}}^2 = \dfrac{1}{9}\dfrac{(\Omega^2 - 1)^2}{(1-\Omega^2)^2+9\Omega^2} = f(\Omega)^{2}
\label{eq:UBr_US}
\end{empheq}
Die theoretische Kreisfrequenz $ \omega_{0,theo} $ ergibt sich aus den Bauteilen der Wien-Robinson-Brücke zu:
\begin{empheq}{equation}
	\omega_{0,theo} = \dfrac{1}{RC}
	\label{eq:Frequenz_0}
\end{empheq} 

Um den Klirrfaktor $k$ zu bestimmen, welcher ein Bewertungskriterium für den Sinusgenerator ist, benötigen wir:
\begin{empheq}{equation}
k := \dfrac{\sqrt{U_2^2 + U_3^2 + \dots + U_{n}^{2\textsl{}}}}{U_1}
\label{eq:Klirrfaktor}
\end{empheq}
Um die Spannung $U_2$ am Ausgang des Generators zu errechnen, benötigt man zusätzlich noch:
\begin{empheq}{equation}
U_2 = \frac{U_{Br}}{f(\Omega)}  
\label{U2} 
\end{empheq}
(mit $\Omega$ = 2).
