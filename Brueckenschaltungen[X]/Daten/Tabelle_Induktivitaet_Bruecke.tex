\begin{table}[!h]
	\centering
	\begin{tabular}{|c|c|c|c|c|}
		\hline
		Induktivität & Widerstand & Quotient & Induktivität & Widerstand\\
		$L_{2}\,[\si{\milli\henry}]$ & $R_{3}\,[\si{\ohm}]$ & $\frac{R_{3}}{R_{4}}$ & $L_{x}\,[\si{\milli\henry}]\,\cref{std:Quotient}$ & $R_{x}\,[\si{\ohm}]\,\cref{std:Quotient}$\\\hline\hline
		\num{20.10(4)}  & \num{305}  & \num{0.437(2)}  & \num{8.78(5)}  & \num{437(13)} \\
		\num{27.50(6)}  & \num{321}  & \num{0.471(2)}  & \num{12.94(7)}  & \num{471(14)} \\
		\hline
	\end{tabular}
	\caption{Werte der Messung einer realen Induktivität mit einer Induktivitätsmessbrücke \label{tab:Induktivitaets_Bruecke}}
\end{table}