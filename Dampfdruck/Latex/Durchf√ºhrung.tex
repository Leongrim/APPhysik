\begin{wrapfigure}[16]{r}{0.4\textwidth}
        \includegraphics[scale=0.4]{Abb3V203.jpg}
        \caption{Schematische Darstellung der Messaparatur aus Versuchsteil 1}
        \label{fig:Abb3}
\end{wrapfigure}
Im ersten Versuchsteil soll der Druck $p \leq 1 Bar$ sein.
Zunächst wird der Druck im Mehrhalskolben auf ca. 40 mBar verringert, 
dies geschieht durch Öffnen des Absperrhahns und Drosselventils und schließen des Belüftungsventils 
an der woulffschen Flasche, sobald dies durch die Unterzuhilfenahme der Wasserstahlpumpe geschehen ist, 
wird der Absperrhahn verschlossen und die Wasserstahlpumpe abgestellt, danach wird auch das Drosselventil geschlossen 
und der Mehrhalskolben erhitzt, dabei wird noch der Rückflusskühler eingestellt, bei ca. 80 \SI{4}{Celsius} wird
der Rückflusskühler langsam gedrosselt, da sonst höhere Temperaturen nur schwer zu erreichen sind.
Temperaturen können jetzt ca alle 2 Grad an den Thermometern und der Druck am Manometer abgelesen werden. \\


\\ Im zweiten Versuchsteil (s. Abb.4) wird der über den Aufbau für 
$p > 1$ Bar bestimmt.
Bei dieser Messapparatur wird, entgegen der Anleitung, lediglich der Stahlbolzen erhitzt und nun wie schon zuvor Druck und Temperatur in bestimmten Abständen abgelesen.
Abbildung 3 wurde leicht verändert, da weder eine Kühlschale existiert, noch wird die Verschraubung gelöst. 
\begin{wrapfigure}[16]{r}{0.7\textwidth}
        \includegraphics[scale=0.4]{Abb4V203.jpg}
        \caption{Schematische Darstellung der Messapparatur aus Versuchsteil 2 Schematische Darstellung der Messapparatur (leicht verändert)}
        \label{fig:Abb4}
\end{wrapfigure}
