Die allgemeine Gasgleichung ist eine fundamentale Gleichung der Thermodynamik 
mit:
\begin{empheq}{equation}
pV = RT
\label{eq:allgGas}
\end{empheq}
R ist die allgemeine Gaskonstante.\\
Für die Auswertung der temperaturabhängigen Verdampfungswärme
wird eine Vereinfachte Form der van-der-Waalsschen Zustandsgleichung 
\begin{empheq}{equation}
	\left(p + \frac{a}{V^2}\right)V = RT
	\label{eq:vanderWaals}
\end{empheq}
mit einer stoffspezifischen Konstante $a$ verwendet,
mit der eine, im Vergleich zur allgemeinen Gasgleichung,
bessere Beschreibung von realen Gasen möglich ist. 

Die Clausius-Clapeyronische Gleichung 
\begin{empheq}{equation}
(V_D - V_F)dp = \frac{L}{T}\cdot dT
\label{eq:Clausius}
\end{empheq}
wird verwendet um Dampfdruckkurve eines Stoffes zu ermitteln,
weiterhin lässt sich sagen, dass die Gleichung nur schwer integrierbar ist, da alle Variablen kompliziert von T abhängen.
Für manche Temperaturbereiche ist die Integration jedoch vereinfacht möglich, damit folgt:
\begin{empheq}{equation}
\ln(p) = -\frac{L}{R}\cdot\frac{1}{T} + const.
\label{eq:pT_ln}
\end{empheq}
bzw.
\begin{empheq}{equation}
p = p_0 \cdot \exp(-\frac{L}{R}\cdot\frac{1}{T})
\label{eq:pT_exp}
\end{empheq}
Für die Auswertung wird 
\begin{empheq}{equation}
L_i := L - L_a
\label{eq:L_i}
\end{empheq}
definiert, wobei $L_a$ die Verdampfungswärme ist, 
die benötigt wird, um das Volumen $V_F$ der Flüssigkeit auf das Volumen des Dampfes $V_D$ auszudehnen. 
