Im Folgenden sind die während des Versuchs aufgenommenen Messwerte und die 
aus diesen bestimmten Größen tabellarisch aufgeführt. An entsprechender Stelle
sind Erklärungen zu den Messwerten und Umrechnungen gegeben.
Aus Gründen der Stringenz, wird hier mit der Auswertung der \fbox{Kalorimetermessung}
begonnen, um die daraus gewonnene Wärmekapazität in der Auswertung, der \fbox{Materialmessung}
verwenden zu können.


\subsection{Bestimmung der Wärmekapazität des Kalorimeters}\label{sec:CM_Kalorimeter}
	In \autoref{tab:DataII} sind die mit dem Thermoelement gemessenen Spannungen, die aus diesen,
	über \eqref{eq:ThermoSpannung}, berechneten Temperaturen und die jeweiligen Massen des kalten,
	des heißen und des gemischten Wassers, für jeden der drei Durchgänge angegeben.
	
	\begin{table}[!h]
		\centering
		\begin{tabular}{|c|c|c|c|}
			\hline
			Durch- &        \multicolumn{3}{c|}{Massen\,[\si{g}]}         \\ \cline{2-4}
			 gang  &      Kalt       &       Heiß       &      Misch      \\
			 Nr.   &     $m_{c}$     &     $ m_{h}$     &     $m_{m}$     \\ \hline
			  1    & \num{304.5(1)}  &  \num{302.9(1)}  & \num{607.4(1)}  \\
			  2    & \num{301.9(1)}  &  \num{307.0(1)}  & \num{608.9(1)}  \\
			  3    & \num{303.6(1)}  &  \num{306.0(1)}  & \num{609.6(1)}  \\ \hline\hline
			Durch- &      \multicolumn{3}{c|}{Spannungen\,[\si{mV}]}      \\ \cline{2-4}
			 gang  &      Kalt       &       Heiß       &      Misch      \\
			 Nr.   &     $U_{c}$     &     $ U_{h}$     &     $U_{m}$     \\ \hline
			  1    &  \num{0.90(1)}  &  \num{4.06(1)}   &  \num{2.30(1)}  \\
			  2    &  \num{0.91(1)}  &  \num{4.10(1)}   &  \num{2.22(1)}  \\
			  3    &  \num{0.92(1)}  &  \num{4.09(1)}   &  \num{2.27(1)}  \\ \hline\hline
			Durch- &     \multicolumn{3}{c|}{Temperaturen\,[\si{°C}]}     \\ \cline{2-4}
			 gang  &      Kalt       &       Heiß       &      Misch      \\
			 Nr.   & $\vartheta_{c}$ & $ \vartheta_{h}$ & $\vartheta_{m}$ \\ \hline
			  1    & \num{21.38(25)} & \num{96.49(24)}  & \num{54.76(25)} \\
			  2    & \num{21.64(24)} & \num{97.52(24)}  & \num{52.76(24)} \\
			  3    & \num{21.88(25)} & \num{97.23(24)}  & \num{54.00(24)} \\ \hline
		\end{tabular}
		\caption{Messwerte der \fbox{Kalorimetermessung} \label{tab:DataII}}
	\end{table}    

	Unter Verwendung der Messwerte aus \autoref{tab:DataII} erhält man durch \eqref{eq:CM_Kalorimeter}
	für jeden Durchgang einen Wert für die Wärmekapazität des Kalorimeters, welche in \autoref{tab:CM_Kalorimeter}
	zu finden sind.
	
	\begin{table}[!h]
		\centering
		\begin{tabular}{|c|c|}	
		\hline
			Durchgang & Wärmekapazität\\
			Nr.		  & $c_{g}m_{g}\,[\si{JK^{-1}}]$\\ \hline
			1& \num{310(26)}    \\    
			2& \num{583(30)}	  \\
			3& \num{453(28)}   \\
			\hline
		\end{tabular}
		\caption{Errechnete Wärmekapazitäten des Kalorimeters}
	\end{table}
	
	
	 
\subsection{Bestimmung der Wärmekapazität der Metalle}\label{sec:WärmekapMetalle}
	
	
	
	Für die Wärmekapazität des Kalorimeters wird der Mittelwert aus den


