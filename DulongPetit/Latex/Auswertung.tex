Im Folgenden sind die während des Versuchs aufgenommenen Messwerte und die 
aus diesen bestimmten Größen tabellarisch aufgeführt. An entsprechender Stelle
sind Erklärungen zu den Messwerten und Umrechnungen gegeben.
Aus Gründen der Stringenz, wird hier mit der Auswertung der \fbox{Kalorimetermessung}
begonnen, um die daraus gewonnene Wärmekapazität in der Auswertung, der \fbox{Materialmessung}
verwenden zu können.


\subsection{Bestimmung der Wärmekapazität des Kalorimeters}\label{sec:CM_Kalorimeter}
	In \autoref{tab:DataII} sind die mit dem Thermoelement gemessenen Spannungen, die aus diesen,
	über \eqref{eq:ThermoSpannung}, berechneten Temperaturen und die jeweiligen Massen des kalten,
	des heißen und des gemischten Wassers, für jeden der drei Durchgänge angegeben.
	
	\begin{table}[!h]
		\centering
		\begin{tabular}{|c|c|c|c|}
			\hline
			        &        \multicolumn{3}{c|}{Massen\,[\si{g}]}         \\ \cline{2-4}
			Messung &      Kalt       &       Heiß       &      Misch      \\
			  Nr.   &     $m_{c}$     &     $ m_{h}$     &     $m_{m}$     \\ \hline\hline
			   1    & \num{304.5(1)}  &  \num{302.9(1)}  & \num{607.4(1)}  \\
			   2    & \num{301.9(1)}  &  \num{307.0(1)}  & \num{608.9(1)}  \\
			   3    & \num{303.6(1)}  &  \num{306.0(1)}  & \num{609.6(1)}  \\ \hline\hline
			        &      \multicolumn{3}{c|}{Spannungen\,[\si{mV}]}      \\ \cline{2-4}
			Messung &      Kalt       &       Heiß       &      Misch      \\
			  Nr.   &     $U_{c}$     &     $ U_{h}$     &     $U_{m}$     \\ \hline\hline
			   1    &  \num{0.90(1)}  &  \num{4.06(1)}   &  \num{2.30(1)}  \\
			   2    &  \num{0.91(1)}  &  \num{4.10(1)}   &  \num{2.22(1)}  \\
			   3    &  \num{0.92(1)}  &  \num{4.09(1)}   &  \num{2.27(1)}  \\ \hline\hline
			        &     \multicolumn{3}{c|}{Temperaturen\,[\si{°C}]}     \\ \cline{2-4}
			Messung &      Kalt       &       Heiß       &      Misch      \\
			  Nr.   & $T_{c}$ & $ T_{h}$ & $T_{m}$ \\ \hline\hline
			   1    & \num{21.38(25)} & \num{96.49(24)}  & \num{54.76(25)} \\
			   2    & \num{21.64(24)} & \num{97.52(24)}  & \num{52.76(24)} \\
			   3    & \num{21.88(25)} & \num{97.23(24)}  & \num{54.00(24)} \\ \hline
		\end{tabular}
		\caption{Messwerte zur Bestimmung der Wärmekapazität des Kalorimeters \label{tab:DataII}}
	\end{table}    

	Unter Verwendung der Messwerte aus \autoref{tab:DataII} erhält man durch \eqref{eq:CM_Kalorimeter}
	für jeden Durchgang einen Wert für die Wärmekapazität des Kalorimeters, welche in \autoref{tab:CM_Kalorimeter}
	zu finden sind.
	
	\begin{table}[!h]
		\centering
		\begin{tabular}{|c|c|}	
		\hline
			Messung   & Wärmekapazität\\
			Nr.		  & $c_{g}m_{g}\,[\si{JK^{-1}}]$\\ \hline \hline
			1& \num{310(26)}    \\    
			2& \num{583(30)}	  \\
			3& \num{453(28)}   \\
			\hline
		\end{tabular}
		\caption{Errechnete Wärmekapazitäten des Kalorimeters \label{tab:CM_Kalorimeter}}
	\end{table}
	
	Für die Auswertung in \autoref{sec:C_Metalle} wird das Mittel
	dieser Werte $\mean{c_{g}m_{g}} = \SI{450(60)}{JK^{-1}}$ verwendet.
	Dabei ist der angegeben Fehler die Standardabweichung des Mittelwerts
	und nicht aus der Fehlerfortpflanzung berechnet.
	
	 
\subsection{Bestimmung der Wärmekapazität der Metalle}\label{sec:C_Metalle}
	
	Die Messwerte der Messungen für die Berechnung der Wärmekapazitäten 
	von Kupfer (\ce{Cu}) und Aluminium (\ce{Al}) sind in \autoref{tab:DataI_Cu}
	und \ref{tab:DataI_Al} zu finden. Dabei ist die jeweilig angegebene Temperatur
	$T_{c}$, die des Wassers im Kalorimeter, $T_{h}$, die des
	entsprechenden Metallzylinders und $T_{m}$ die des Wassers nach dem
	eintauchen des Zylinders.    
	
	\begin{table}[!h]
		\centering
		\begin{tabular}{|c|c|c|c|}
			\hline
			        &      \multicolumn{3}{c|}{Spannungen\,[\si{mV}]}      \\ \cline{2-4}
			Messung &      Kalt       &       Heiß       &      Misch      \\
			  Nr.   &     $U_{c}$     &     $ U_{h}$     &     $U_{m}$     \\ \hline\hline
			   1    &  \num{0.78(1)}  &  \num{4.06(1)}   &  \num{0.95(1)}  \\
			   2    &  \num{0.95(1)}  &  \num{4.08(1)}   &  \num{1.15(1)}  \\
			   3    &  \num{0.93(1)}  &  \num{4.08(1)}   &  \num{1.10(1)}  \\ \hline\hline
			        &     \multicolumn{3}{c|}{Temperaturen\,[\si{°C}]}     \\ \cline{2-4}
			Messung &      Kalt       &       Heiß       &      Misch      \\
			  Nr.   & $T_{c}$ & $ T_{h}$ & $T_{m}$ \\ \hline\hline
			   1    & \num{18.61(25)} & \num{97.65(24)}  & \num{22.67(25)} \\
			   2    & \num{22.82(24)} & \num{97.85(24)}  & \num{27.62(24)} \\
			   3    & \num{22.33(25)} & \num{97.91(24)}  & \num{26.38(25)} \\ \hline
		\end{tabular}
		\caption{Messwerte der Messung mit Kupfer \label{tab:DataI_Cu}}
	\end{table}   
	 
	\begin{table}[!h]
		\centering
		\begin{tabular}{|c|c|c|c|}
				\hline
				        &      \multicolumn{3}{c|}{Spannungen\,[\si{mV}]}      \\ \cline{2-4}
				Messung &      Kalt       &       Heiß       &      Misch      \\
				  Nr.   &     $U_{c}$     &     $ U_{h}$     &     $U_{m}$     \\ \hline
				   1    &  \num{0.85(1)}  &  \num{4.06(1)}   &  \num{1.05(1)}  \\
				   2    &  \num{0.89(1)}  &  \num{4.09(1)}   &  \num{0.93(1)}  \\
				   3    &  \num{0.95(1)}  &  \num{4.09(1)}   &  \num{1.06(1)}  \\ \hline\hline
				        &     \multicolumn{3}{c|}{Temperaturen\,[\si{°C}]}     \\ \cline{2-4}
				Messung &      Kalt       &       Heiß       &      Misch      \\
				  Nr.   & $T_{c}$ & $ T_{h}$ & $T_{m}$ \\ \hline
				   1    & \num{20.37(25)} & \num{97.51(24)}  & \num{25.32(25)} \\
				   2    & \num{21.39(24)} & \num{98.31(24)}  & \num{22.31(24)} \\
				   3    & \num{22.86(25)} & \num{98.16(24)}  & \num{25.54(25)} \\ \hline
		\end{tabular}
		\caption{Messwerte der Messung mit Aluminium \label{tab:DataI_Al}}
	\end{table} 

		

\newpage
	Da das Wasser im Kalorimeter einmal ausgewechselt wurde, wird im weiteren der Mittelwert
	der verwendeten Volumina  $\mean{V_{\ce{H2O}}} = \SI{574.35(7)}{cm^{3}} $ verwendet. 
	Durch wiegen wurden die Massen des Kupfer- und des Aluminiumzylinders zu $m_{\ce{Cu}} = \SI{378.3(1)}{g}$
	und $m_{\ce{Al}} = \SI{254.6(1)}{g}$ bestimmt.

	Aus den Werten in \autoref{tab:DataI_Cu} und \ref{tab:DataI_Al}, der Wärmekapazität des Kalorimeters $\mean{c_{g}m_{g}} = \SI{450(60)}{JK^{-1}}$
	aus \autoref{sec:CM_Kalorimeter}, den Massen der Metallzylinder und der spezifischen Wärmekapazität von Wasser $c_{\ce{H2O}} = \SI{4.18}{Jg^{-1}K^{-1}}$ \cite{V201},
	lassen sich nun mit Hilfe von \eqref{eq:C_Metall} die spezifischen Wärmekapazitäten der beiden Metalle bestimmen.
	In \autoref{tab:Cp_Metalle} sind sowohl die spezifische Wärmekapazität bezogen auf ein Gramm als auch mit Bezug auf ein Mol angegeben, wobei zur Umrechnung
	die molaren Massen aus \autoref{tab:Konstanten} verwendet wurden. 
	
	\begin{table}[!h]
		\centering
		\begin{tabular}{|c|c|c|}
			\hline
			Messung   &  spezifische Wärmekapazität &  spezifische Wärmekapazität\\
			Nr.		  & $c_{\ce{Cu}}\,[\si{Jg^{-1}K^{-1}}]$ & $c_{\ce{Al}}\,[\si{Jg^{-1}K^{-1}}]$ \\ \hline \hline
			1& \num{0.649(60)} & \num{1.704(131)} \\    
			2& \num{0.821(63)} & \num{0.302(114)} \\
			3& \num{0.679(62)} & \num{0.917(123)} \\
			\hline \hline
			Messung   &  spezifische Wärmekapazität &  spezifische Wärmekapazität\\
			Nr.		  & $c_{\ce{Cu}}\,[\si{\joule\per\mol\per\kelvin}]$ & $c_{\ce{Al}}\,[\si{\joule\per\mol\per\kelvin}]$ \\ \hline \hline
			1& \num{41.3(38)} & \num{46.0(35)} \\    
			2& \num{52.1(40)} & \num{8.2(31)} \\
			3& \num{43.2(40)} & \num{24.8(34)} \\
			\hline
		\end{tabular}
		\caption{Spezifische Wärmekapazitäten von Kupfer und Aluminium \label{tab:Cp_Metalle}}
	\end{table}
	
	
	
	Durch \eqref{eq:Cp_Cv} erhält man, aus den Wärmekapazitäten bei konstantem Druck in \autoref{tab:Cp_Metalle}, die gesuchten 
	Wärmekapazitäten bei konstantem Volumen. Die für diese Umrechnung benötigten Konstanten des Kupfers und des Aluminiums sind 
	in \autoref{tab:Konstanten} zu finden. Die Molvolumen der Metalle wurden dabei aus ihrer molaren Masse und der jeweiligen Dichte
	bestimmt. Als Temperatur wurden die Mischtemperaturen $T_{m}$ aus \autoref{tab:DataI_Cu} und \ref{tab:DataI_Al} verwendet.
	\begin{table}[!h]
		\centering
		\begin{tabular}{|c||c|c|}
			\hline
			           Material-            & Kupfer (\ce{Cu}) & Aluminium (\ce{Al}) \\
			konstanten\cite{Mende09, V201}  &                  &  \\ \hline\hline
			            Dichte              &                  &  \\
			    $\rho\,[\si{gcm^{-3}}]$     &    \num{8.96}    &     \num{2.70}      \\ \hline
			           Molmassen            &                  &  \\
			    $M\,[\si{\g\per\mol}]$      &    \num{63.5}    &     \num{27.0}      \\ \hline
			          Molvolumen            &                  &  \\
			$V_{0}\,[\si{cm^{3}mol^{-1}}]$  &    \num{7.09}    &     \num{10.0}      \\ \hline
			Ausdehnungskoeff.\footnotemark  &                  &  \\
			$\alpha\,[\si{10^{-6}K^{-1}}]$  &    \num{16.8}    &     \num{23.5}      \\ \hline
			       Kompressionsmodul        &                  &  \\
			$\kappa\,[\si{10^{9} Nm^{-2}}]$ &    \num{136}     &      \num{75}       \\ \hline
			spez. Wärmekapazität Literatur & &\\
			$c_{lit}\,[\si{\joule\per\mol\per\kelvin}]$ & \num{24.3205} & \num{24.192} \\ \hline
		\end{tabular}
		\caption{Materialkonstanten von Kupfer und Aluminium \label{tab:Konstanten}}
	\end{table}
	\footnotetext{linearer Ausdehnungskoeffizient}

	Die daraus erhaltenen Werte, der Vergleich zwischen diesen und dem Literaturwert aus \autoref{tab:Konstanten}
	sowie die relative Abweichung der Messwerte von dem, durch das Dulong-Petitsche Gesetz vorhergesagte Wert
	$C_{V} = 3R$, wobei $R = \SI{8.314}{\joule\per\mol\per\kelvin}$ \cite{SciPy} ist, 
	sind in \autoref{tab:Cv_Metalle} zu finden.
	
	\begin{table}[!h]
		\centering
		\begin{tabular}{|c|c|c|c|}
			\hline
			Messung & spezifische Wärmekapazität & Relative Abweichung & Literaturvergleich  \\
			Nr.		& $ C_{V,\ce{Cu}}\,[\si{\joule\per\mol\per\kelvin}]  $ &$ \frac{\envert{C_{V,\ce{Cu}}-3R}}{3R}$ & 
			$\frac{ C_{V,\ce{Cu}}}{C_{V,lit}}$\\ \hline \hline
			1 &\num{35.7(37)}& \num{0.43}& \num{1.47} \\
			2 &\num{45.3(40)}& \num{0.82}& \num{1.86} \\
		    3 &\num{36.7(39)}& \num{0.47}& \num{1.51} \\ \hline \hline
	    	Messung & spezifische Wärmekapazität & Relative Abweichung & Literaturvergleich \\
   			Nr.		& $ C_{V,\ce{Al}}\,[\si{\joule\per\mol\per\kelvin}]  $ & $ \frac{ \envert{C_{V,\ce{Al}} - 3R}}{3R}$ &
   			$\frac{ C_{V,\ce{Cu}}}{C_{V,lit}}$\\ \hline \hline
   			1 &\num{36.6(34)}& \num{0.47}& \num{1.51} \\
   			2 &\num{-0.2(30)}& \num{1.01}& \num{-0.01}\\
	   	    3 &\num{15.3(33)}& \num{0.39}& \num{0.63} \\ \hline
		    
		\end{tabular}
		\caption{Vergleich der Wärmekapazitäten von Kupfer und Aluminium \label{tab:Cv_Metalle}}
	\end{table}


