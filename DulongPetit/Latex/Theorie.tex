Die Kernaussage des Dulong-Petitschen Gesetzes ist:
\[
C_v = (\frac{dQ}{dT})_v = (\frac{dU}{dT})_v
\]
Für das klassische System geht dann hervor, dass
\[
C_v = 3R
\]
gilt, wobei $R$ die allgemeine Gaskonstante ist.
Nun wollen wir prüfen, ob auch die Wärmekapazität quantenmechanisch betrachtet werden muss, da angenommen wurde, dass beliebig kleine Wärmemengen abgegeben werden können, dies steht im Widerspruch zur Quantenmechanik.
Für $T \longrightarrow \infty$ gilt jedoch,
\[
\langle U \rangle \approx 3RT
\]
was für eine Gültigkeit der klassischen Physik spricht.
Weiterhin benötigen wir $C_p$, dies kann bestimmt werden aus:
\[
C_p - C_v = 9\alpha ^2\kappa V_0T
\]
Mit dem linearen Ausdehnungskoeffizienten $\alpha$, dem Kompressionsmodul $\kappa$ und $V_0$ dem Molvolumen, wobei durch
\[
\kappa = V\left(\frac{\partial p}{\partial V} \right)_T
\] 
gegeben ist.
Um nun die spezifische Wärmekapazität der Probekörper bestimmen zu können gilt:
\[
c_k = \frac{(c_wm_w + c_gm_g)(T_m - T_w)}{m_k(T_k - T_m)}
\]
$c_g$ und $m_g$ ist Wärmekapazität und Masse des Kalorimeters, Index $w$ für das Wasser und Index $k$ für den Probekörper.
Für das Kalorimeter gilt:
\[
c_gm_g = \frac{c_wm_y(T_y - T'_m) - c_wm_x(T'_m - T_x)}{T'_m - T_x}
\]
$m_x$ und $m_y$ sind die Massen der beiden Wassermengen, die im zweiten Versuchsteil zusammen gemischt werden sollen und $T_x$, $T_y$ die dazu gehörigen Temperaturen, wobei gilt $T_x < T_y$. \\
$T'_m$ ist die Mischtemperatur der beiden Wassermengen $T_x$, $T_y$.
Um die Temperatur aus der ausgegebenen Spannung des Digitalvoltmeters bestimmen zu können gilt:
\[
T = 25,257U_{th} - 0.19U_{th}^2
\]
