Die Kernaussage des Dulong-Petitschen Gesetzes ist:
\begin{empheq}{equation} 
C_v = \biggl(\frac{dQ}{dT}\biggr)_v = \biggl(\frac{dU}{dT}\biggr)_v
\end{empheq}
Für das klassische System geht dann hervor, dass
\begin{empheq}{equation} 
C_v = 3R
\end{empheq}
gilt, wobei $R$ die allgemeine Gaskonstante ist.\\
Nun wollen wir prüfen, ob auch die Wärmekapazität quantenmechanisch betrachtet werden muss, da angenommen wurde, dass beliebig kleine Wärmemengen abgegeben werden können, dies steht im Widerspruch zur Quantenmechanik.
Für $T \longrightarrow \infty$ gilt jedoch,
\begin{empheq}{equation} 
\langle U \rangle \approx 3RT
\end{empheq}
was für die Gültigkeit des, aus der klassischen Physik hergeleiteten, Gesetzes spricht.
Weiterhin benötigen wir $C_p$, dies kann bestimmt werden aus:
\begin{empheq}{equation} 
C_p - C_v = 9\alpha ^2\kappa V_0T
\label{eq:Cp_Cv}
\end{empheq}
Mit dem linearen Ausdehnungskoeffizienten $\alpha$, dem Kompressionsmodul $\kappa$ und $V_0$ dem Molvolumen, des 
untersuchten Materials.\\

Um nun die spezifische Wärmekapazität der Probekörper bestimmen zu können gilt:
\begin{empheq}{equation} 
c_k = \frac{(c_{\ce{H2O}}m_{\ce{H2O}} + c_gm_g)(T_m - T_c)}{m_k(T_h - T_m)}
\label{eq:C_Metall}
\end{empheq}
$c_gm_g$ ist dabei die Wärmekapazität des Kalorimeters, Index $\ce{H2O}$  steht für das Wasser und Index $k$ für den Probekörper.
Die Temperaturen $T$ mit den Indizes $c$ (\enquote{cold}) und $h$ (\enquote{hot}) stehen in dieser Reihenfolge für die Temperatur des Wassers und des Probekörpers. 
Für das Kalorimeter gilt:
\begin{empheq}{equation} 
c_gm_g = \dfrac{c_{\ce{H2O}} m_h(T_h - T'_m) - c_{\ce{H2O}}m_c(T'_m - T_c)}{T'_m - T_c}
\label{eq:CM_Kalorimeter}
\end{empheq}
$m_c$ und $m_h$ sind die Massen der beiden Wassermengen, die im zweiten Versuchsteil zusammen gemischt werden sollen und $T_c$, $T_h$ die dazu gehörigen Temperaturen, wobei gilt $T_c < T_h$. 
$T'_m$ ist die Mischtemperatur der beiden Wassermengen $m_{c}$, $m_{h}$.
Um die Temperatur aus der ausgegebenen Spannung des Digitalvoltmeters bestimmen zu können gilt:
\begin{empheq}{equation} 
T = \num{25.157}\, U_{\vartheta} - \num{0.19}\,U_{\vartheta}^2
\label{eq:ThermoSpannung}
\end{empheq}

