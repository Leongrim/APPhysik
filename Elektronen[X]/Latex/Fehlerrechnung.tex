\renewcommand{\theerrorEquation}{\Roman{errorEquation}}

In diesem Abschnitt sind die zur Berechnung der Fehler in \cref{sec:Auswertung} 
verwendeten Fehlergleichungen aufgelistet, die mittels der Gaußschen Fehlerfortpflanzung
berechnet wurden.\\

Den Fehler der berechneten Sinusfrequenz $f_{sin}$ erhält man vereinfacht durch:
\begin{errorEquation}
	\sigma_{f_{sin}} = n \cdot \sigma_{f_{sz}}
\end{errorEquation}

Der Fehler der magnetischen Flussdichte $B_{d}$ berechnet sich durch die Gleichung:
\begin{errorEquation}
	\sigma_{B_d}=\frac{8}{\sqrt{125}} \frac{\mu_{0} N}{R} \sigma_{I}
\end{errorEquation}

Der Fehler der spezifischen Ladung $e_{spez} = \frac{e_{0}}{m_{e}}$ ergibt sich aus:
\begin{errorEquation}
	\sigma_{e_{spez}}= 8\sqrt{\gamma^{4} \sigma_{U_{b}}^{2} + 4 \gamma^{2} U_{b}^{2} \sigma_{\gamma}^{2}}
\end{errorEquation}

Den Fehler des totalen Magnetfelds $B_{total}$ der Erde wurde bestimmt durch:
\begin{errorEquation}
	\sigma_{B_{total}} = \frac{\sigma_{B_{hor}}}{\Cos{\varphi}}
\end{errorEquation}