\begin{table}[!h]
	\centering
	\begin{tabular}{|c|c|c|c|}
		\hline
		Frequenzen & Gemessene Amplitude & Berechnete Amplitude & relative Abweichung \\ 
		$\nu\,[\si{\hertz}]$ & $b_{n}\,[\si{\volt}]$ & $b_{n,theo}\,[\si{\volt}]$ & $\envert{1 - \tfrac{b_{n}}{b_{n,theo}}}$  \\\hline\hline
		\num{100}  & \num{1.80}  & \num{2.55}  & \num{0.29}  \\ 
		\num{300}  & \num{0.60}  & \num{0.85}  & \num{0.29}  \\ 
		\num{500}  & \num{0.34} & \num{0.51}  & \num{0.33} \\
		\num{700}  & \num{0.25} & \num{0.36}  & \num{0.31} \\ 
		\num{900}  & \num{0.19} & \num{0.28}  & \num{0.32} \\ 
		\num{1100} & \num{0.14} & \num{0.23}  & \num{0.39} \\\hline
	\end{tabular}
	\caption{Gemessene und Berechnete Amplituden der Oberschwingung\\ \hspace*{2.1cm}und deren relative Abweichung für die Rechteckspannung \label{tab:Analyse1}}
\end{table}