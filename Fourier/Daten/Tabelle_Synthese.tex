\begin{table}[!h]
	\centering
	\begin{tabular}{|c|c|c|c|}
		\hline
		Oberwelle & Rechteck Amplitude & Dreieck Amplitude & Sägezahn Amplitude\\
		$n$ & $b_{n,r}\,[\si{\volt}]$ & $b_{n,d}\,[\si{\volt}]$ & $b_{n,s}\,[\si{\volt}]$\\\hline\hline
		\num{1} & \num{0.80}  & \num{0.80}  & \num{0.80} \\
		\num{2} & -  & -  & \num{0.40} \\
		\num{3} & \num{0.27}  & \num{0.09}  & \num{0.27} \\
		\num{4} & -  & -  & \num{0.20} \\
		\num{5} & \num{0.16}  & \num{0.03}  & \num{0.16} \\
		\num{6} & -  & -  & \num{0.13} \\
		\num{7} & \num{0.11}  & \num{0.02}  & \num{0.11} \\
		\num{8} & -  & -  & \num{0.10} \\
		\num{9} & \num{0.09}  & \num{0.01}  & \num{0.09} \\
		\num{10} & -  & -  & \num{0.08} \\
		\hline
	\end{tabular}
	\caption{Zur Synthese verwandte Amplituden der ersten 10 Oberwellen \label{tab:Synthese}}
\end{table}