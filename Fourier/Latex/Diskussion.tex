Im folgenden Abschnitt werden die in der Auswertung gewonnen Ergebnisse
noch einmal abschließend diskutiert und auf ihre Plausibilität hin untersucht.
Dabei werden, falls erforderlich, auch Versuchsauswertung und Durchführung
mit den erhaltenen Ergebnissen in Bezug gesetzt.\\

Die in  \cref{sec:Auswertung_Analyse} gemessenen Amplituden der ersten 
Oberschwingungen weisen trotz großer Abweichungen zu den theoretischen
Werten einen plausiblen und richtigen Trend auf. Bei Betrachtung der 
Amplituden der Rechteck- und Sägezahnspannung \cref{tab:Analyse1} und 
\ref{tab:Analyse3} ist festzustellen, dass die Mehrzahl der Abweichungen 
im Bereich von \SI{25}{\percent} bis \SI{35}{\percent}. Durch diese
eher geringe Streuung der Abweichungen ist anzunehmen, dass es sich um
einen mit der Durchführung des Versuchs zu begründenden systematischen
Fehler handelt. 
So wird die Fouriertransformation (FFT) des Signals nur auf den Bereich
ausgeführt der auf dem Oszilloskop angezeigt wird, wohingegen die 
theoretischen Werte die Koeffizienten einer unendlichen Summe(Reihe) 
darstellen.\\
Die Größere Abweichungen für die höheren Frequenzen lassen sich ebenfalls 
dadurch Begründen, da der so gemachte Fehler für kleine Amplituden 
relativ größer ist.\\
Die teils sehr großen Abweichungen der Amplituden in \cref{tab:Analyse2} 
lassen sich ähnlich Begründen, da die Koeffizienten im Gegensatz zu den 
andren Spannungen bei der Dreiecksspannung mit $n^{-2}$ abfallen 
werden diese wie in \cref{tab:Analyse2} zu sehen schnell klein, wodurch wiederum 
der relative Fehler der gemessen Werte wächst.\\

Die aus den berechneten Werten in \cref{tab:Synthese} erstellten Spannungen,
zeigen wie in \cref{fig:Recht}, \ref{fig:Drei} und \ref{fig:Säge} zu sehen 
einen erkennbaren und somit plausiblen Verlauf, wobei vor allem die Dreieckspannung
als einziges stetiges Signal schon mit $5$ Oberschwingungen eine sehr klare Darstellung 
besitzt. Dies liegt wie bereist gesagt, an dem schnellen Abfall der Koeffizienten, sodass 
höhere Oberschwingungen nur noch geringe Auswirkungen auf den Kurvenverlauf haben.
Bei den anderen beiden Spannungen lassen sich noch größere Abweichungen von dem gewünschten 
Verlauf feststellen, dies liegt jedoch einerseits an dem wenige starken Abfall der 
Koeffizienten und andererseits an der Unstetigkeit der beiden Signale und dem daraus folgenden
Gibbschen Phänomen, welches zu einem Überschwingen an eben diesen Unstetigkeits-/Sprungstellen
führt.          