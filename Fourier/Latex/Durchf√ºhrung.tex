\subsection{Fourier-Analyse}
Der Versuchsaufbau für den ersten Versuchsteil besteht aus einem Funktionengenerator und einem Digitaloszilloskop. \\
Bei der Fourier-Analyse wird vom Funktionsgenerator jeweils eine Dreieck-, Rechteck- und Sägezahnspannung generiert. Die erzeugten Spannungen werden vom Oszilloskop mittels einer Fast-Fourier-Transformation (FFT) in das Frequenzspektrum umgerechnet und angezeigt. Die Amplituden und Frequenzen können am Oszilloskop abgelesen werden.
\subsection{Fourier-Synthese}
Für den Aufbau wird weiterhin das Oszilloskop verwendet, dazu kommt der Oberwellengenerator.
Bei der Fourier-Synthese werden mit dem Oberwellengenerator bis zu zehn Sinus-Schwingungen mit ganzzahligen Frequenzverhältnissen so eingestellt, dass die Amplitudenverhältnisse und Phasen in Relation zur Grundschwingung mit denen der Errechneten übereinstimmen. \\
Die Phasen werden über die Lissajous-Figuren des Phasenverhältnisses der Grundschwingung und der jeweiligen Oberschwingung eingestellt. Alle Phasen bei ungeraden Fourier-Amplituden werden auf $\phi = 0$ ausgerichtet und die geraden auf $\phi = \pi/2$.
Anschließend wird die Summenspannung auf dem Oszilloskop ausgegeben.
