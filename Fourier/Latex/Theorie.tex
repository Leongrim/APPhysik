\begin{equation} 
f(t) = \frac{a_0}{2} + \sum_{n=1}^\infty (a_n cos(n\frac{2\pi}{T}t) + b_n sin(n\frac{2\pi}{T}t)) 
\end{equation}
Mithilfe von Fourier-Reihen (1) lassen sich periodische Funktionen sehr gut approximieren, wobei T die Periodendauer und $a_n$ bzw. $b_n$ die Fourier-Koeffizienten sind. Eine Einschränkung ist hierbei das Gibb'sche Phänomen, es tritt an Unstetigkeiten der ursprünglichen Funktion auf. An diesen Stellen tritt eine endlich große Abweichung auf, welche durch erhöhte Anzahl der Polynome auch nicht verschwindet. \\
Für die Fourier-Koeffizienten $a_n$ und $b_n$ gilt:
\begin{equation}
a_n = \frac{T}{2}\int_{0}^{T} f(t) \cos(n\frac{2\pi}{T}t) dt
\end{equation}
und
\begin{equation}
b_n = \frac{T}{2}\int_{0}^{T} f(t) \sin(n\frac{2\pi}{T}t) dt
\end{equation}
Die Fourier-Transformation einer Funktion $f(t)$ ist gegeben durch:
\begin{equation}
g(v) = \int_{-\infty}^{\infty} f(t) e^{ivt} dt.
\end{equation}
Mithilfe dieser Transformation lässt sich das Frequenzspektrum der Funktion direkt bestimmen. Dabei erhält man für eine bestimme Frequenz $v$ die entsprechende Amplitude.
