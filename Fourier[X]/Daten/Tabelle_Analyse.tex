\begin{table}[!h]
	\centering
	\begin{tabular}{|c|c|c|c|c|}
		\hline
		Frequenzen & Gemessene Amplitude & Berechnete Amplitude & Gemessene Amplitude & Berechnete Amplitude\\
		$\nu\,[\si{\hertz}]$ & $b_{n}\,[\si{\volt}]$ & $b_{n}\,[\si{\volt}]$ & $b_{n}\,[\si{\volt}]$ & $b_{n}\,[\si{\volt}]$\\\hline\hline
		\num{100.000}  & \num{1.800}  & \num{1.800}  & \num{1.160}  & \num{1.160} \\
		\num{300.000}  & \num{0.600}  & \num{0.600}  & \num{0.160}  & \num{0.129} \\
		\num{500.000}  & \num{0.340}  & \num{0.360}  & \num{0.080}  & \num{0.046} \\
		\num{700.000}  & \num{0.250}  & \num{0.257}  & \num{0.046}  & \num{0.024} \\
		\num{900.000}  & \num{0.190}  & \num{0.200}  & \num{0.040}  & \num{0.014} \\
		\num{1100.000}  & \num{0.140}  & \num{0.164}  & \num{0.000}  & \num{0.000} \\
		\hline
	\end{tabular}
	\caption{Gemessene und Berechnete Amplituden der Oberschwingung für Recht- und Dreieckspannung \label{tab:Analyse1}}
\end{table}