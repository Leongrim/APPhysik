Bei der Fourier-Analyse werden bei verschiedenen Spannungen (hier: Dreieck-, Rechteck- und Sägezahnspannung) die Fourier-Koeffizienten vermessen und mit den vorher ermittelten, theoretischen Werten verglichen. Anschließend werden aus errechneten Oberschwingungen verschiedene Spannungen synthetisiert.
