Im folgenden Abschnitt sind die während des Versuchs aufgenommenen Daten 
und die daraus berechneten Ergebnisse tabellarisch und graphisch dargestellt.
An entsprechender Stelle sind Erklärungen zu den Berechnungen
und Ergebnissen gegeben. Die Fehler der Messwerte wurden allgemein als die 
kleinste Skaleneinheit des jeweiligen Messgeräts angenommen.

\subsection{Fehlerrechnung}\label{sec:Fehlerrechnung}
	
	In diesem Abschnitt sind die in der Auswertung zur Berechnung der
	Messfehler verwendeten Fehlergleichungen zu finden. Diese wurden
	mit Hilfe der gaußschen Fehlerrechnung bestimmt.\\
	
	Der allgemeine gaußsche Fehler eines Mittelwerts ergibt sich nach
	\begin{errorEquation}
		\sigma_{M} = \dfrac{1}{n} \sqrt{\sum_{i = 1}^{n} \sigma_{x_{i}}^{2}}.
		 \label{std:Mittel}
	\end{errorEquation}
	
	Den Fehler des Dampfdruckes $p_{sat}$ erhält man vereinfacht aus
	\begin{errorEquation}
		\sigma_{p} = \num{3.8e11}  \cdot \dfrac{\Exp{\frac{-6876}{T}}}{T^{2}} \cdot \sigma_{T}.
	    \label{std:Dampfdruck}
	\end{errorEquation}  
	
	Für die mittlere freie Wegstrecke $\bar{w}$ erhält man den Fehler durch
	\begin{errorEquation}
		\sigma_{\bar{w}} = \dfrac{0.0029 \sigma_{p_{sat}}}{p_{sat}^{2}} .
    \label{std:Wegstrecke}
	\end{errorEquation}  

	Die Wellenlänge $\lambda$ hat den Fehler
	\begin{errorEquation}
		\sigma_{\lambda} = \dfrac{hc \sigma_{E}}{E^{2}} .
		 \label{std:Wellenlänge}
	\end{errorEquation}  
	

\subsection{Messung der integralen Energieverteilung bei Raumtemperatur}
	
	Die Mittels XY-Schreiber aufgenommenen Kurve der Integralen Energieverteilung ist 
	in \cref{fig:Auswertung_Int_Energie_Verteilung_20C} dargestellt. Bei der Temperatur
	$T =\SI{25.0(1)}{\degreeCelsius} = \SI{298.2(1)}{\kelvin}$ herrscht nach \cref{eq:Theorie_Dampfdruck} in der 
	Franck-Hertz-Röhre der Dampfdruck $p_{sat} = \SI{0.00530(4)}{\milli\bar}$. Durch \cref{eq:Theorie_Weglaenge}
	lässt sich daraus die mittlere freie Weglänge der Elektronen zu $\overline{w} = \SI{0.547(4)}{\cm}$ berechnen.
	
	\includeFigure[scale=0.6]{Grafiken/Raumtemperatur}{Kurve der Integralen Energieverteilung bei Raumtemperatur}{\label{fig:Auswertung_Int_Energie_Verteilung_20C}} 
	
	Durch grafische Differentiation, der integralen Energieverteilung erhält man die in
	\cref{fig:Auswertung_Diff_Energie_Verteilung_20C} dargestellte differentielle Energieverteilung
	der Elektronen. \\
	
	\noindent
	Die dafür aus \cref{fig:Auswertung_Int_Energie_Verteilung_20C} genommenen Werte
	befinden sich in \cref{tab:Auswertung_Diff_Energie_Verteilung_20C}. Der Wert des Auffängerstoms
	ist dabei in der Anzahl der Kästchen über der x-Achse angegeben. 
	
	\begin{table}[!h]
	\centering
	\begin{adjustbox}{width=1\textwidth, center}
	\begin{tabular}{|c|c|c|c|c|c|c|c|}
		\hline
		Bremsspannung & $\Delta$ Bremsspannung & Auffängerstrom & $\Delta$ Auffängerstrom & Bremsspannung & $\Delta$ Bremsspannung & Auffängerstrom & $\Delta$ Auffängerstrom\\
		$U_{A}$ [\si{\volt}] & $\Delta U_{A}$ [\si{\volt}] & $\propto I_{A}(U_{A})$ & $\propto \Delta I_{A}(U_{A})$ & $U_{A}$ [\si{\volt}] & $\Delta U_{A}$ [\si{\volt}] & $\propto I_{A}(U_{A})$ & $\propto \Delta I_{A}(U_{A})$\\
\hline\hline
		\num{0.00(5)} & \num{0.25(7)} & \num{8.4(1)} & \num{0.0(1)} & \num{5.50(5)} & \num{0.25(7)} & \num{8.1(1)} & \num{0.1(1)}\\
		\num{0.25(5)} & \num{0.25(7)} & \num{8.4(1)} & \num{0.0(1)} & \num{5.75(5)} & \num{0.25(7)} & \num{8.0(1)} & \num{0.1(1)}\\
		\num{0.50(5)} & \num{0.25(7)} & \num{8.4(1)} & \num{0.0(1)} & \num{6.00(5)} & \num{0.25(7)} & \num{7.9(1)} & \num{0.1(1)}\\
		\num{0.75(5)} & \num{0.25(7)} & \num{8.4(1)} & \num{0.0(1)} & \num{6.25(5)} & \num{0.25(7)} & \num{7.8(1)} & \num{0.0(1)}\\
		\num{1.00(5)} & \num{0.25(7)} & \num{8.4(1)} & \num{0.0(1)} & \num{6.50(5)} & \num{0.25(7)} & \num{7.8(1)} & \num{0.1(1)}\\
		\num{1.25(5)} & \num{0.25(7)} & \num{8.4(1)} & \num{0.0(1)} & \num{6.75(5)} & \num{0.25(7)} & \num{7.7(1)} & \num{0.1(1)}\\
		\num{1.50(5)} & \num{0.25(7)} & \num{8.4(1)} & \num{0.0(1)} & \num{7.00(5)} & \num{0.25(7)} & \num{7.6(1)} & \num{0.1(1)}\\
		\num{1.75(5)} & \num{0.25(7)} & \num{8.4(1)} & \num{0.0(1)} & \num{7.25(5)} & \num{0.25(7)} & \num{7.5(1)} & \num{0.2(1)}\\
		\num{2.00(5)} & \num{0.25(7)} & \num{8.4(1)} & \num{0.0(1)} & \num{7.50(5)} & \num{0.25(7)} & \num{7.3(1)} & \num{0.1(1)}\\
		\num{2.25(5)} & \num{0.25(7)} & \num{8.4(1)} & \num{0.0(1)} & \num{7.75(5)} & \num{0.25(7)} & \num{7.2(1)} & \num{0.2(1)}\\
		\num{2.50(5)} & \num{0.25(7)} & \num{8.4(1)} & \num{0.1(1)} & \num{8.00(5)} & \num{0.20(7)} & \num{7.0(1)} & \num{0.2(1)}\\
		\num{2.75(5)} & \num{0.25(7)} & \num{8.3(1)} & \num{0.0(1)} & \num{8.20(5)} & \num{0.20(7)} & \num{6.8(1)} & \num{0.2(1)}\\
		\num{3.00(5)} & \num{0.25(7)} & \num{8.3(1)} & \num{0.0(1)} & \num{8.40(5)} & \num{0.20(7)} & \num{6.5(1)} & \num{0.3(1)}\\
		\num{3.25(5)} & \num{0.25(7)} & \num{8.3(1)} & \num{0.0(1)} & \num{8.60(5)} & \num{0.20(7)} & \num{6.2(1)} & \num{0.4(1)}\\
		\num{3.50(5)} & \num{0.25(7)} & \num{8.3(1)} & \num{0.0(1)} & \num{8.80(5)} & \num{0.20(7)} & \num{5.8(1)} & \num{0.6(1)}\\
		\num{3.75(5)} & \num{0.25(7)} & \num{8.3(1)} & \num{0.1(1)} & \num{9.00(5)} & \num{0.20(7)} & \num{5.2(1)} & \num{1.5(1)}\\
		\num{4.00(5)} & \num{0.25(7)} & \num{8.2(1)} & \num{0.1(1)} & \num{9.20(5)} & \num{0.20(7)} & \num{3.7(1)} & \num{2.3(1)}\\
		\num{4.25(5)} & \num{0.25(7)} & \num{8.2(1)} & \num{0.0(1)} & \num{9.40(5)} & \num{0.20(7)} & \num{1.4(1)} & \num{1.1(1)}\\
		\num{4.50(5)} & \num{0.25(7)} & \num{8.2(1)} & \num{0.0(1)} & \num{9.60(5)} & \num{0.20(7)} & \num{0.3(1)} & \num{0.2(1)}\\
		\num{4.75(5)} & \num{0.25(7)} & \num{8.2(1)} & \num{0.1(1)} & \num{9.80(5)} & \num{0.20(7)} & \num{0.1(1)} & \num{0.1(1)}\\
		\num{5.00(5)} & \num{0.25(7)} & \num{8.1(1)} & \num{0.0(1)} & \num{10.00(5)} & - & \num{0.0(1)} & -\\
		\num{5.25(5)} & \num{0.25(7)} & \num{8.1(1)} & \num{0.0(1)} & - & - & - & -\\
		\hline
	\end{tabular}
	\end{adjustbox}
	\caption{Messwerte zur Bestimmung der differntiellen Energieverteilung bei Raumtemperatur \label{tab:Auswertung_Diff_Energie_Verteilung_20C}}
\end{table}

	
	\includeFigure[scale=0.7]{Grafiken/Diff_EVerteilung_20.pdf}{Kurve der differentiellen Energieverteilung bei Raumtemperatur}{\label{fig:Auswertung_Diff_Energie_Verteilung_20C}} 
	
	Aus der Lage des Maximums der differentiellen Energieverteilung $U_{A,max} = \SI{9.20(5)}{\volt}$ 
	ergibt sich mit dem Wert der eingestellten Beschleunigungsspannung $U_{B} = \SI{11(1)}{\volt}$ das
	Kontaktpotential der Elektroden zu 
	\begin{empheq}{equation}
			K = U_{B} - U_{A,max} = \SI{2(1)}{\volt}.  
		    \label{val:Auswertung_K_1}
	\end{empheq}
	
			
\subsection{Messung der integralen Energieverteilung bei 150°C}
	
	Die für die Temperatur $T =\SI{150.0(1)}{\degreeCelsius} = \SI{423.2(1)}{\kelvin}$
	aufgenommene integrale Energieverteilung ist in \cref{fig:Auswertung_Int_Energie_Verteilung_150C}
	zu finden. Bei diesen Temperaturen sind die Werte für Dampfdruck und mittlere freie Weglänge
	$p_{sat} = \SI{4.82(2)}{\milli\bar}$ und $\overline{w} = \SI{0.000601(2)}{\cm}$. 
	
	\includeFigure[scale=0.6]{Grafiken/Grad100}{Kurve der Integralen Energieverteilung bei \SI{150}{\degreeCelsius}}{\label{fig:Auswertung_Int_Energie_Verteilung_150C}} 
	
	Aus dieser Kurve wurden zur grafischen Differentiation Werte entnommen die in \cref{tab:Auswertung_Diff_Energie_Verteilung_150C} gelistet sind. Die grafische
	Darstellung selbst befindet sich in \cref{fig:Auswertung_Diff_Energie_Verteilung_150C}.
	
	\begin{table}[!h]
	\centering
	\begin{adjustbox}{width=1\textwidth, center}
	\begin{tabular}{|c|c|c|c|c|c|c|c|}
		\hline
		Bremsspannung & $\Delta$ Bremsspannung & Auffängerstrom & $\Delta$ Auffängerstrom & Bremsspannung & $\Delta$ Bremsspannung & Auffängerstrom & $\Delta$ Auffängerstrom\\
		$U_{A}$ [\si{\volt}] & $\Delta U_{A}$ [\si{\volt}] & $\propto I_{A}(U_{A})$ & $\propto \Delta I_{A}(U_{A})$ & $U_{A}$ [\si{\volt}] & $\Delta U_{A}$ [\si{\volt}] & $\propto I_{A}(U_{A})$ & $\propto \Delta I_{A}(U_{A})$\\
\hline\hline
		\num{0.00(5)} & \num{0.20(7)} & \num{14.4(1)} & \num{0.3(1)} & \num{3.60(5)} & \num{0.20(7)} & \num{4.5(1)} & \num{0.5(1)}\\
		\num{0.20(5)} & \num{0.20(7)} & \num{14.1(1)} & \num{0.4(1)} & \num{3.80(5)} & \num{0.20(7)} & \num{4.0(1)} & \num{0.0(1)}\\
		\num{0.40(5)} & \num{0.20(7)} & \num{13.7(1)} & \num{0.6(1)} & \num{4.00(5)} & \num{0.20(7)} & \num{4.0(1)} & \num{0.4(1)}\\
		\num{0.60(5)} & \num{0.20(7)} & \num{13.1(1)} & \num{0.6(1)} & \num{4.20(5)} & \num{0.20(7)} & \num{3.6(1)} & \num{0.3(1)}\\
		\num{0.80(5)} & \num{0.20(7)} & \num{12.5(1)} & \num{0.7(1)} & \num{4.40(5)} & \num{0.20(7)} & \num{3.3(1)} & \num{0.0(1)}\\
		\num{1.00(5)} & \num{0.20(7)} & \num{11.8(1)} & \num{0.5(1)} & \num{4.60(5)} & \num{0.20(7)} & \num{3.2(1)} & \num{0.0(1)}\\
		\num{1.20(5)} & \num{0.20(7)} & \num{11.3(1)} & \num{0.6(1)} & \num{4.80(5)} & \num{0.20(7)} & \num{3.2(1)} & \num{0.0(1)}\\
		\num{1.40(5)} & \num{0.20(7)} & \num{10.7(1)} & \num{0.6(1)} & \num{5.00(5)} & \num{0.20(7)} & \num{3.2(1)} & \num{0.0(1)}\\
		\num{1.60(5)} & \num{0.20(7)} & \num{10.1(1)} & \num{0.5(1)} & \num{5.20(5)} & \num{0.20(7)} & \num{3.2(1)} & \num{0.0(1)}\\
		\num{1.80(5)} & \num{0.20(7)} & \num{9.6(1)} & \num{0.6(1)} & \num{5.40(5)} & \num{0.20(7)} & \num{3.2(1)} & \num{0.0(1)}\\
		\num{2.00(5)} & \num{0.20(7)} & \num{9.0(1)} & \num{0.7(1)} & \num{5.60(5)} & \num{0.20(7)} & \num{3.2(1)} & \num{0.0(1)}\\
		\num{2.20(5)} & \num{0.20(7)} & \num{8.3(1)} & \num{0.6(1)} & \num{5.80(5)} & \num{0.20(7)} & \num{3.2(1)} & \num{0.0(1)}\\
		\num{2.40(5)} & \num{0.20(7)} & \num{7.8(1)} & \num{0.6(1)} & \num{6.00(5)} & \num{0.20(7)} & \num{3.2(1)} & \num{0.0(1)}\\
		\num{2.60(5)} & \num{0.20(7)} & \num{7.2(1)} & \num{0.6(1)} & \num{6.20(5)} & \num{0.20(7)} & \num{3.2(1)} & \num{0.0(1)}\\
		\num{2.80(5)} & \num{0.20(7)} & \num{6.6(1)} & \num{0.4(1)} & \num{6.40(5)} & \num{0.20(7)} & \num{3.2(1)} & \num{0.0(1)}\\
		\num{3.00(5)} & \num{0.20(7)} & \num{6.2(1)} & \num{0.6(1)} & \num{6.60(5)} & \num{0.20(7)} & \num{3.2(1)} & \num{0.0(1)}\\
		\num{3.20(5)} & \num{0.20(7)} & \num{5.6(1)} & \num{0.5(1)} & \num{6.80(5)} & \num{0.20(7)} & \num{3.2(1)} & \num{0.0(1)}\\
		\num{3.40(5)} & \num{0.20(7)} & \num{5.1(1)} & \num{0.5(1)} & \num{7.00(5)} & - & \num{3.2(1)} & -\\
		\hline
	\end{tabular}
	\end{adjustbox}
	\caption{Messwerte zur Bestimmung der differntiellen Energieverteilung bei  \SI{150}{\degreeCelsius} \label{tab:Auswertung_Diff_Energie_Verteilung_150C}}
\end{table}

	
	\includeFigure[scale=0.7]{Grafiken/Diff_EVerteilung_150.pdf}{Kurve der differentiellen Energieverteilung bei Raumtemperatur}{\label{fig:Auswertung_Diff_Energie_Verteilung_150C}} 
	
\subsection{Bestimmung der Anregungsenergie des Quecksilbers}
	
	Zur Bestimmung der Anregungsenergie des Quecksilbers wurde die in \cref{fig:Auswertung_FranckHertz_Kurve}
	dargestellt Franck-Hertz-Kurve aufgenommen. Aus den Abständen der Maxima lässt sich die zur
	Anregung nötige Beschleunigungsspannung $U_{B}$ und daraus die Anregungsenergie bestimmen, diese sind in 
	\cref{tab:Auswertung_FranckHertz_Maxima} eingetragen. Diese Messung wurde bei einer Temperatur
	$T =\SI{165.0(1)}{\degreeCelsius} = \SI{438.2(1)}{\kelvin}$ und daraus folgend unter einem Dampfdruck
	von $p_{sat} = \SI{8.41(3)}{\milli\bar}$ und einer mittleren freien Weglänge der Elektronen 
	von $\overline{w} = \SI{0.000344(1)}{\cm}$ durchgeführt.
	
	\includeFigure[scale=0.5]{Grafiken/Franck_Hertz_Kurve.png}{Franck-Hertz-Kurve zur Bestimmung der Anregungsenergie von Quecksilber}{\label{fig:Auswertung_FranckHertz_Kurve}} 
	
	
	\begin{table}[!h]
	\centering
	\begin{tabular}{|c|c|c|}
		\hline
		maximal Stellen & $\Delta$ maximal Stellen & Anregungsenergie\\
		$U_{A,max}^{(i)}$ [\si{\volt}] & $U_{A,max}^{(i+1)} - U_{A,max}^{(i)}$ [\si{\volt}] & $E$ [\si{\eV}]\\
\hline\hline
%		\num{0.0(4)} & \num{7.2(6)} & \num{7.2(6)}\\
		\num{7.2(4)} & \num{4.6(6)} & \num{4.6(6)}\\
		\num{11.8(4)} & \num{4.9(6)} & \num{4.9(6)}\\
		\num{16.7(4)} & \num{4.9(6)} & \num{4.9(6)}\\
		\num{21.7(4)} & \num{5.2(6)} & \num{5.2(6)}\\
		\num{26.9(4)} & - & -\\
		\hline
	\end{tabular}
	\caption{Messwerte zur Bestimmung der differntiellen Energieverteilung bei \SI{150}{\degreeCelsius} \label{tab:Auswertung_Diff_Energie_Verteilung_150C}}
\end{table}

	
	Der Mittelwert Anregungsenergien hat damit den Wert
	
	\begin{empheq}{equation}
		\label{val:Auswertung_Anregungsenergie}
		\mean{E} = \SI{4.9(1)}{\eV}.
	\end{empheq}
	
	Über die Energiegleichung des Photons $E = hc\lambda^{-1}$ lässt sich aus der
	erhaltenen, gemittelten Energie, die Wellenlänge $\lambda$ des nach der Anregung emittierten Photons
	zu
	\begin{empheq}{equation}
		\lambda = \SI{252(6)}{\nm}
	\end{empheq}	 	
	bestimmen.
		 	
	Durch Subtraktion des Mittelwertes der Anregungsenergie, von der Energie des ersten Maximums
	ergibt sich für das Kontaktpotential der Elektroden der Wert
	\begin{empheq}{equation}
			K = \SI{2.3(6)}{\volt}.  
			\label{val:Auswertung_K_2}
	\end{empheq}
	  
\subsection{Bestimmung der Ionisierungsspannung des Quecksilbers}
	
	Die Messung zur Bestimmung der Ionisierungsspannung $U_{ion}$ des Quecksilbers wurde mit einer Temperatur 
	von $T =\SI{100.0(1)}{\degreeCelsius} = \SI{373.2(1)}{\kelvin}$ bei dem Dampfdruck
    $p_{sat} = \SI{0.547(3)}{\milli\bar}$ durchgeführt. Unter diesen Bedingungen beträgt die 
    mittlere freie Weglänge der Elektronen $\overline{w} = \SI{0.00531(3)}{\cm}$.
    Zur Bestimmung der Ionisierungsspannung $U_{ion}$ wird eine tangentiale Gerade an die aufgenommenen
    Kurve, in \cref{fig:Auswertung_Ionisierungsspannung} gelegt.
    
   
    
    \includeFigure[scale=0.5]{Grafiken/Ionisierungsspannung}{Grafische Bestimmung der Ionisierungsspannung }{\label{fig:Auswertung_Ionisierungsspannung}} 
    
    
    
    
    Die Nullstelle dieser Geraden hat den Wert $U_{0} = U_{ion} + K$, wobei $K$ das zuvor berechnete Kontaktpotential
    der beiden Elektroden ist. Mit dem aus \cref{fig:Auswertung_Ionisierungsspannung} bestimmten 
    Wert  $U_{0} = \SI{12.5(2)}{\volt}$ errechnet sich, unter Verwendung des Mittelwerts 
    \begin{empheq}{equation}
    	\mean{K} = \SI{2.0(6)}{\volt}
    	\label{val:Auswertung_K_Mittel}
    \end{empheq}
     aus 
    \cref{val:Auswertung_K_1} und \cref{val:Auswertung_K_2}, die Ionisierungsspannung zu
   	\begin{empheq}{equation}
   			\label{val:Asuwertung_U_Ion}
   			U_{ion} = \SI{10.5(6)}{\volt}.  
   	\end{empheq}