In folgendem Abschnitt werden die in \cref{sec:Auswertung} erhaltenen 
Ergebnisse noch einmal abschließend diskutiert und dabei auf ihre Plausibilität
hin überprüft. \\

Die Verläufe der bei $T_{1} = \SI{25}{\celsius}$ und $T_{2} = \SI{150}{\celsius}$ aufgenommenen 
$(U_{A}|I_{A})$ Kurven lassen sich qualitative durch den, in der in der 
Franck-Hertz-Röhre herrschenden Dampfdruck und die daraus resultierende mittlere freie 
Weglänge begründen.

Für $T_{1}$ ist der Quotient aus mittlerer freier Weglänge $\bar{w}$ und der Beschleunigungsstrecke
$s = \SI{1}{\cm}$, mit $\sfrac{\bar{w_{1}}}{s} = \num{0.547(4)}$ um 3 Größenordnungen größer
als der selbe Quotient bei $T_{2}$, $\sfrac{\bar{w_{1}}}{s} =  \num{0.000601(2)}$.
Durch die wesentlich kleinere Stoßwahrscheinlichkeit der Elektronen für den kälteren Fall,
ist die für diese Temperatur aufgenommene Kurve über einen großen Bremsspannungsbereich 
nahezu konstant und fällt erst ab, wenn der Wert der Bremsspannung den Wert der effektiven
Beschleunigungsspannung $U_{B,eff}$ überschreitet. Das Einbrechen des Stromes ist
in diesem Fall auf die hohe Bremsspannung und nicht auf Stöße mit den Quecksilberatomen 
zurückzuführen.

Ein andres Verhalten ist bei der Temperatur $T_{2}$ zu beobachten, hier ist die Stoßwahrscheinlichkeit
wesentlich höher, wodurch der schon bei geringen Bremsspannungen abfallende Auffängerstrom zu begründen ist.
Der nach dem Abfall anschließende nahezu konstante Verlauf ist mit dem Verlauf der Kurve bei $T_{1}$ 
zu vergleichen, somit wird dieser Strom durch Elektronen hervorgerufen, die ohne Stöße an der Auffängeranode
ankommen.\\

Die aus der Lage der Maxima der Franck-Hertz-Kurve bestimmte Mittelwert der Anregungsenergie 
\cref{val:Auswertung_Anregungsenergie}, 
 stimmt mit dem Literaturwert $E_{lit} = \SI{4.9}{\eV}$ der Herren Franck und Hertz 
 überein \cite{FranckHertz14}, wodurch dieser
als Plausibel angesehen werden kann. Eine Berücksichtigung der Energieverluste durch elastische Stöße
der Elektronen mit dem Quecksilberatomen ist nicht nötig, da diese Energiedifferenzen noch unter dem 
Fehler der durchgeführten Messungen liegen.   

Das aus der Franck-Hertz-Kurve bestimmte Kontaktpotential $K$ hat einen im Rahmen der Messgenauigkeit 
mit dem, aus der Messung bei $T_{1}$ zu vergleichenden Wert und auch der Mittelwert \cref{val:Auswertung_K_Mittel} aus 
beiden ist in seiner Höhe plausibel.\\
 
