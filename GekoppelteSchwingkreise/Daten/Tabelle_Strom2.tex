\begin{table}[!h]
	\centering
	\begin{tabular}{|c|c|c|c|}
		\hline
		\multicolumn{2}{|c|}{Berechnete}& \multicolumn{2}{c|}{Gemessene}\\\hline
		Stromstärke & Stromstärke & Stromstärke & Stromstärke\\
		$I^{+}\,[\si{\ampere}]$ & $I^{-}\,[\si{\ampere}]$ & $I^{+}\,[\si{\ampere}]$ & $I^{-}\,[\si{\ampere}]$\\\hline\hline
		\num{0.015(6)}  & \num{0.03(4)}  & \num{0.0256(6)}  & \num{0.0141(6)} \\
		\num{0.03(2)}  & \num{0.02(3)}  & \num{0.0256(6)}  & \num{0.0141(6)} \\
		\num{0.018(6)}  & \num{0.02(1)}  & \num{0.0263(6)}  & \num{0.0147(6)} \\
		\num{0.021(4)}  & \num{0.03(4)}  & \num{0.0263(6)}  & \num{0.0147(6)} \\
		\num{0.023(2)}  & \num{0.04(7)}  & \num{0.0269(6)}  & \num{0.0147(6)} \\
		\num{0.028(2)}  & \num{0.06(9)}  & \num{0.0269(6)}  & \num{0.0154(6)} \\
		\num{0.04(2)}  & \num{0.07(5)}  & \num{0.0269(6)}  & \num{0.0160(6)} \\
		\hline
	\end{tabular}
	\caption{Theoretisch bestimmte und gemessene Stromstärken \label{tab:I2}}
\end{table}