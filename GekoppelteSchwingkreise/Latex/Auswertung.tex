Im Folgenden sind die während des Versuchs aufgenommenen Messwert 
und die aus diesen berechneten Größen tabellarisch aufgeführt.
An entsprechender Stelle sind Erklärungen zu den Rechnungen und
Messwerten gegeben.     

\subsection{Messung des Frequenzverhältnisses}\label{sec:Auswertung_FrequenzVerhältnis}

	Die durch Abzählen bestimmte Anzahl der Schwingungamplituden $A_{1}$ pro 
	Schwebungsamplitude, das entsprechende Verhältnis sind in 
	\cref{tab:Schwebung} mit der jeweiligen Kapazität eingetragen dargestellt.  
	
	\begin{table}[!h]
	\centering
	\begin{tabular}{|c|c|c|c|c|c|c|}
		\hline
 		Gang & \multicolumn{3}{c|}{Frequenzdifferenz hin} & \multicolumn{3}{c|}{Frequenzdifferenz zurück} \\
		$g$ & $\Delta\nu_{h1}\,[\si{\hertz}]$ & $\Delta\nu_{h2}\,[\si{\hertz}]$ & $\left<\Delta\nu_{h}\right>\,[\si{\hertz}]$ & $\Delta\nu_{r1}\,[\si{\hertz}]$ & $\Delta\nu_{r2}\,[\si{\hertz}]$ & $\left<\Delta\nu_{r}\right>\,[\si{\hertz}]$\\\hline\hline
		\num{6}  & -  & -  & -  & -  & -  & - \\
		\num{12}  & \num{4(1)}  & \num{4(1)}  & \num{4.0(7)}  & \num{-11(1)}  & \num{-12(1)}  & \num{-11.5(7)} \\
		\num{18}  & \num{14(1)}  & \num{14(1)}  & \num{14.0(7)}  & \num{-18(1)}  & \num{-18(1)}  & \num{-18.0(7)} \\
		\num{24}  & \num{17(1)}  & \num{18(1)}  & \num{17.5(7)}  & \num{-24(1)}  & \num{-23(1)}  & \num{-23.5(7)} \\
		\num{30}  & \num{16(1)}  & \num{14(1)}  & \num{15.0(7)}  & \num{-28(1)}  & \num{-29(1)}  & \num{-28.5(7)} \\
		\num{36}  & \num{20(1)}  & \num{19(1)}  & \num{19.5(7)}  & \num{-34(1)}  & \num{-34(1)}  & \num{-34.0(7)} \\
		\num{42}  & \num{28(1)}  & \num{24(1)}  & \num{26.0(7)}  & \num{-34(1)}  & \num{-35(1)}  & \num{-34.5(7)} \\
		\num{48}  & \num{33(1)}  & \num{32(1)}  & \num{32.5(7)}  & \num{-34(1)}  & \num{-33(1)}  & \num{-33.5(7)} \\
		\num{54}  & \num{35(1)}  & \num{36(1)}  & \num{35.5(7)}  & \num{-30(1)}  & \num{-28(1)}  & \num{-29.0(7)} \\
		\num{60}  & \num{39(1)}  & \num{56(1)}  & \num{47.5(7)}  & \num{-27(1)}  & \num{-23(1)}  & \num{-25.0(7)} \\
		\hline
	\end{tabular}
	\caption{Direkt gemessene Frequenzen des Wagens in den verschiedenen Gängen \label{tab:Auswertung_Frequenz_Schwebung}}
\end{table}
	
	Mit Hilfe von \cref{eq:f+} und \cref{eq:f-} können die Fundamentalfrequenzen
	$ \nu^{+} \text{und}\, \nu^{-}$ aus den gegebenen Größen
	\begin{subequations}
		\begin{empheq}{align}
			C &= \SI{0.79320(5)}{\nano\farad} \\
			C_{sp} &= \SI{0.028}{\nano\farad} \\
			L &= \SI{23.95(5)}{\milli\henry}
		\end{empheq}
	\end{subequations}
	bestimmt werden. Diese Frequenzen sind in \cref{tab:Fundamental_Freqs} zu finden,
	in der auch das unter Verwendung von \cref{eq:SchwebungsFreq} und \cref{eq:SchwingungsFreq}
	jeweils bestimmte Verhältnis $\tfrac{2(\nu^{-} - \nu^{+})}{\nu^{-} + \nu^{+}}$ eingetragen 
	ist.
	
	\begin{table}[!h]
	\centering
	\begin{tabular}{|c|c|c|c|}
		\hline
		Kapazitäten & Fundamentalfrequenz & Fundamentalfrequenz & Frequenzverhältnis\\
		$C_{K}\,[\si{\nano\farad}]$ & $\nu^{+}\,[\si{\kilo\hertz}]$ & $\nu^{-}\,[\si{\kilo\hertz}]$ & $\tfrac{2(\nu^{-} - \nu^{+})}{\nu^{-} + \nu^{+}}$\\\hline\hline
		\num{1.00(3)}  & \num{35.88(4)}  & \num{56.3(5)}  & \num{0.442(8)} \\
		\num{2.19(7)}  & \num{35.88(4)}  & \num{46.6(3)}  & \num{0.259(6)} \\
		\num{2.86(9)}  & \num{35.88(4)}  & \num{44.3(2)}  & \num{0.210(5)} \\
		\num{4.7(1)}  & \num{35.88(4)}  & \num{41.2(2)}  & \num{0.138(4)} \\
		\num{6.9(2)}  & \num{35.88(4)}  & \num{39.7(1)}  & \num{0.100(3)} \\
		\num{8.2(2)}  & \num{35.88(4)}  & \num{39.1(1)}  & \num{0.085(2)} \\
		\num{10.0(3)}  & \num{35.88(4)}  & \num{38.52(9)}  & \num{0.071(2)} \\
		\num{12.0(4)}  & \num{35.88(4)}  & \num{38.10(8)}  & \num{0.060(2)} \\
		\hline
	\end{tabular}
	\caption{Berechnete Fundamentalfrequenzen und das Frequenzverhältnis der Schwebung \label{tab:Fundamental_Freqs}}
\end{table}
	
	Die relativen Abweichungen der gemessenen von den berechneten Werten des Frequenzverhältnisses
	$ N_{t} := \tfrac{2(\nu^{-} - \nu^{+})}{\nu^{-} + \nu^{+}}$ und $N_{m} := \tfrac{1}{A_{1}}$ sind in \cref{tab:FrequenzVerhältnis} enthalten.
	
	\input{Daten/Tabelle_FrequenzVerhältnis.tex}
	
\subsection{Messung der Fundamentalfrequenzen}\label{sec:Auswertung_FundamentalFrequenz}
	
	Die gemessenen Fundamentalfrequenzen der gekoppelten Schwingkreise
	sind in \cref{tab:Fundamental_Messung} zusammen mit den berechneten
	Fundamentalfrequenzen aus \cref{tab:Fundamental_Freqs}  zu finden.
	Neben diesen sind dort auch die Verhältnisse von gemessener zu 
	berechneter Frequenz angegeben. 
	
	\begin{table}[!h]
	\centering
	\begin{tabular}{|c|c|c|c|c|c|}
		\hline
%		Fundamentalfrequenz & Fundamentalfrequenz & Fundamentalfrequenz & Fundamentalfrequenz & Frequenzverhältnis & Frequenzverhältnis\\
		\multicolumn{2}{|c|}{Berechnete} & \multicolumn{2}{c|}{Gemessene} & \multicolumn{2}{c|}{ } \\
		\multicolumn{2}{|c|}{Fundamentalfrequenzen} & \multicolumn{2}{c|}{Fundamentalfrequenzen} &
		\multicolumn{2}{c|}{Frequenzverhältnis} \\
		$\nu^{+}_{theo}\,[\si{\kilo\hertz}]$ & $\nu^{-}_{theo}\,[\si{\kilo\hertz}]$ & $\nu^{+}\,[\si{\kilo\hertz}]$ & $\nu^{-}\,[\si{\kilo\hertz}]$ & $\sfrac{\nu^{+}}{\nu^{+}_{theo}}$ & $\sfrac{\nu^{-}}{\nu^{-}_{theo}}$\\\hline\hline
		\num{35.88(4)}  & \num{56.3(5)}  & \num{35.65(1)}  & \num{40.68(1)}  & \num{0.993(1)}  & \num{0.723(6)} \\
		\num{35.88(4)}  & \num{46.6(3)}  & \num{35.65(1)}  & \num{39.61(1)}  & \num{0.993(1)}  & \num{0.851(5)} \\
		\num{35.88(4)}  & \num{44.3(2)}  & \num{35.65(1)}  & \num{39.06(1)}  & \num{0.993(1)}  & \num{0.881(5)} \\
		\num{35.88(4)}  & \num{41.2(2)}  & \num{35.65(1)}  & \num{39.25(1)}  & \num{0.993(1)}  & \num{0.952(4)} \\
		\num{35.88(4)}  & \num{39.7(1)}  & \num{35.63(1)}  & \num{37.83(1)}  & \num{0.993(1)}  & \num{0.954(3)} \\
		\num{35.88(4)}  & \num{39.1(1)}  & \num{35.62(1)}  & \num{37.22(1)}  & \num{0.993(1)}  & \num{0.952(2)} \\
		\num{35.88(4)}  & \num{38.52(9)}  & \num{35.62(1)}  & \num{37.15(1)}  & \num{0.993(1)}  & \num{0.964(2)} \\
		\num{35.88(4)}  & \num{38.10(8)}  & \num{35.60(1)}  & \num{37.09(1)}  & \num{0.992(1)}  & \num{0.974(2)} \\
		\hline
	\end{tabular}
	\caption{Berechnete und gemessene Fundamentalfrequenzen mit jeweiligem Verhältnis \label{tab:Fundamental_Messung}}
\end{table}