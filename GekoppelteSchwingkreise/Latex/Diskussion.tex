Im folgenden Abschnitt werden die in \cref{sec:Auswertung} erarbeiteten Ergebnisse,
durch Vergleich mit den theoretisch berechneten Werten auf Richtigkeit beziehungsweise
Plausibilität hin untersucht. Dabei wird auch auf den verwendeten Versuchsaufbau und die 
Versuchsdurchführung Bezug genommen.\\

Die in \cref{sec:Auswertung_FrequenzVerhältnis} erhaltenen Amplitudenverhältnisse in \cref{tab:Schwebung} 
stimmen mit den, aus den berechneten Werten der Fundamentalschwingung bestimmten, Frequenzverhältnissen
in \cref{tab:Fundamental_Messung} nicht exakt, jedoch in plausibler Weise überein. Wie aus den Werten in 
\cref{tab:FrequenzVerhältnis} zu berechnen ist weichen die gemessenen Werte im Mittel um weniger als
$\SI{13}{\percent}$ ab. Dabei sind die erhalten Messwerte jeweils größer als der theoretisch bestimmte 
Wert, es handelt sich demnach um eine allgemeine Abweichung nach oben.\\
Eine Erklärung für dies Abweichung ist die verwendete Messmethode, da sich durch Abzählen der Schwingungsamplituden
pro Schwebungsamplitude nur Frequenzverhältnisse bestimmen lassen die als Bruch mit ganzzahligem Nenner beschreibbar
sind, wodurch nicht ganzzahlige Verhältnisse nicht bestimmbar sind. Eine Erhöhung der Genauigkeit erhielte man durch 
mitteln der Anzahl an Schwingungsamplituden über mehrere Schwebungsamplituden.   
     
Die in \cref{sec:Auswertung_FundamentalFrequenz} gemessenen Fundamentalfrequenzen stimmen vom Verlauf mit den 
theoretischen Fundamentalfrequenzen der gekoppelten Schwingkreise überein. So ist die gemessene Fundamentalfrequenz
$\nu^{+}$  aus \cref{tab:Fundamental_Freqs} genau wie die berechnete Frequenz bis auf geringe Änderungen 
konstant und mit einer Abweichung von unter $\SI{1}{\percent}$ vom Theoriewert sehr genau. 
Die gemessenen Fundamentalfrequenzen $\nu^{-}$ zeigen für die ersten drei Messungen mi durchschnittlich $\SI{18}{\percent}$
größere Abweichungen als die der folgenden Messungen bei denen die Abweichung bei durchschnittlich $\SI{4}{\percent}$
liegt. Auch bei dieser Messung haben alle Abweichungen das selbe (negative) Vorzeichen, wodurch auch hier wider
von einem systematischen Fehler auszugehen ist.\\
Die in \cref{sec:Auswertung_Wobbel} theoretisch berechneten Stromstärken zeigen deutliche Abweichungen von den
gemessen. So haben einige der berechneten Stromstärken $I^{+}$ eine geringe Abweichung zu den gemessenen,
wohingegen die Werte für $I^{-}$ deutlich von den gemessen abweichen. Eine Begründung für diese mitunter
starken Abweichungen könnte das Einbrechen der Generatorspannung sein die beim erreichen der Resonanzfrequenz
Eintritt, wodurch diese folglich nicht mehr konstant wäre. Eine weiterführende Begründung ist die Tatsache, dass
die aus den Abbildungen wie \ref{fig:WobbelVerlauf} abgelesene Resonanzfrequenzen nicht exakt der 
tatsächlichen Entspricht, wodurch diese Abweichung ebenfalls in die Berechnung der Ströme einfließt.

              