Der Lock-In-Verstärker kann stark verrauschte Signale messen, dazu wird das verrauschte Signal in mehreren Schritten bearbeitet. Als erstes wird das verrauschte Signal im Preamplifier verstärkt. Anschließend wird das Signal von Frequenzen, die stark von einer beim Bandpassfilter einstellbaren Frequenz abweichen, befreit.
Im nächsten Schritt wird das Signal in einem Mischer mit einem
Referenzsignal, welches eine einstellbare Phase besitzt, multipliziert. Die Frequenz des Referenzsignals wird dabei der des zu messenden Signals angeglichen. Das Referenzsignal ist hier eine Sinusspannung. 
Damit das Referenzsignal hat die Form:

\begin{equation}
  \label{eq:ref}
  U_{ref} = \frac{4}{\pi} \left( \sin(\omega t) + \frac{1}{3} \sin(\omega t) + \frac{1}{5} \sin(\omega t) + ... \right)
\end{equation}

Die Formel für das zu messende Signal ist gegeben durch:

\begin{equation}
  \label{eq:sig}
  U_{sig} = U_{0} \sin(\omega t)
\end{equation}

Die Multiplikation der beiden Signale $U_{sig}$ und $U_{ref}$ ergibt:

\begin{equation*}
  U_{sig} \times U_{ref} = \frac{2}{\pi} U_{0} \left(1 - \frac{2}{3} \cos(2\omega t) + \frac{2}{15} \cos(4\omega t) + \frac{2}{35} \cos(6\omega t) + ... \right)
\end{equation*}

Da der Tiefpass nur niedrige Frequenzen durch lässt, ergibt sich die Form von $U_{out}$, nach Mischer und Tiefpass.

Zusätzlich muss noch der Phasenunterschied $\phi$ beachtet werden, sollte einer vorliegen, muss dieser mit betrachtet werden.

\begin{equation}
  \label{eq:out}
   U_{out} = \frac{2}{\pi} U_{0} \cos(\phi)
\end{equation}

Das endgültige Signal ist damit eine Gleichspannung, die proportional zum
Produkt der Amplituden der Referenzspannung und des Signals ist.
