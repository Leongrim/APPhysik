Im folgenden Abschnitt sind die während des Versuchs aufgenommenen Messwerte,
sowie die daraus berechneten Ergebnisse tabellarisch und graphisch dargestellt.
Die erhaltenen Fehler der Ergebnisse wurden mit Hilfe der in \cref{sec:Fehlerrechnung}
aufgestellten Fehlergleichungen berechnet.

\subsection{Bestimmung der Halbwertszeiten der zwei möglichen Zerfälle von Rhodium}
	
	Die bei der Messung des Zerfalls von Rhodium aufgenommenen Messwerte
	für die Zeit $t$ und die Anzahl der gemessenen Zerfälle $N$ in \cref{tab:Auswertung_Messwerte_Rhodium}
	eingetragen. Auch die um den, vor dem Versuch bestimmte Nulleffekt 
	\begin{empheq}{align}
		N_{0} &= \dfrac{306}{900}\ \si{\per\second} \cdot \Delta t\\    
		&= \SI{0.34}{\per\second}  \cdot \Delta t 
	\end{empheq} 
    verringerte Anzahl an Zerfällen ist zusammen mit dem natürlichen Logarithmus
    aus diesen Werten in \cref{tab:Auswertung_Messwerte_Rhodium} zu finden.
    
   	\begin{table}[!h]
	\centering
	\begin{adjustbox}{width=1\textwidth, center}
	\begin{tabular}{|r|c|c|c|r|c|c|c|}
		\hline
		Zeit & Zerfälle & Zerfälle & ln der Zerfälle & Zeit & Zerfälle & Zerfälle & ln der Zerfälle\\
		$t$ [\si{\second}] & $N$ & $N - N_{0}$ & $\ln(N - N_{0})$ & $t$ [\si{\second}] & $N$ & $N - N_{0}$ & $\ln(N - N_{0})$\\
\hline\hline
		\num{20} & \num{1.9(1)e+02} & \num{1.9(1)e+02} & \num{5.24(7)} & \num{380} & \num{36(6)} & \num{29(5)} & \num{3.4(2)}\\
		\num{40} & \num{1.6(1)e+02} & \num{1.5(1)e+02} & \num{5.03(8)} & \num{400} & \num{48(7)} & \num{41(6)} & \num{3.7(2)}\\
		\num{60} & \num{1.5(1)e+02} & \num{1.4(1)e+02} & \num{4.97(8)} & \num{420} & \num{36(6)} & \num{29(5)} & \num{3.4(2)}\\
		\num{80} & \num{9(1)e+01} & \num{86(9)} & \num{4.5(1)} & \num{440} & \num{43(7)} & \num{36(6)} & \num{3.6(2)}\\
		\num{100} & \num{1.1(1)e+02} & \num{1.1(1)e+02} & \num{4.7(1)} & \num{460} & \num{30(5)} & \num{23(5)} & \num{3.1(2)}\\
		\num{120} & \num{84(9)} & \num{77(9)} & \num{4.3(1)} & \num{480} & \num{38(6)} & \num{31(6)} & \num{3.4(2)}\\
		\num{140} & \num{1.0(1)e+02} & \num{89(9)} & \num{4.5(1)} & \num{500} & \num{29(5)} & \num{22(5)} & \num{3.1(2)}\\
		\num{160} & \num{75(9)} & \num{68(8)} & \num{4.2(1)} & \num{520} & \num{27(5)} & \num{20(4)} & \num{3.0(2)}\\
		\num{180} & \num{61(8)} & \num{54(7)} & \num{4.0(1)} & \num{540} & \num{32(6)} & \num{25(5)} & \num{3.2(2)}\\
		\num{200} & \num{76(9)} & \num{69(8)} & \num{4.2(1)} & \num{560} & \num{22(5)} & \num{15(4)} & \num{2.7(3)}\\
		\num{220} & \num{48(7)} & \num{41(6)} & \num{3.7(2)} & \num{580} & \num{20(4)} & \num{13(4)} & \num{2.6(3)}\\
		\num{240} & \num{62(8)} & \num{55(7)} & \num{4.0(1)} & \num{600} & \num{35(6)} & \num{28(5)} & \num{3.3(2)}\\
		\num{260} & \num{46(7)} & \num{39(6)} & \num{3.7(2)} & \num{620} & \num{33(6)} & \num{26(5)} & \num{3.3(2)}\\
		\num{280} & \num{49(7)} & \num{42(6)} & \num{3.7(2)} & \num{640} & \num{21(5)} & \num{14(4)} & \num{2.7(3)}\\
		\num{300} & \num{52(7)} & \num{45(7)} & \num{3.8(1)} & \num{660} & \num{14(4)} & \num{7(3)} & \num{2.0(4)}\\
		\num{320} & \num{55(7)} & \num{48(7)} & \num{3.9(1)} & \num{680} & \num{19(4)} & \num{12(3)} & \num{2.5(3)}\\
		\num{340} & \num{51(7)} & \num{44(7)} & \num{3.8(2)} & \num{700} & \num{24(5)} & \num{17(4)} & \num{2.8(2)}\\
		\num{360} & \num{45(7)} & \num{38(6)} & \num{3.6(2)} & \num{720} & \num{22(5)} & \num{15(4)} & \num{2.7(3)}\\
		\hline
	\end{tabular}
    \end{adjustbox}
	\caption{Gemessene Anzahl der Zerfäll, Anzahl der Zerfälle nach Subtraktion des Nulleffekts und Werte des natürlichen Logarithmusses von diesen \label{tab:Auswertung_Messwerte_Rhodium}}
\end{table}
 
   	
   	In \cref{fig:} ist die logarithmierten Anzahl der Zerfälle $\Ln{N - N_{0}}$ aus \cref{tab:Auswertung_Messwerte_Rhodium} gegen die Zeit $t$ aufgetragen.  
   	
    \includeFigure[scale=0.7]{Grafiken/Messwerte_Rh.pdf}{Graphische Darstellung der logarithmierten Zerfälle ohne den Nulleffekt}{\label{fig:Auswertung_Messwerte_Rh}}
     
    Der Zeitpunkt ab dem nur noch der Zerfall mit der höheren Halbwertzeit messbar ist
    wurde für die folgenden Berechnungen $t^{*} = \SI{400}{\second}$ gewählt.
    Die Messwerte für $t > t*$ sind noch einmal in \cref{tab:Auswertung_Messwerte_Rhodium_lang} gelistet und 
    in  \cref{fig:Auswertung_Messwerte_Rh_lang} graphisch dargestellt. Diese Darstellung ist um die Regressiongerade
    dieser Messwerte ergänzt dir mittels \emph{SciPy} \cite{SciPy} berechnet wurde.
    Die lineare Regression für den Ansatz
    \begin{empheq}{equation}
    \Ln{N} = \lambda_{l} \cdot t + c_{l},
    \end{empheq}
    ergibt die Parameter
    \addtocounter{equation}{-1}
    \begin{subequations}
    	\begin{empheq}{align}
    		\lambda_{l} &= \SI{0.003(2)}{\per\second} \quad \text{und} \\
    		c_{l} &= \SI{4.4(9)}{}.
    	\end{empheq}
    \end{subequations}

	\begin{table}[!h]
	\centering
	\begin{tabular}{|r|c|c|}
		\hline
		Zeit & Zerfälle & ln der Zerfälle\\
		$t$ [\si{\second}] & $N_{l}$ & $\ln(N - N_{0})$\\
\hline\hline
		\num{420} & \num{29(5)} & \num{3.4(2)}\\
		\num{440} & \num{36(6)} & \num{3.6(2)}\\
		\num{460} & \num{23(5)} & \num{3.1(2)}\\
		\num{480} & \num{31(6)} & \num{3.4(2)}\\
		\num{500} & \num{22(5)} & \num{3.1(2)}\\
		\num{520} & \num{20(4)} & \num{3.0(2)}\\
		\num{540} & \num{25(5)} & \num{3.2(2)}\\
		\num{560} & \num{15(4)} & \num{2.7(3)}\\
		\num{580} & \num{13(4)} & \num{2.6(3)}\\
		\num{600} & \num{28(5)} & \num{3.3(2)}\\
		\num{620} & \num{26(5)} & \num{3.3(2)}\\
		\num{640} & \num{14(4)} & \num{2.7(3)}\\
		\num{660} & \num{7(3)} & \num{2.0(4)}\\
		\num{680} & \num{12(3)} & \num{2.5(3)}\\
		\num{700} & \num{17(4)} & \num{2.8(2)}\\
		\num{720} & \num{15(4)} & \num{2.7(3)}\\
		\hline
	\end{tabular}
	\caption{Messwerte zur Bestimmung der Halbwertszeit des langlebigen Zerfalls für t > t* \label{tab:Auswertung_Messwerte_Rhodium_lang}}
\end{table}


    \includeFigure[scale=0.7]{Grafiken/Messwerte_Rh_langlebig.pdf}{Graphische Darstellung der logarithmierten Zerfälle für $t > t^{*}$ }{\label{fig:Auswertung_Messwerte_Rh_lang}}
    
    Dabei gilt $c_{l} = \Ln{N_{a}(1 - \E{-\lambda_{l}\Delta t})}$ und somit erhält man hieraus
    die gesuchte Konstante
   	\begin{empheq}{equation}
   		\E{c_{l}} = N_{a}(1 - \E{-\lambda_{l}\Delta t}) = \SI{70(70)}{}.
   	\end{empheq}
   	
   	Aus der erhaltenen Steigung $\lambda_{l}$ der Regressionsgerade, welche der 
   	Zerfallskonstante des langlebigeren Zerfalls entspricht, lässt sich mit 
   	Hilfe von \cref{eq:Theorie_Halbwertzeit} dessen Halbwertzeit zu
  	\begin{empheq}{equation}
  		t_{\sfrac{1}{2},l} =  \SI{277(168)}{\second}
  	\end{empheq}
    bestimmen.\\
    
    Durch die zuvor bestimmten Parameter ist es nun möglich, das Zerfallsgesetz für die langlebigeren Kerne
    aufzustellen und somit die Zerfallskurve vor dem Zeitpunkt $t*$ zu bestimmen. Durch Subtraktion dieser
    Zerfälle von den Messwerten für $t << t*$ erhält man die Zerfälle der kurzlebigen Kerne. Die Ergebnisse
    dieses Vorgehens sind in \cref{tab:Auswertung_Messwerte_Rhodium_kurz} zu finden und in \cref{fig:Auswertung_Messwerte_Rh_kurz} graphisch dargestellt.
    
    \begin{table}[!h]
	\centering
	\begin{tabular}{|r|c|c|c|}
		\hline
		Zeit & Zerf�lle & Zerf�lle & ln der Zerf�lle\\
		$t$ [\si{\second}] & $N$ & $N - N_{l}$ & $\ln(N - N_{0})$\\
\hline\hline
		\num{20} & \num{1.9(1)e+02} & \num{1.1(2)e+02} & \num{4.7(1)}\\
		\num{40} & \num{1.5(1)e+02} & \num{8(1)e+01} & \num{4.4(2)}\\
		\num{60} & \num{1.4(1)e+02} & \num{8(1)e+01} & \num{4.3(2)}\\
		\num{80} & \num{86(9)} & \num{2(1)e+01} & \num{3.1(5)}\\
		\num{100} & \num{1.1(1)e+02} & \num{5(1)e+01} & \num{3.8(3)}\\
		\num{120} & \num{77(9)} & \num{2(1)e+01} & \num{3.0(6)}\\
		\num{140} & \num{89(9)} & \num{3(1)e+01} & \num{3.5(3)}\\
		\num{160} & \num{68(8)} & \num{2(1)e+01} & \num{2.8(7)}\\
		\num{180} & \num{54(7)} & \num{0(1)e+01} & \num{2(2)}\\
		\num{200} & \num{69(8)} & \num{2(1)e+01} & \num{3.1(5)}\\
		\hline
	\end{tabular}
	\caption{Messwerte zur Bestimmung der Halbwertszeit des langlebigen Zerfalls f�r t << t* \label{tab:Auswertung_Messwerte_Rhodium_kurz}}
\end{table}

    
    \includeFigure[scale=0.7]{Grafiken/Messwerte_Rh_kurzlebig.pdf}{Graphische Darstellung der logarithmierten Zerfälle% 
    für $t << t^{*}$ }{\label{fig:Auswertung_Messwerte_Rh_kurz}} 
     
    Aus der, mit \emph{SciPy} durchgeführten, linearen Regression mit dem Ansatz
    \begin{empheq}{equation}
    \Ln{N} = \lambda_{k} \cdot t + c_{k},
    \end{empheq}
    ergeben sich die Parameter zu
    \addtocounter{equation}{-1}
    \begin{subequations}
       	\begin{empheq}{align}
       		\lambda_{k} &= \SI{0.013(3)}{\per\second} \quad \text{und} \\
       		c_{k} &= \SI{4.9(4)}{}.
       	\end{empheq}
    \end{subequations}  
    Die daraus erhaltene Gerade ist ebenfalls in \cref{fig:Auswertung_Messwerte_Rh_kurz} eingezeichnet.
    
    Analog zu dem Zerfall der langlebigeren Kerne, erhält man aus den bestimmten Regressionsparametern
    die gesuchte Konstante $\E{c_{k}}$ und die Halbwertzeit $t_{\sfrac{1}{2},k}$ dieses Zerfalls zu
    \begin{empheq}{align}
   		\E{c_{k}} &= N_{a}(1 - \E{-\lambda_{k}\Delta t}) = \SI{128(50)}{} \quad \text{und}\\
   		t_{\sfrac{1}{2},k} &=  \SI{54(13)}{\second}.
   	\end{empheq}
    
    Die auf diese Weise bestimmten Zerfallsgesetze sind in \cref{fig:} zusammen mit der Summe beider Zerfäll
    und den ursprünglichen Messwerten ohne den Nulleffekt aufgetragen.
    
    \includeFigure[scale=0.7]{Grafiken/Theoriekurven.pdf}{Graphische Darstellung der in der Auswertung bestimmten Zerfallsgesetze und deren summierter Zerfall im Vergleich zu den Messwerten}{\label{fig:Auswertung_Rh_Ergebniss}} 
     
