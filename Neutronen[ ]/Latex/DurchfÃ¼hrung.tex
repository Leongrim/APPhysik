\subsection{Versuchsaufbau}
	Für die Durchführung des Versuchs wird der in \cref{fig:Aufbau_Aufbau} dargestellte
	Aufbau verwendet. Dieser besteht aus einem Bleiblock in dessen Inneren das zu untersuchende
	aktivierte und somit radioaktive Material eingebracht wird. Nebst diesem wird ein 
	Geiger-Müller-Zählrohr in den Bleiblock eingeschoben, um die Beta-Zerfälle und die 
	Gamma-Strahlung des jeweils untersuchten Materials zu detektieren.
	
	Das verstärkte Signal des Geiger-Müller-Zählrohrs wird zu einer alternierenden 
	Zählwerk geleitet. Diese zählt die, in einen einzustellenden Messintervall $\Delta t$,
    Anzahl der ankommenden Signale des Geiger-Müller-Zählrohrs und gibt somit nach Ablauf 
    dieser Zeit die Anzahl der detektiereten Zerfälle bzw. Abstrahlungen an.
 	Nach Ablauf eines Messintervalls schaltet das Zählwerk automatisch,
 	innerhalb eines Zeitraums von \SI{100}{\nano\second} auf einen zweiten 
 	Zähler um, um die Aufnahme der Messungen zu erleichtern.  
    
    \includeFigure[scale=0.7]{Grafiken/Aufbau.png}{Graphische Darstellung des, zur Versuchsdurchführung verwendeten, 
    Aufbaus \cite{V702}}{\label{fig:Aufbau_Aufbau}}  
   
\subsection{Versuchsdurchführung}
	
	Vor der eigentlichen Untersuchung der aktivierten Elemente muss der Nulleffekt $N_{0}$ bestimmte werden.
	Bei diesem handelte es sich um Strahlung, die von natürlich vorkommenden radioaktiven Elementen 
	sowie der Höhenstrahlung herrührt. Dieser Nulleffekt muss vor dem eigentlichen Versuch gemessen werden, 
	da das Geiger-Müller-Zählrohr bei der Untersuchung der aktivierten Elemente sowohl die Zerfälle dieser
	als auch den Nulleffekt detektiert und die Ergebnisses somit verfälscht werden. Für die Auswertung wird
	der so bestimmte Nulleffekt, dann von den aufgenommenen Messwerten abgezogen.  
	Durch die Messung des Nulleffekts über ein langes Intervall $\Delta t = \SI{900}{\second}$ lässt sich dieser
	mit ausreichender Genauigkeit bestimmen, sodass dessen statistische Fehler vernachlässigt werden kann.  

    Für die Untersuchung der radioaktiven Zerfälle wird das entsprechende Material in den Bleiblock eingebracht,
    welcher dann mit dem Geiger-Müller-Zählrohr verschlossen wird.
    Von einer bereitgestellten Tabelle werden die zur Untersuchung des jeweiligen Elements ideale
    Länge eines Messintervalls $\Delta T$ und die Gesamtmesszeit $T$ abgelesen. Für das untersuchte Rhodium 
    wird für $T = \SI{720}{\second}$ in Intervallen von $\Delta t = \SI{20}{\second}$ gemessen. 
    Das Untersuchte Indium wird anschließend für $T = \SI{3600}{\second}$ in Intervallen von 
    $\Delta t = \SI{200}{\second}$ gemessen.
    Nach Ablauf jedes Messintervalls wird, bei beiden Untersuchungen, der vom Zählwerk angezeigte Wert $N$ und die
    entsprechend Zeit $t$ notiert.    
      
	