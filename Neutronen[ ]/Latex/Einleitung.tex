Die Stabilität von Atomkernen wird maßgeblich durch das Zahlenverhältnis von Neutronen und Protonen bestimmt. Wenn die 
Anzahl der Neutronen ca. 20-30\% über der der Protonen liegt, handelt es sich um einen stabilen Kern. Ist dieses 
Kriterium nicht erfüllt, zefällt der Kern solange bis ein stabiler Zustand erreicht wird. 
\newline
\newline
Ein Maß für die Zerfallswahrscheinlichkeit eines Atomkerns ist die Halbwertszeit T$_{1/2}$, die den Zeitraum darstellt,
in dem die Hälfte der ursprünglich vorhandenen Kerne zerfällt. Da T$_{1/2}$ eine wichtige kernphysikalische Größe ist,
existieren verschiedene Möglichkeiten diese zu ermitteln. Die Methode, die in diesem Experiment angewandt wird, erlaubt
es  vergleichsweise geringe Halbwertszeiten zu messen, indem die jeweiligen zunächst stabilen Atomkerne mit
Neutronenbeschossen werden. Die so erzeugten Isotope sind recht instabil und zerfallen nach kurzer Dauer. Theoretisch
ist ebenfalls ein Beschuss mit Protonen denkbar, allerdings müssten diese aufgrund der repulsiven Kraft des Coulomb 
Potentials eine wesentlich höhere kinetische Energie aufweisen, um in den Kern vorzudringen.