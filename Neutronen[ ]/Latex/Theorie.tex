
\subsection{Aktivierung mit Neutronen}
In diesem Versuch sollen stabile Kerne durch Absorption eines zusätzlichen Neutrons in instabile Isotope verwandelt werden. Ein auf diese Weise erzeugter Kern wird als Zwischenkern oder Compoundkern bezeichnet, dessen Gesamtenergie um den Betrag der kinetischen Energie und Bindungsenergie des eingefangenen Neutrons erhöht ist. Dadurch, dass die gewonnene Energie schnell auf alle Nukleonen verteilt wird, kann der Kern, um in wieder in seinen energetischen Grundzustand überzugehen, nur ein $\gamma$-Quant emittieren. Die stattfindene Reaktion hat folgende Gestalt:
\begin{empheq}{align*}
\ce{^m_zA + ^1_0n -> _z^{m+1} A^\ast}\ce{-> _z^{m+1} A +\gamma}.
\end{empheq}
\begin{empheq}{align*}
m&=\text{Massenzahl}\\
z&=\text{Ordnungszahl}\\
\end{empheq}

Der neue Kern $_z^{m+1}A$ ist zwar nicht stabil, da er zu viele Neutronen enthält, aber aufgrund der $\gamma$-Emmission langlebiger als $_z^{m+1}A^\ast$. Eine weiter Kernreaktion ermöglicht dann die Rückkehr in einen stabileren Zustand. Hierbei wird gemäß der folgenden Reaktionsgleichung ein Neutron in ein Elektron und Proton verwandelt. 
\begin{empheq}{equation*}
\ce{_z^{m+1}A -> _{z+1}^{m+1}C + \beta^- + E_{kin} + \overline{\nu_e}}
\end{empheq}
\begin{empheq}{align*}
\overline{\nu_e}&= \text{Antineutrino}\\
\beta^-&=\text{Elektron}\\
\end{empheq}
\newline
%\newline
Der Wirkungsquerschnitt $\sigma$ ist ein Maß für die Wahrscheinlickkeit mit der ein stabiler Kern ein freies Neutron einfängt. Anschaulich ist damit eine Trefferfläche gemeint, die dem jeweiligen Targetteilchen zugeordnet wird. Zur Definition des Wirkungsquerschnitts zieht man \SI{1}{\square\centi\meter} einer dünnen Folie heran, in einer Sekunde von $n$ Elektronen getroffen wird. Die Anzahl der Einfänge wird mit $u$ bezeichnet.
\begin{empheq}{equation*}
\sigma=\frac{u}{nKd}
\end{empheq}

\begin{empheq}{align*}
K&=\text{Atome}/\si{\centi\meter\cubed}\\
d&=\text{Dicke der Goldfolie}\\
\end{empheq}
Die Wahrscheinlichkeit, dass ein Neutron eingefangen wird ist in hohem Maße abhängig von der Neutronengeschwindigkeit. Ein Kriterium zur Unterscheidung zwischen schnellen und langsamen Neutronen lässt sich mithilfe der De-Broglie Relation formulieren. Ein Neutron mit der Geschwindigkeit v hat folgende De-Broglie Wellenlänge:
\begin{empheq}{equation*}
\lambda=\frac{h}{m_n\cdot v}
\end{empheq}
Ist hierbei die Wellenlänge $\lambda$ klein gegenüber dem Radius R des Atomkerns, so kann die Streuung der Neutronen, ähnlich wie in der Optik, geometrisch betrachtet werden. Diese Herangehensweise verliert jedoch ihre Gültigkeit bei langsamen Elektronen mit $R\le \lambda$, da hier ebenfalls in Analogie zur Optik, Interferenzeffekte bei Streuungen berücksichtigt werden.
\newline
\newline
Der Wirkungsquerschnitt lässt sich als Funktion der Neutronenenergie angeben, wie Breit und Wigner herausfanden:
\begin{empheq}{equation*}
\sigma=(E)=\sigma_0\sqrt{\frac{E_{r_i}}{E}}\frac{\tilde c}{(E-E_{r_i})^2+\tilde c}
\end{empheq}
Hierbei stellen $\tilde c$ und $\sigma_0$ charakteristische Konstanten und $E_{r_i}$ die Energieniveaus des Compoundkerns dar. Unter der Annahme, dass $E_{r_i}\gg E$ folgt eine einfache Regel für die Abhängigkeit des Wirkungsquerschnitts von der Neutronenenergie:
\begin{empheq}{equation*}
\sigma \propto \frac{1}{\sqrt{E}} \propto \frac{1}{v}
\end{empheq}

Dieses Ergebnis ist konsistent mit der intuitiven Vorstellung, dass ein langsames Neutron sich längere Zeit in dem 
Wirkungsbereich der starken Kernkraft aufhält, was zu einer Erhöhung der Einfangwahrscheinlichkeit führt. 

\subsection{Erzeugung thermischer Neutronen}
Aufgrund der genannten Argumente erscheint es nun sinnvoll in diesem Versuch niedrig energetische Neutronen zu verwenden, da hier eine optimale Ausbeute an Kernreaktionen wahrscheinlich ist. Derartige Neutronen müssen allerdings zunächst durch einen geeigneten Prozess erzeugt werden. In diesem Fall werden durch Beschuss von $^9Be$ mit $\alpha$-Teilchen Neutronen freigesetzt.

\begin{equation*}
	\ce{_4^9Be + _2^4\alpha}\ce{->} \ce{_6^{12}Be +_0^1n} 
\end{equation*}

Um die kinetische Energie der Neutronen zu senken, ist die Neutronenquelle mit einem Mantel aus Paraffin umgeben. Beim Diffundieren durch diese Schicht geben die Neutronen Energie durch Stoßprozesse ab, bis ihre mittlere kinetische Energie etwa \SI{0.025}{\eV} bei ca. $T= \SI{290}{\kelvin}$ beträgt. Diese Neutronen werden als thermische Neutronen bezeichnet.


\subsection{Elemente mit einem Zerfall}

Ein Ziel dieses Versuchs ist es die Halbwertszeit des Isotops $^{115}In$ zu ermitteln. Dazu wird zunächst eine Probe des Indiums mithilfe der Neutronenquelle aktiviert, bevor die Zerfälle mit dem Geiger-Müller Zählrohr gezählt werden können. Die Aktivierung und der Zerfall laufen nach folgender Reaktion ab:
\begin{align*}
\ce{_{49}^{115}In + _0^1n} \ce{->} \ce{_{49}^{116}In -> _{50}^{116}Sn + \beta^- +\overline{\nu_e}}
\end{align*}
Für radioaktive Zerfälle gilt das Zerfallsgesetz:
\begin{align*}
N(t)&=N_0\cdot e^{-\lambda t}\\
N_0&=\text{Anzahl der Kerne bei t=0}\\
\lambda&=\text{Zerfallskonstante}
\end{align*}
Aus dieser Gleichung lässt sich durch Logarithmisierung die Halbwertzeit $T_{1/2}$ gewinnnen:
\begin{align}
\notag
\frac{1}{2}N_0&=N_0\cdot e^{-\lambda t}\\
\Rightarrow T_{1/2}&=\frac{ln2}{\lambda}
\label{eq:Theorie_Halbwertszeit}
\end{align}
Zwar ließe sich aus dieser Beziehung die Halbwertszeit errechnen, jedoch ist es schwierig einen guten Wert für N(t) zu bestimmen. Daher wählt man ein Zeitintervall $\Delta t$ und ermittelt $N_{\Delta t}(t)$ in Abhängigkeit von der Zeit mithilfe eines Strahlungsdetektors. Die Definition von $N_{\Delta t}(t)$ lautet:
\begin{align*}
N_{\Delta t}(t)=N(t)-N(t+\Delta t)
\end{align*}
Mithilfe des bereits genannten Zerfallsgesetz folgt daraus:
\begin{align*}
N_{\Delta t}(t)&=N_0e^{-\lambda t}-N_0e^{-\lambda(t+\Delta t)}\\
\Longleftrightarrow \ln N_{\Delta t}(t)&=\ln N_0(1-e^{-\lambda \Delta t})-\lambda\cdot t\\
\end{align*}
Die Wahl des richtigen Zeitintervalls $\Delta t$ ist sensibel und muss gegebenenfalls durch Vorversuche ermittelt werden. In diesem Versuch sind nützliche Werte im Labor angegeben. 



\subsection{Elemente mit zwei unterschiedlichen Zerfällen}
	Neben den Elementen die durch die Aktivierung mit Neutronen nur einfach zerfallen, 
	wie das in diesem Versuch untersuchte \ce{^{116}_{49}In}, existieren auch Elemente
	die durch die Aktivierung in zwei unterschiedlichen weisen zerfallen.
	Als Beispiele sollen hier Silber und das in diesem Versuch untersuchte Rhodium 
	betrachtet werden um zwei unterschiedliche Begründungen für diesen Umstand zu 
	klären.
	
	Einen Grund für zwei unterschiedliche Zerfälle ist die Zusammensetzung
	der natürlichen Elemente aus verschiedenen Isotopen. So besteht das natürliche 
    Silber zu ungefähr gleichen Teilen aus den Isotopen \ce{^{107}_{47}Ag} und \ce{^{109}_{47}Ag},
    die beide durch die Neutronen aktiviert werden und sich für die dabei entstandenen 
    instabilen Kerne die beiden Zerfälle 
	\begin{subequations}
	 	\begin{empheq}{align}
	 	      &\ce{^{108}_{47}Ag -> ^{108}_{48}Cd + \beta - + \overline{\nu_{e}}}\\
	 	      &\ce{^{110}_{47}Ag -> ^{110}_{48}Cd + \beta - + \overline{\nu_{e}}}
	 	\end{empheq}
	\end{subequations}
    ergeben.
    
    Neben dieser gibt es noch eine weitere Begründung für das Ablaufen von zwei
    unterschiedlichen Zerfällen. Diese tritt bei dem in diesem Versuch untersuchtem
    Rhodium \ce{^{103}_{45}Rh} auf, welches nur aus einem natürlichen Isotop besteht.
    Bei der Aktivierung der Kerne dieses Elements entsteht in \SI{10}{\percent} der
    Fälle ein, zum sonst entstehenden \ce{^{104}_{45}Rh}, isomerer Kern \ce{^{104i}_{45}Rh}.
    Dieser unterscheidet sich nicht in der Anzahl sondern in der Konfiguration der 
    Nukleonen und damit in der Energie des Kerns. Es ergeben sich somit die 
    zwei möglichen Zerfälle der aktivierten Kerne
   	\begin{subequations}
   	 	\begin{empheq}{align}
   	 		  &\ce{^{104}_{45}Rh -> ^{104}_{46}Pd + \beta - + \overline{\nu_{e}}} \quad \text{und}\\
	 	      &\cee{^{104i}_{45}Rh -> ^{104}_{45}Rh + \gamma}\ce{->  ^{104}_{46}Pd + \beta - + \overline{\nu_{e}}}. 
   	 	\end{empheq}
   	\end{subequations}

	In beiden Fällen, sowohl bei Silber als auch bei Rhodium, können beide Zefälle gleichzeitig Untersucht werden.
	Dies ist zum einen möglich da, das verwendete Geiger-Müller-Zählrohr sowohl die  Betazerfälle der Silberisotope
    und des Rhodiumisotops \ce{^{104}_{45}Rh} als auch die Gammastrahlung des isomeren Kerns \ce{^{104i}_{45}Rh}
    nachweisen kann. Zum anderen besitzen die beiden Zerfälle jeweils eines Elements unterschiedliche Halbwertszeiten,
    so dass sie sich in der Auswertung des Versuchs  lassen und so getrennt von einander bestimmen lassen.\\   
    
    Zur Auswertung beider Zerfälle werden die logarithmierten Messwerte $\Ln{N(t)}$, wie in
     \cref{fig:Theorie_DoppelteAuswertung}, gegen die Zeit $t$ aufgetragen. 
    Da einer der beiden Zerfälle, wegen der unterschiedlichen Halbwertszeit, jeweils schneller abklingt
    als der andere, lässt sich ein Übergang des zunächst gekrümmten Verlaufs der Messwerte in ein linearen beobachten.
    An der Stelle dieses Übergangs wird der Zeitpunkt $t^{*}$ gewählt, ab dem die Messwerte wie bei einem 
    einfachen Zerfall linear ausgeglichen werden.
    
    Mit dem so bestimmten langlebigen Zerfall $N_{l}(t)$ lässt sich nun durch Subtraktion der Werte $N_{l}(t_{i})$
    für $t_{i} << t^{*}$ von den Messwerten $N(t)$ der Verlauf des kurzlebigen Zerfalls bestimmen, der analog 
 	zum langlebigen Zerfall linear ausgeglichen wird. Durch dieses Vorgehen erhält man nun den das Zerfallsgesetz
 	$N_{k}(t)$ der kurzlebigen Kerne.   
       
    \includeFigure[scale=0.7]{Grafiken/DoppelteAuswertung.png}{Graphische Darstellung des Vorgehens zur Auswertung von
    zwei gleichzeitig ablaufenden Zerfällen \cite{V702}}{\label{fig:Theorie_DoppelteAuswertung}}