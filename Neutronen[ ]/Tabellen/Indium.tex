\begin{table}[htbp]
\centering\begin{tabular}{c|c|c|c|c|c|c}
$t [s]$ & $Zerfälle N$ & $\sigma_N$ & $N-N_0$ & $\sigma_{N-N_0}$ & $ln(N-N_0)$ & $\sigma_{ln}$\\
\hline \hline
200&	2853&	53&	2785&	54&	7.932&	0.019\\
400&  2617&	51&	2549&	52&	7.843&	0.020\\
600&	2535&	50&	2467&	51&	7.811&	0.021\\
800&	2355&	49&	2287&	49&	7.735&	0.022\\
1000&	2437&	49&	2369&	50&	7.770&	0.021\\
1200&	2233&	47&	2165&	48&	7.680&  0.022\\
1400&	2161&	46&	2093&	47&	7.646&	0.023\\
1600&	2149&	46&	2081&	47&	7.641&	0.023\\
1800&	1917&	44&	1849&	45&	7.522&	0.024\\
2000&	1816&	43&	1748&	43&	7.466&	0.025\\
2200&	1843&	43&	1775&	44&	7.482&	0.025\\
2400&	1639&	40&	1571&	41&	7.359&	0.026\\
2600&	1711&	41&	1643&	42&	7.404&	0.026\\
2800&	1625&	40&	1557&	41&	7.351&	0.026\\
3000&	1585&	40&	1517&	41&	7.324&	0.027\\
3200&	1432&	38&	1364&	39&	7.218&	0.028\\
3400&	1443&	38&	1375&	39&	7.226&	0.028\\
3600&	1358&	37&	1290&	38&	7.162&	0.029\\

\end{tabular}
\caption{Messdaten und Ergebnisse für die Bestimmung der Halbwertszeit von Indium}
\label{fig:Tab1}
\end{table}