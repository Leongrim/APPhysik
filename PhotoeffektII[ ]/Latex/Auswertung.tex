Im folgenden Abschnitt sind während des Versuchs aufgenommenen Messwerte
und die aus diesen berechneten Ergebnisse sowohl tabellarisch als auch grafische 
dargestellt. An entsprechender Stelle sind Anmerkungen und Erklärungen zu den Rechnungen 
und Ergebnissen gegeben. 

Die Messfehler der aufgenommenen Größen wurden allgemein mit der kleinsten 
Skaleneinteilung des jeweiligen Messgeräts angenommen.



\subsection{Bestimmung der Dispersionsspannung von Dioden}

	Die für die fünf Dioden aufgenommenen Messwerte für Strom $I$ und Spannung $U$
	sind für jeweils eine Diode in den Tabellen \ref{tab:Auswertung_Diode_Blau},
	\ref{tab:Auswertung_Diode_Gruen}, \ref{tab:Auswertung_Diode_Gelb},
	 \ref{tab:Auswertung_Diode_Orange} und \ref{tab:Auswertung_Diode_Rot} aufgelistet.
	
	In den Abbildungen \ref{fig:Auswertung_Diode_Blau} bis \ref{fig:Auswertung_Diode_Rot}
	sind die
	Messwerte für die Stromstärke $I$ gegen die der Spannung, der jeweiligen Diode, 
	aufgetragen. In diesen Abbildungen befinden sich auch die jeweiligen Regressionsgeraden 
	der Messwerte für den annähernd linearen Teil der $I$-$U$-Kennlinie. Diese Regressionen
	wurde mit Hilf der \emph{Python}-Bibliothek \emph{SciPy} \cite{SciPy} und dem Ansatz
	\begin{empheq}{equation}
		I(U) = A \cdot U + I_{0}
	\end{empheq}  
	bestimmt.
	
	Die Parameter der fünf Regressionsgeraden sind in \cref{tab:Auswertung_Parameter} zu 
	finden. In dieser Tabelle befinden sich ebenfalls die, als Nullstellen der 
	Regressionsgeraden bestimmten, Dispersionsspannungen der jeweiligen Diode.
	
	
	\begin{table}[!h]
	\centering
	\begin{tabular}{|c|c|c|c|}
		\hline
		Spannung & Strom & Spannung & Strom\\
		$U$ [\si{\volt}] & $I$ [\si{\ampere}] & $U$ [\si{\volt}] & $I$ [\si{\ampere}]\\
\hline\hline
		\num{4.60(5)} & \num{20.0(5)} & \num{4.15(5)} & \num{7.0(5)}\\
		\num{4.55(5)} & \num{19.0(5)} & \num{4.10(5)} & \num{6.0(5)}\\
		\num{4.50(5)} & \num{18.0(5)} & \num{4.05(5)} & \num{5.0(5)}\\
		\num{4.45(5)} & \num{16.0(5)} & \num{3.95(5)} & \num{4.0(5)}\\
		\num{4.40(5)} & \num{14.0(5)} & \num{3.85(5)} & \num{3.0(5)}\\
		\num{4.35(5)} & \num{13.0(5)} & \num{3.80(5)} & \num{2.0(5)}\\
		\num{4.30(5)} & \num{11.0(5)} & \num{3.70(5)} & \num{1.5(5)}\\
		\num{4.25(5)} & \num{10.0(5)} & \num{3.60(5)} & \num{1.0(5)}\\
		\num{4.20(5)} & \num{9.0(5)} & \num{3.40(5)} & \num{0.5(5)}\\
		\num{4.15(5)} & \num{8.0(5)}\\
		\hline
	\end{tabular}
	\caption{Messwerte der Spannung und des Stroms f�r die blaue Diode mit der Wellenl�nge \SI{465}{\nm} \label{tab:Auswertung_Diode_Blau}}
\end{table}
  
	
	\includeFigure[scale=0.7]{Grafiken/Diode_Blau.pdf}{Grafische % 
		Darstellung der Messwerte für die $I$-$U$-Kennlinie der %
		 blauen Diode und der Bestimmung der Dispersionsspannung %
		 }{\label{fig:Auswertung_Diode_Blau}}
	
	\begin{table}[!h]
	\centering
	\begin{tabular}{|c|c|c|c|}
		\hline
		Spannung & Strom & Spannung & Strom\\
		$U$ [\si{\volt}] & $I$ [\si{\ampere}] & $U$ [\si{\volt}] & $I$ [\si{\ampere}]\\
\hline\hline
		\num{2.25(5)} & \num{19.0(5)} & \num{2.05(5)} & \num{8.0(5)}\\
		\num{2.25(5)} & \num{18.0(5)} & \num{2.00(5)} & \num{5.0(5)}\\
		\num{2.20(5)} & \num{17.0(5)} & \num{1.95(5)} & \num{2.5(5)}\\
		\num{2.20(5)} & \num{16.0(5)} & \num{1.90(5)} & \num{1.0(5)}\\
		\num{2.15(5)} & \num{15.0(5)} & \num{1.85(5)} & \num{0.5(5)}\\
		\num{2.10(5)} & \num{12.0(5)}\\
		\hline
	\end{tabular}
	\caption{Messwerte der Spannung und des Stroms f�r die gr�ne Diode mit der Wellenl�nge \SI{565}{\nm}  \label{tab:Auswertung_Diode_Gruen}}
\end{table}
  
	
		\includeFigure[scale=0.7]{Grafiken/Diode_Gruen.pdf}{Grafische % 
			Darstellung der Messwerte für die $I$-$U$-Kennlinie der %
			grünen Diode und der Bestimmung der Dispersionsspannung %
		}{\label{fig:Auswertung_Diode_Gruen}}
	
	
	\begin{table}[!h]
	\centering
	\begin{tabular}{|c|c|c|c|}
		\hline
		Spannung & Strom & Spannung & Strom\\
		$U$ [\si{\volt}] & $I$ [\si{\ampere}] & $U$ [\si{\volt}] & $I$ [\si{\ampere}]\\
\hline\hline
		\num{2.10(5)} & \num{19.0(5)} & \num{1.95(5)} & \num{9.0(5)}\\
		\num{2.05(5)} & \num{17.0(5)} & \num{1.90(5)} & \num{6.0(5)}\\
		\num{2.05(5)} & \num{15.0(5)} & \num{1.90(5)} & \num{5.0(5)}\\
		\num{2.00(5)} & \num{13.0(5)} & \num{1.85(5)} & \num{3.0(5)}\\
		\num{2.00(5)} & \num{12.0(5)} & \num{1.80(5)} & \num{1.0(5)}\\
		\hline
	\end{tabular}
	\caption{Messwerte der Spannung und des Stroms f�r die gelbe Diode  mit der Wellenl�nge \SI{585}{\nm}  \label{tab:Auswertung_Diode_Gelb}}
\end{table}
  
	
			\includeFigure[scale=0.7]{Grafiken/Diode_Gelb.pdf}{Grafische % 
				Darstellung der Messwerte für die $I$-$U$-Kennlinie der %
				gelben Diode und der Bestimmung der Dispersionsspannung%
			}{\label{fig:Auswertung_Diode_Gelb}}
	
	\begin{table}[!h]
	\centering
	\begin{tabular}{|c|c|c|c|}
		\hline
		Spannung & Strom & Spannung & Strom\\
		$U$ [\si{\volt}] & $I$ [\si{\ampere}] & $U$ [\si{\volt}] & $I$ [\si{\ampere}]\\
\hline\hline
		\num{2.05(5)} & \num{19.0(5)} & \num{1.85(5)} & \num{6.0(5)}\\
		\num{2.00(5)} & \num{17.0(5)} & \num{1.85(5)} & \num{5.0(5)}\\
		\num{2.00(5)} & \num{15.0(5)} & \num{1.80(5)} & \num{3.0(5)}\\
		\num{1.95(5)} & \num{13.0(5)} & \num{1.75(5)} & \num{2.0(5)}\\
		\num{1.95(5)} & \num{12.0(5)} & \num{1.70(5)} & \num{1.0(5)}\\
		\num{1.90(5)} & \num{9.0(5)}\\
		\hline
	\end{tabular}
	\caption{Messwerte der Spannung und des Stroms f�r die orangefarbene Diode  mit der Wellenl�nge \SI{635}{\nm}  \label{tab:Auswertung_Diode_Orange}}
\end{table}
  
	
		\includeFigure[scale=0.7]{Grafiken/Diode_Orange.pdf}{Grafische % 
			Darstellung der Messwerte für die $I$-$U$-Kennlinie der %
			orangefarbenen Diode und der Bestimmung der Dispersionsspannung%
		}{\label{fig:Auswertung_Diode_Orange}}
	
	\begin{table}[!h]
	\centering
	\begin{tabular}{|c|c||c|c|}
		\hline
		Spannung & Strom & Spannung & Strom\\
		$U$ [\si{\volt}] & $I$ [\si{\ampere}] & $U$ [\si{\volt}] & $I$ [\si{\ampere}]\\
\hline\hline
		\num{2.25(5)} & \num{19.0(5)} & \num{2.05(5)} & \num{9.0(5)}\\
		\num{2.25(5)} & \num{17.0(5)} & \num{2.05(5)} & \num{8.0(5)}\\
		\num{2.20(5)} & \num{16.0(5)} & \num{2.00(5)} & \num{6.0(5)}\\
		\num{2.15(5)} & \num{14.0(5)} & \num{1.95(5)} & \num{4.0(5)}\\
		\num{2.15(5)} & \num{13.0(5)} & \num{1.90(5)} & \num{2.0(5)}\\
		\num{2.10(5)} & \num{11.0(5)} & \num{1.85(5)} & \num{0.5(5)}\\
		\hline
	\end{tabular}
	\caption{Messwerte der Spannung und des Stroms für die roten Diode  mit der Wellenlänge \SI{657}{\nm}  \label{tab:Auswertung_Diode_Rot}}
\end{table}
  
	
		\includeFigure[scale=0.7]{Grafiken/Diode_Rot.pdf}{Grafische % 
			Darstellung der Messwerte für die $I$-$U$-Kennlinie der %
			roten Diode und der Bestimmung der Dispersionsspannung%
		}{\label{fig:Auswertung_Diode_Rot}}

		\begin{table}[!h]
	\centering
	\begin{tabular}{|c|c|c|c|c|}
		\hline
		Wellenlänge & Frequenz & Steigung & y-Achsenabschnitt & Dispersionsspannung\\
		$\lambda$ [\si{\nm}] & $f$ [\si{\peta\hertz}] & $A$ [\si{\milli\ampere\per\volt}] & $I_{0}$ [\si{\milli\ampere}] & $U_{D}$ [\si{\volt}]\\
\hline\hline
		\num{465} & \num{0.645} & \num{28(1)} & \num{-110(3)} & \num{3.887}\\
		\num{565} & \num{0.531} & \num{54(3)} & \num{-103(6)} & \num{1.898}\\
		\num{585} & \num{0.512} & \num{67(3)} & \num{-121(6)} & \num{1.813}\\
		\num{635} & \num{0.472} & \num{67(3)} & \num{-117(6)} & \num{1.763}\\
		\num{657} & \num{0.456} & \num{47(2)} & \num{-88(3)} & \num{1.865}\\
		\hline
	\end{tabular}
	\caption{Regressionsparameter für die jeweils angegbenen Wellenlängen und Frequenzen \label{tab:Auswertung_Parameter}}
\end{table}

		
\subsection{Bestimmung des Abhängigkeit der Dispersionsspannung von der Lichtfrequenz}


	Die Werte für die Dispersionsspannung $U_{D}$ aus \cref{tab:Auswertung_Parameter} 
	sind in \cref{fig:Auswertung_Dispersion_Frequenz} gegen die Frequenzen des emittierten Lichtes der jeweiligen 
	Diode aufgetragen.

	\includeFigure[scale=0.7]{Grafiken/Dispersion_Frequenz.pdf}{%
	Grafische Darstellung des Zusammenhangs zwischen der Dispersionsspannung 
	und der Frequenz einer Diode mit Regressionsgerade %
	}{\label{fig:Auswertung_Dispersion_Frequenz}}  
	
	Die lineare Regression wurde wiederum mit \emph{SciPy} berechnet, wodurch sich
	die Parameter für den Ansatz 
	\begin{empheq}{equation}
		U(f) = B \cdot f + U_{0}
	\end{empheq}  
	\addtocounter{equation}{-1}
	\begin{subequations}
		\begin{empheq}{align}
			B &= \SI{1.1(3)e-14}{\volt\second} \\
			U_{0} &= \SI{-4(2)}{\volt} 
		\end{empheq}  
	\end{subequations}
	ergeben.
	Der Theorie nach entspricht die Steigung dieser Geraden gerade dem Quotienten 
	aus dem planckschen Wirkungsquantum $h$ und der Elementarladung  $e_{0}$, sodass 
		\begin{empheq}{equation}
			\label{val:Auswertung_h_e0}
			\dfrac{h}{e_{0}} = B = \SI{1.1(3)e-14}{\volt\second}
		\end{empheq}  
	gilt.