Eine Leuchtdiode Funktioniert wie die Umkehrung des Photoeffektes. Es befinden sich zwei verschiedene Materialien an einander. Je nach Materialien ändert sich die Wellenlänge $\lambda$ beziehungsweise die Frequenz $f$. Wenn an die Diode eine Spannung angelegt wird werden die Elektronen mit der Energie
\begin{align}
E=e_0U
\end{align}
beschleunigt. Die Elektronen geben dabei ihre Energie in Form von Strahlung oder an die Gitteratome und regen sie dadurch zum schwingen an.
\begin{align}
e_0U=hf+A_S,
\end{align}
wobei $A_S$ die Energie ist die an die Gitteratome abgegeben wird.
