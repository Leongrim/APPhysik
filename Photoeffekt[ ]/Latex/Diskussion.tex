Im Folgenden werden die in \cref{sec:Auswertung} erhaltenen Ergebnisse
noch einmal abschließend diskutiert und dabei auf ihre Plausibilität hin überprüft.
Dabei wird auch Bezug auf den Versuchsaufbau und die -durchführung genommen.\\

Die im ersten Versuchsteil aufgenommenen Messdaten, zeigen in der 
gewählten Darstellung $\sqrt{I} \propto U$ den theoretisch erwarteten 
linearen Verlauf und sind in sofern plausibel. Weiter ist
an den Werten zu erkennen, dass der Photostrom proportional zur Lichtintensität ist.
Der Photostrom der intensiven Spektrallinien, wie der ersten violetten
\cref{fig:Messwerte_Violett1}, ist zum Beispiel in Relation zu den wenig intensiven 
Linien wie der cyanen \cref{fig:Messwerte_Cyan} um den Faktor $6$ größer.
Von den für jede Frequenz unterschiedlichen Grenzspannungen $U_{g}$ ist 
abzuleiten, dass die Energie der ausgelösten Elektronen in Abhängigkeit zu der 
jeweiligen Frequenz steht. Lediglich die Unabhängigkeit der Elektronenenergie von der 
Lichtintensität konnte mit dem verwendeten Versuchsaufbau nicht untersucht werden,
da die Intensität der Spektrallinien nicht verändert werden konnte.\\

Der durch lineare Regression aus den berechneten Werten der Grenzspannungen 
$U_{g}$ und der jeweiligen Frequenz des Lichtes $f$ bestimmte Quotient
$\tfrac{h}{e_{0}}$ \cref{val:Auswertung_h_e0} weist mit \SI{17}{\percent} eine
deutliche Abweichung vom Literaturwert \SI{4.136e-15}{\volt\second}\cite{SciPy} auf.
Dies lässt sich durch die, mit $5$ Messung, geringe Anzahl an Messwerten
begründen.
Im Rahmen dieser Genauigkeit ist es möglich, anhand der erhaltenen Auslösearbeit
$A_{k}$ \cref{val:Auswertung_Wa}, die eine Stoffkonstante darstellt, das Material 
der Photokathode zu bestimmen. So ergibt sich aus Vergleichen mit der Literatur 
\cite{Mende09} Wolfram mit einer Auslösearbeit von $A_{k}(\ce{W}) = \SI{1.4}{\eV}$
als wahrscheinlichstes Material.\\

Die Messung zur Abhängigkeit des Photostrom von der Bremsspannung liefern
ebenfalls das in der Anleitung \cite{V500} beschriebene Ergebnis.
Entgegen dem zunächst zu vermutenden Verlauf - konstanter Photostrom bis zur
Grenzspannung, danach kein Photostrom - zeigen die Messwerte einen mit steigender 
Bremsspannung abnehmenden Photostrom, der für große Spannungen, sogar negativ
wird. Bei Betrachtung des Verlaufs bei höheren Beschleunigungsspannungen 
(negative Bremsspannungen) ist zu erkennen, dass sich ein Sättigungswert 
des Photostroms einstellt. \\
Der generelle Kurvenverlauf lässt sich durch die Verteilung der 
Energie der Elektronen in der Photokathode erklären. Da nicht jedes 
Elektron die gleiche Energie besitzt, sondern diese \enquote{Startenergie}
statistisch verteilt ist (Fermi-Dirac-Statistik), ist der Photostrom diesbezüglich 
nicht nur von der Energie der Photonen abhängig. 

Negative Photoströme sind dadurch erreichbar, dass eine hohe Bremsspannung 
angelegt wird. Durch dies ist es den Elektronen die aus der Photokathode ausgelöst 
werden nicht möglich die Anode zu erreichen, da sie nicht genug Energie besitzen.
Durch das Auslösen der Elektronen entsteht ein Elektronendefizit auf der 
Kathode, welches dazuführt, dass sich ein Strom von Anode zu Kathode einstellt,
da die Anode im Vergleich einen Elektronenüberschuss aufweist. Für hohe 
Bremsspannungen, vertauschen sich demnach die Rollen von Kathode und Anode.
Tritt dieser Effekt schon bei energiearmen Licht auf, so ist darauf zu schließen,
dass das Anodenmaterial eine dem Kathodenmaterial ähnliche Auslösearbeit besitzt,
da nur wenig Energie nötig ist um die Elektronen aus der Anode auszulösen.

Die Einstellung eines Sättigungswertes für hohe beschleunigende Spannungen ist 
durch die Abhängigkeit des Photostroms von der Lichtintensität zu begründen.     
Dies ist der Fall, da für eine feste Intensität (Energie pro Zeit und Fläche)
pro Zeiteinheit eine feste Anzahl an Elektronen ausgelöst wird, da jedes 
Photon höchstens ein Elektron auslösen kann. Das Erhöhen der beschleunigenden 
Spannung führt nun dazu, dass alle Elektronen die pro Zeit ausgelöst werden auch die 
Anode erreichen. Dieses Anzahl an Elektronen pro Zeit (Strom) ist wie oben erklärt 
durch die Intensität begrenzt und somit auch der resultierende Photostrom.   
 
 

   