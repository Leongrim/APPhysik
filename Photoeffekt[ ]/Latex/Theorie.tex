Experimentelle Beobachtungen, die beim Photoeffekt festzustellen sind, stimmen nicht mit denen aus der klassischen Betrachtung überein. Diese werden nun untersucht.

\subsection{Klassische Erklärung des Photoeffektes}
 Die klassische Annahme ist nun, dass die Elektronen durch die einfallende elektromagnetische Welle zu Schwingungen angeregt werden. Dabei wird pro Zeiteinheit eine gewisse Energie auf das Elektron übertragen. Die Schwingungsamplitude (bzw. die Energie) steigt und wird immer größer, bis sie irgendwann ausreicht um sich aus dem Atom zu lösen.\\
 %
Daraus folgt, dass die Energien der Elektronen abhängig von der Lichtintensität sein müssten und es müssten auch bestimmte Resonanzfrequenzen beobachtbar sein, bei denen der Photoeffekt besonders gut auftritt.

\subsection{Widersprüche innerhalb der klassischen Betrachtung des Photoeffektes}
Betrachtet man das Licht als elektromagnetische Welle müssten, egal bei welcher Wellenlänge $\lambda$ die Bestrahlung erfolgt, Elektronen austreten, solange man nur lange genug abwartet bis die Welle eine Energie $E \geq W_A$ auf das Metall übertragen hat.\\
Die Beobachtungen die man macht, stehen jedoch im Widerspruch zu diesen Annahmen.\\
Zu nächst treten Elektronen erst unterhalb einer bestimmten Lichtwellenlänge $\lambda_gr$ aus, darüber hinaus hängt die Energie eines ausgetretenen Elektrons nur von der Wellenlänge und nicht von der Intensität des Lichts ab. Die Anzahl der ausgetretenen Elektronen ist außerdem abhängig von der Intensität\\

Eine Erklärung für diese Phänomene liefert dabei die Annahme, dass das Licht auch als Teilchen betrachtet werden muss, die jeweils eine genau definierte Energiemenge $E_{Ph}$ besitzen.

\begin{equation}
E_{Ph} = h \cdot \nu
\end{equation}
mit $h$ dem planck'schen Wirkungsquantum.\\
Die Frequenz $\nu$, Wellenlänge $\lambda$ und Ausbreitungsgeschwindigkeit $c$ sind dabei über die Beziehung
\begin{equation}
\lambda = c \cdot \nu
\end{equation}
mit einander verknüpft.\\

Aus der sog. Korpuskulartheorie des Lichtes folgt, dass man den Photoeffekt als Stoß eines Photons mit einem Elektron auffassen kann, das dadurch seine Energie übergibt. Wichtig ist dabei, dass das Photon immer nur seine gesamte Energie abgeben kann (entspricht dem inelastischen Stoß) und es somit vernichtet wird.\\
Daraus ergibt sich, dass eine Grenzwellenlänge existiert ab der erst der Photoeffekt auftritt. Denn ist $E < W_A$ reicht die Energie nicht aus um das Elektron aus dem Festkörper zu lösen.\\
Auch die Intensitätsabhängigkeit des Photostroms lässt sich somit erklären, da bei höherer Intensität mehr Photonen auf die Oberfläche treffen und entsprechend mehr Elektronen ausgelöst werden.\\

\subsubsection{Der Teilchen-Welle-Dualismus}
Die Korpuskluartheorie liefert jedoch keine Erklärung für das Auftreten von Beugungsmustern am Spalt oder Interferenzerscheinungen. Diese lassen 
sich nur durch eine Wellentheorie des Lichtes erklären.\\
Das bedeutet, dass das Licht sowohl Teilchen- als auch Welleneigenschaften besitzt.\\
Man bezeichnet dies als Welle-Teilchen Dualismus des Lichtes.
