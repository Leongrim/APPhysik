Zunächst lässt sich die Entladung des RC-Kreises beschreiben durch:
\begin{equation} 
Q(t) = Q_0\cdot exp(-\frac{t}{RC})
\end{equation}
Die Aufladung durch:
\begin{equation} 
Q(t) = C U_0 (1 - exp(-\frac{t}{RC}))
\end{equation}
Beim Wechsel von Gleich- auf Wechselstrom wird eine Spannung mit der Frequenz $\omega$ angelegt:
\begin{equation} 
U(t) = U_0 cos(\omega t)
\end{equation}
mit $\omega << \frac{1}{RC}$. \\
Für die Amplitude gilt nun:
\begin{equation} 
A(\omega) = \frac{U_0}{\sqrt{1 + \omega ^2R^2C^2}}
\end{equation}
Wegen der gesteigerten Frequenz lässt nun die resultierende Phasenverschiebung bestimmen durch:
\begin{equation} 
\varphi = \frac{a}{b}\cdot 2\pi 
\end{equation}
bzw.
\begin{equation} 
\varphi = \frac{a}{b}\cdot 360 
\end{equation}
Über die Theorie zur Integration des RC-Kreises lässt sich folgendes sagen:
\begin{equation} 
U_c(t) = \frac{1}{RC}\int_{0}^{t} U(t') dt'
\end{equation}
