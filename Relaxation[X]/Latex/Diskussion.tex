In diesem Abschnitt werden die in \cref{sec:Auswertung} erhaltenen Ergebnisse
noch einmal abschließend diskutiert und dabei auf Plausibilität geprüft.
Dabei wird auch ein Bezug zu dem verwendeten Aufbau und der Versuchsdurchführung genommen.\\

Der Vergleich, der in den Abschnitten \ref{sec:Auswertung_Entladung} bis \ref{sec:Auswertung_Phasendifferenz}
erhaltenen Werte \cref{eq:Auswertung_RC1}, \cref{eq:Auswertung_RC2} und \cref{eq:Auswertung_RC3} 
für die Zeitkonstante $RC$ des RC-Gliedes teilweise weisen systematische Abweichungen auf.
So sind die beiden Zeitkonstanten, die über die Frequenzabhängigkeiten von Amplitude beziehungsweise
Phasendifferenz bestimmt wurden mit einem relativen Unterschied von \SI{2}{\percent} sehr nah beieinander
und liegen jeweils innerhalb einer Standardabweichung des anderen Wertes. 
Dahingegen weißt der über die Entladung des Kondensators bestimmte Wert der Zeitkonstante mit einem
mittleren Unterschied von \SI{58}{\percent} eine wesentlich größere Abweichung zu den anderen Werten
auf.\\
Eine Begründung dieser starken Abweichung, könnte die Genauigkeit des jeweils verwendeten Messverfahrens
sein, die wie sich mit den gewonnenen Werten vermuten lässt bei den beiden Messungen zur 
Frequenzabhängigkeit größer ist.  Diese Vermutung lässt sich anstellen, da für diese beiden Verfahren 
insgesamt mehr Messwerte zur Verfügung standen und somit ein genaueres Ergebnis zu erwarten 
ist. Ferner ist ein Ergebnis, welches mit nur geringen Abweichungen aus zwei unterschiedlichen Versuchen 
folgt als plausibler anzunehmen als ein einzelnes abweichendes Ergebnis.
Die Abweichung der Messung über die Entladung könnte darin begründet sein, dass der theoretisch 
prognostizierte exponentielle Verlauf in der Realität nicht vollständig richtig ist. Dies 
lässt sich auch an den letzten Messwerten in \cref{fig:Auswertung_EntladungLog} die dem linearen Verlauf 
nicht mehr exakt folgen.\\
Andrerseits zeigt das in \cref{sec:Auswertung_Amplitude_Polar} erhaltene Ergebnis, dass die Amplitude 
der Kondensatorspannung auch eine Abhängigkeit von der Phasendifferenz zwischen Generator- und 
Kondensatorspannung besitzt. Diese wiederum ist eine weitere mögliche Begründung die geringe Abweichungen
zwischen den beiden frequenzabhängigen Messungen, wodurch letztlich nur die größere Menge an Messdaten 
für diese beiden Versuche bleibt, die dieses Ergebnis plausibler erscheinen lässt.\\

Das Wirken eines RC-Gliedes als Integrator ist in \cref{sec:Auswertung_Integrator} gezeigt,
da die erhaltenen Verläufe der Kondensatorspannung proportional zu den berechneten Stammfunktionen  
sind. Unterschiede lassen sich dabei lediglich in Phasenverschiebung und Amplitude der einzelnen 
Kondensatomspannungen finden.
