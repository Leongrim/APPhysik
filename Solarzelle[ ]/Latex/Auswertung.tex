Im folgenden Abschnitt sind die während des Versuchs aufgenommenen Messwerte und die 
daraus berechneten Ergebnisse tabellarisch und graphisch dargestellt.
An entsprechender Stelle sind Anmerkungen und Erklärungen zu den Rechnungen und
Ergebnissen gegeben.
Als Messfehler wurde allgemein die kleinste Skaleneinteilung bzw. die
letzte angezeigte Stelle des jeweils verwendeten Messinstruments angenommen. 
Die für die Fehlerrechnung verwendeten Gleichungen sind in \cref{sec:Fehlerrechnung}
aufgeführt und werden mit römischen Ziffern referenziert. 

\includeFigure[scale=0.3]{Grafiken/Intensitaet}{Intensität des Lichtes einer Halogenlampe $J_{Ph}$ %
	in Abhängigkeit des Abstand $d$ \cite[verändert]{NHV1}}{\label{fig:Auswertung_Intensitaet}}

Die in der Auswertung benötigten Fläche der Solarzelle $A_{SZ}$ berechnet sich aus der Anzahl $n = \num{8}$
der quadratischen Elemente, mit der jeweiligen Seitenlänge $a = \SI{2.5(1)}{\cm}$ zu
\begin{empheq}{equation}
	\label{val:Auswertung_Solarzelle_Flaeche}
	A_{SZ} = n \cdot a^{2} = \SI{50(4)}{\square\cm}.
\end{empheq} 


\subsection{Bestimmung des Kurzschlussstroms der Solarzelle}
	
	Die aufgenommenen Messwerte für den Abstand $d'$ den offsetbereinigen Abstand\\
	$d = d' - \del{d_{\text{off}, 1} + d_{\text{off}, 2}}$ mit
	$d_{\text{off}, 1} = \SI{3.5(1)}{\cm}$ und $d_{\text{off}, 1} = \SI{14.0(1)}{\cm}$ 
	und der in diesem Abstand gemessene Kurzschlussstrom $I_{K}$ sind in \cref{tab:Auswertung_Kurzschlussstrom}
	eingetragen. Außerdem sind die für den jeweiligen Abstand die aus \cref{fig:Auswertung_Intensitaet}
	abgelesenen Werte für die Lichtintensität der Halogenlampe angegeben.
	

	
	\begin{table}[!h]
	\centering
	\begin{tabular}{|c|c|c|c|}
		\hline
		Kurzschlussstrom & Abstand+Offset & Abstand & Lichtintensität\\
		$I_{K}$ [\si{\milli\ampere}] & $d'$ [\si{\cm}] & $d$ [\si{\cm}] & $J_{Ph}$ [\si{\milli\watt}]\\
\hline\hline
		\num{-31.7(1)} & \num{99.0(1)} & \num{81.5(2)} & \num{6.4(1)}\\
		\num{-34.2(1)} & \num{95.0(1)} & \num{77.5(2)} & \num{7.2(1)}\\
		\num{-37.3(1)} & \num{91.0(1)} & \num{73.5(2)} & \num{8.0(1)}\\
		\num{-40.8(1)} & \num{87.0(1)} & \num{69.5(2)} & \num{9.0(1)}\\
		\num{-44.9(1)} & \num{83.0(1)} & \num{65.5(2)} & \num{9.8(1)}\\
		\num{-47.3(1)} & \num{81.0(1)} & \num{63.5(2)} & \num{10(1)}\\
		\num{-49.9(1)} & \num{79.0(1)} & \num{61.5(2)} & \num{11(1)}\\
		\num{-50.1(1)} & \num{77.0(1)} & \num{59.5(2)} & \num{12(1)}\\
		\num{-55.6(1)} & \num{75.0(1)} & \num{57.5(2)} & \num{12(1)}\\
		\num{-56.0(1)} & \num{73.0(1)} & \num{55.5(2)} & \num{13(1)}\\
		\num{-62.0(1)} & \num{71.0(1)} & \num{53.5(2)} & \num{14(1)}\\
		\num{-70.8(1)} & \num{67.0(1)} & \num{49.5(2)} & \num{16(1)}\\
		\num{-81.4(1)} & \num{63.0(1)} & \num{45.5(2)} & \num{17(1)}\\
		\num{-92.6(1)} & \num{59.0(1)} & \num{41.5(2)} & \num{19(1)}\\
		\num{-97.1(1)} & \num{57.0(1)} & \num{39.5(2)} & \num{21(1)}\\
		\hline
	\end{tabular}
	\caption{Messwerte zur Bestimmung der Abhängigkeit des Kurzschlussstrom $I_{K}$ von der Lichtintensität $J_{Ph}$ \label{tab:Auswertung_Kurzschlussstorm}}
\end{table}
	
	
	In \cref{fig:Auswertung_Kurzschlussstrom} sind die Messwerte für den Kurzschlussstrom $I_{K}$ gegen die 
	Lichtintensität $J$ aufgetragen. Die Parameter der ebenfalls eingezeichnete und mit Hilfe der
	\emph{Python}-Bibliothek \emph{SciPy} \cite{SciPy} bestimmte Regressionsgerade der Form
	\begin{empheq}{equation}
		I_{K}(J) = A \cdot J + I_{K,0}
	\end{empheq} sind
	\addtocounter{equation}{-1}
	\begin{subequations}
		\begin{empheq}{align}
			A &= \SI{-4.8(2)}{\ampere\per\watt}\quad \text{und}\\
			I_{K,0} &= \SI{1(2)}{\ampere}.
		\end{empheq}
	\end{subequations}
	
	\includeFigure[scale=0.7]{Grafiken/Kurzschlussstrom.pdf}{Grafische Darstellung des Zusammenhangs zwischen dem % 
	Kurzschlussstrom $ I_{K} $ und der Lichtintensität $J_{Ph}$}{\label{fig:Auswertung_Kurzschlussstrom}}
	
\subsection{Bestimmung der Leerlaufspannung der Solarzelle}
	
	\cref{tab:Auswertung_Leerlaufspannung} enthält die Messwerte der Leerlaufspannung $ U_{L} $,
	des Abstandes $ d $ und die entsprechenden, aus \cref{fig:Auswertung_Intensitaet} abgelesenen,
	Lichtintensitäten $ J_{Ph} $. Die grafische Darstellung der Abhängigkeit von Leerlaufspannung $ U_{L} $
	und Intensität $ J_{Ph} $ ist in \cref{fig:Auswertung_Leerlaufspannung} gegeben.


	\begin{table}[!h]
	\centering
	\begin{tabular}{|c|c|c|c|}
		\hline
		Leerlaufspannung & Abstand+Offset & Abstand & Lichtintensität\\
		$U_{L}$ [\si{\volt}] & $d'$ [\si{\cm}] & $d$ [\si{\cm}] & $J_{Ph}$ [\si{\milli\watt}]\\
\hline\hline
		\num{1.87(1)} & \num{99.0(1)} & \num{81.5(2)} & \num{6.4(1)}\\
		\num{1.89(1)} & \num{95.0(1)} & \num{77.5(2)} & \num{7.2(1)}\\
		\num{1.92(1)} & \num{91.0(1)} & \num{73.5(2)} & \num{8.0(1)}\\
		\num{1.94(1)} & \num{87.0(1)} & \num{69.5(2)} & \num{9.0(1)}\\
		\num{1.96(1)} & \num{83.0(1)} & \num{65.5(2)} & \num{9.8(1)}\\
		\num{1.98(1)} & \num{81.0(1)} & \num{63.5(2)} & \num{10(1)}\\
		\num{2.00(1)} & \num{74.0(1)} & \num{56.5(2)} & \num{13(1)}\\
		\num{2.01(1)} & \num{73.0(1)} & \num{55.5(2)} & \num{13(1)}\\
		\num{2.02(1)} & \num{71.0(1)} & \num{53.5(2)} & \num{14(1)}\\
		\num{2.03(1)} & \num{69.0(1)} & \num{51.5(2)} & \num{14(1)}\\
		\num{2.04(1)} & \num{67.0(1)} & \num{49.5(2)} & \num{16(1)}\\
		\num{2.05(1)} & \num{63.0(1)} & \num{45.5(2)} & \num{17(1)}\\
		\num{2.08(1)} & \num{59.0(1)} & \num{41.5(2)} & \num{19(1)}\\
		\num{2.09(1)} & \num{55.0(1)} & \num{37.5(2)} & \num{22(1)}\\
		\num{2.11(1)} & \num{51.0(1)} & \num{33.5(2)} & \num{24(1)}\\
		\hline
	\end{tabular}
	\caption{Messwerte zur Bestimmung der Abhängigkeit der Leerlaufspannung $U_{L}$ von der Lichtintensität $J_{Ph}$ \label{tab:Auswertung_Leerlaufspannung}}
\end{table}

	
	\includeFigure[scale=0.7]{Grafiken/Leerlaufspannung.pdf}{Grafische Darstellung des Zusammenhangs zwischen der % 
	Leerlaufspannung $ U_{L} $ und der Lichtintensität $J_{Ph}$}{\label{fig:Auswertung_Leerlaufspannung}}	
	
\subsection{Bestimmung der I-U-Kennlinie}
	In den folgenden Abschnitten befinden sich die Auswertungen der jeweils für
	einen festen Abstand $d$ aufgenommenen Messwerte für die Bestimmung der $I \text{-}U$-Kennlinien
	der untersuchen Solarzelle.  
	
	Die Messwerte der $I\text{-}U$-Kennlinie für den jeweils festen Abstand $d = \SI{81.5}{\cm}$
	befinden sich in \cref{tab:Auswertung_Kennlinie_30mA} bis \ref{tab:Auswertung_Kennlinie_100mA}.
    Bei diesen Messwerten handelte es
	sich um den an der Widerstandsdekade eingestellten Widerstand $R$ und die jeweils aufgenommenen 
	Spannungen $U$ und der Strom $I$. Desweiteren sind die durch $P_{SZ} = U \cdot I$ bestimmte
	Leistung der Solarzelle, der daraus berechnete Wirkungsgrad $\eta$ nach \cref{eq:Theorie_Wirkungsgrad} 
	und der Lastwiderstand $R_{\text{last}} = \sfrac{U}{I}$ aufgeführt.
	Die für die Berechnung des Wirkungsgrades nötigen Leistungen $P_{Ph}$ und die dafür verwendeten 
	Intensitäten $J_{Ph}$ sind in \cref{tab:Auswertung_Kennlinien_Leistung} angegeben.   
	



		\begin{table}[!h]
	\centering
	\begin{tabular}{|c|c|c|}
		\hline
		Abstand & Lichtintensität & Lichtleistung\\
		$d$ [\si{\cm}] & $J_{Ph}$ [\si{\milli\watt\per\cm\squared}] & $P_{Ph}$ [\si{\milli\watt}]\\
\hline\hline
		\num{81.5(2)} & \num{6.4(1)} & \num{3.2(3)e+02}\\
		\num{60.5(2)} & \num{12(1)} & \num{6.0(7)e+02}\\
		\num{48.0(2)} & \num{17(1)} & \num{8.5(8)e+02}\\
		\num{40.0(2)} & \num{21(1)} & \num{1.1(1)e+03}\\
		\hline
	\end{tabular}
	\caption{Lichtintensität und -leistung bei den vier Messungen der Kennlinien \label{tab:Auswertung_Intensitaet_Leistung}}
\end{table}

		\begin{table}[!h]
	\centering
	\begin{adjustbox}{width=\textwidth,center}
	\begin{tabular}{|r|c|c|c|c|c|}
		\hline
		Widerstand & Spannung & Strom & Leistung & Lastwiderstand & Wirkungsgrad\\
		$R$ [\si{\ohm}] & $U$ [\si{\volt}] & $I$ [\si{\milli\ampere}] & $J$ [\si{\milli\watt}] \cref{std:Solarzellenleistung} & $R_{\text{last}}$ [\si{\kilo\ohm}] \cref{std:Lastwiderstand}& $\eta$ [\si{\percent}] \cref{std:Wirkungsgrad}\\
\hline\hline
		\num{0} & \num{30.30(1)} & \num{-0.2(1)} & \num{5(3)} & \num{2(1)e+02} & \num{2(1)}\\
		\num{10} & \num{31.20(1)} & \num{-0.4(1)} & \num{14(3)} & \num{7(2)e+01} & \num{4(1)}\\
		\num{20} & \num{31.40(1)} & \num{-0.8(1)} & \num{24(3)} & \num{42(6)} & \num{7(1)}\\
		\num{30} & \num{31.10(1)} & \num{-1.1(1)} & \num{34(3)} & \num{29(3)} & \num{11(1)}\\
		\num{40} & \num{30.10(1)} & \num{-1.4(1)} & \num{41(3)} & \num{22(2)} & \num{13(1)}\\
		\rowcolor[gray]{0.9} \num{50} & \num{27.50(1)} & \num{-1.5(1)} & \num{41(3)} & \num{18(1)} & \num{13(1)}\\
		\num{60} & \num{24.90(1)} & \num{-1.6(1)} & \num{39(2)} & \num{16(1)} & \num{12(1)}\\
		\num{70} & \num{22.20(1)} & \num{-1.6(1)} & \num{37(2)} & \num{13.5(8)} & \num{11(1)}\\
		\num{80} & \num{20.10(1)} & \num{-1.7(1)} & \num{34(2)} & \num{11.9(7)} & \num{11(1)}\\
		\num{90} & \num{18.50(1)} & \num{-1.7(1)} & \num{32(2)} & \num{10.8(6)} & \num{10(1)}\\
		\num{100} & \num{15.70(1)} & \num{-1.8(1)} & \num{28(2)} & \num{8.9(5)} & \num{8.6(9)}\\
		\num{110} & \num{14.60(1)} & \num{-1.8(1)} & \num{26(1)} & \num{8.2(5)} & \num{8.1(8)}\\
		\num{120} & \num{13.60(1)} & \num{-1.8(1)} & \num{24(1)} & \num{7.6(4)} & \num{7.6(8)}\\
		\num{130} & \num{12.70(1)} & \num{-1.8(1)} & \num{23(1)} & \num{7.1(4)} & \num{7.1(7)}\\
		\num{140} & \num{11.90(1)} & \num{-1.8(1)} & \num{22(1)} & \num{6.6(4)} & \num{6.7(7)}\\
		\num{150} & \num{11.20(1)} & \num{-1.8(1)} & \num{20(1)} & \num{6.2(3)} & \num{6.4(6)}\\
		\num{200} & \num{8.60(1)} & \num{-1.8(1)} & \num{15.7(9)} & \num{4.7(3)} & \num{4.9(5)}\\
		\num{210} & \num{8.20(1)} & \num{-1.8(1)} & \num{15.1(8)} & \num{4.5(2)} & \num{4.7(5)}\\
		\num{220} & \num{7.90(1)} & \num{-1.8(1)} & \num{14.5(8)} & \num{4.3(2)} & \num{4.5(4)}\\
		\num{230} & \num{7.60(1)} & \num{-1.8(1)} & \num{14.0(8)} & \num{4.1(2)} & \num{4.4(4)}\\
		\num{240} & \num{7.30(1)} & \num{-1.8(1)} & \num{13.4(7)} & \num{4.0(2)} & \num{4.2(4)}\\
		\num{250} & \num{7.10(1)} & \num{-1.8(1)} & \num{13.1(7)} & \num{3.9(2)} & \num{4.1(4)}\\
		\hline
	\end{tabular}
	\end{adjustbox}
	\caption{Messwerte für die Bestimmung der $I$-$U$-Kennlinie der Solarzelle im Abstand $81{,}5$cm \label{tab:Auswertung_Kennlinie_30mA}}
\end{table}

		\begin{table}[!h]
	\centering
	\begin{adjustbox}{width=\textwidth,center}
	\begin{tabular}{|r|c|c|c|c|c|}
		\hline
		Widerstand & Spannung & Strom & Leistung & Lastwiderstand & Wirkungsgrad\\
		$R$ [\si{\ohm}] & $U$ [\si{\volt}] & $I$ [\si{\milli\ampere}] & $J$ [\si{\milli\watt}] & $R_{\text{last}}$ [\si{\kilo\ohm}] & $\eta$ [\si{\percent}]\\
\hline\hline
		\num{0} & \num{51.50(1)} & \num{-0.3(1)} & \num{13(5)} & \num{2.0(8)e+02} & \num{2.2(9)}\\
		\num{10} & \num{51.10(1)} & \num{-0.7(1)} & \num{37(5)} & \num{7(1)e+01} & \num{6(1)}\\
		\num{20} & \num{50.90(1)} & \num{-1.2(1)} & \num{62(5)} & \num{42(3)} & \num{10(1)}\\
		\num{30} & \num{46.10(1)} & \num{-1.6(1)} & \num{73(5)} & \num{29(2)} & \num{12(2)}\\
		\num{40} & \num{38.70(1)} & \num{-1.7(1)} & \num{67(4)} & \num{22(1)} & \num{11(1)}\\
		\num{50} & \num{33.10(1)} & \num{-1.8(1)} & \num{59(3)} & \num{18(1)} & \num{10(1)}\\
		\num{100} & \num{18.50(1)} & \num{-1.9(1)} & \num{35(2)} & \num{9.7(5)} & \num{5.9(7)}\\
		\num{110} & \num{17.00(1)} & \num{-1.9(1)} & \num{33(2)} & \num{8.9(5)} & \num{5.4(7)}\\
		\num{120} & \num{15.70(1)} & \num{-1.9(1)} & \num{30(2)} & \num{8.1(4)} & \num{5.1(6)}\\
		\num{130} & \num{14.50(1)} & \num{-1.9(1)} & \num{28(1)} & \num{7.5(4)} & \num{4.7(6)}\\
		\num{140} & \num{13.50(1)} & \num{-1.9(1)} & \num{26(1)} & \num{7.0(4)} & \num{4.4(6)}\\
		\num{150} & \num{12.70(1)} & \num{-1.9(1)} & \num{25(1)} & \num{6.5(3)} & \num{4.1(5)}\\
		\num{200} & \num{9.60(1)} & \num{-1.9(1)} & \num{19(1)} & \num{4.9(3)} & \num{3.1(4)}\\
		\num{210} & \num{9.20(1)} & \num{-2.0(1)} & \num{18.0(9)} & \num{4.7(2)} & \num{3.0(4)}\\
		\num{230} & \num{8.80(1)} & \num{-2.0(1)} & \num{17.2(9)} & \num{4.5(2)} & \num{2.9(4)}\\
		\num{240} & \num{8.40(1)} & \num{-2.0(1)} & \num{16.5(8)} & \num{4.3(2)} & \num{2.7(3)}\\
		\num{250} & \num{8.00(1)} & \num{-2.0(1)} & \num{15.7(8)} & \num{4.1(2)} & \num{2.6(3)}\\
		\num{220} & \num{7.70(1)} & \num{-2.0(1)} & \num{15.1(8)} & \num{3.9(2)} & \num{2.5(3)}\\
		\hline
	\end{tabular}
	\end{adjustbox}
	\caption{Messwerte für die Bestimmung der $I$-$U$-Kennlinie der Solarzelle im Abstand $d = 68{,}5$cm \label{tab:Auswertung_Kennlinie_50mA}}
\end{table}

		\begin{table}[!h]
	\centering
	\begin{adjustbox}{width=\textwidth,center}
	\begin{tabular}{|r|c|c|c|c|c|}
		\hline
		Widerstand & Spannung & Strom & Leistung & Lastwiderstand & Wirkungsgrad\\
		$R$ [\si{\ohm}] & $U$ [\si{\volt}] & $I$ [\si{\milli\ampere}] & $J$ [\si{\milli\watt}] & $R_{\text{last}}$ [\si{\kilo\ohm}] & $\eta$ [\si{\percent}]\\
\hline\hline
		\num{0} & \num{75.90(1)} & \num{-0.4(1)} & \num{27(8)} & \num{2.1(6)e+02} & \num{3.2(9)}\\
		\num{10} & \num{75.20(1)} & \num{-1.1(1)} & \num{80(8)} & \num{71(7)} & \num{9(1)}\\
		\num{20} & \num{69.20(1)} & \num{-1.6(1)} & \num{114(7)} & \num{42(3)} & \num{13(2)}\\
		\num{30} & \num{54.40(1)} & \num{-1.8(1)} & \num{100(5)} & \num{30(2)} & \num{12(1)}\\
		\num{40} & \num{43.80(1)} & \num{-1.9(1)} & \num{83(4)} & \num{23(1)} & \num{10(1)}\\
		\num{50} & \num{36.50(1)} & \num{-1.9(1)} & \num{71(4)} & \num{19(1)} & \num{8.3(9)}\\
		\num{100} & \num{19.50(1)} & \num{-2.0(1)} & \num{39(2)} & \num{9.8(5)} & \num{4.6(5)}\\
		\num{110} & \num{17.90(1)} & \num{-2.0(1)} & \num{36(2)} & \num{8.9(4)} & \num{4.2(5)}\\
		\num{120} & \num{16.40(1)} & \num{-2.0(1)} & \num{33(2)} & \num{8.2(4)} & \num{3.9(4)}\\
		\num{130} & \num{14.80(1)} & \num{-2.0(1)} & \num{30(1)} & \num{7.4(4)} & \num{3.5(4)}\\
		\num{140} & \num{13.90(1)} & \num{-2.0(1)} & \num{28(1)} & \num{6.9(3)} & \num{3.3(4)}\\
		\num{150} & \num{13.00(1)} & \num{-2.0(1)} & \num{26(1)} & \num{6.5(3)} & \num{3.1(3)}\\
		\num{200} & \num{10.00(1)} & \num{-2.0(1)} & \num{20(1)} & \num{5.0(2)} & \num{2.4(3)}\\
		\num{210} & \num{9.50(1)} & \num{-2.0(1)} & \num{19(1)} & \num{4.7(2)} & \num{2.2(2)}\\
		\num{220} & \num{9.10(1)} & \num{-2.0(1)} & \num{18.3(9)} & \num{4.5(2)} & \num{2.2(2)}\\
		\num{230} & \num{8.70(1)} & \num{-2.0(1)} & \num{17.5(9)} & \num{4.3(2)} & \num{2.1(2)}\\
		\num{240} & \num{8.30(1)} & \num{-2.0(1)} & \num{16.7(8)} & \num{4.1(2)} & \num{2.0(2)}\\
		\num{250} & \num{8.00(1)} & \num{-2.0(1)} & \num{16.1(8)} & \num{4.0(2)} & \num{1.9(2)}\\
		\hline
	\end{tabular}
	\end{adjustbox}
	\caption{Messwerte für die Bestimmung der $I$-$U$-Kennlinie der Solarzelle im Abstand $d = 48{,}0$cm \label{tab:Auswertung_Kennlinie_75mA}}
\end{table}

		\begin{table}[!h]
	\centering
	\begin{adjustbox}{width=\textwidth,center}
	\begin{tabular}{|r|c|c|c|c|c|}
		\hline
		Widerstand & Spannung & Strom & Leistung & Lastwiderstand & Wirkungsgrad\\
		$R$ [\si{\ohm}] & $U$ [\si{\volt}] & $I$ [\si{\milli\ampere}] & $J$ [\si{\milli\watt}] \cref{std:Solarzellenleistung} & $R_{\text{last}}$ [\si{\milli\ohm}] \cref{std:Lastwiderstand}& $\eta$ [\si{\percent}] \cref{std:Wirkungsgrad}\\
\hline\hline
		\num{0} & \num{101.80(1)} & \num{-0.5(1)} & \num{5(1)e+01} & \num{2.1(4)e+02} & \num{5(1)}\\
		\rowcolor[gray]{0.9}\num{10} & \num{100.20(1)} & \num{-1.4(1)} & \num{1.4(1)e+02} & \num{70(5)} & \num{14(2)}\\
		\num{20} & \num{76.40(1)} & \num{-1.9(1)} & \num{142(8)} & \num{41(2)} & \num{14(1)}\\
		\num{30} & \num{53.80(1)} & \num{-2.0(1)} & \num{105(5)} & \num{27(1)} & \num{10(1)}\\
		\num{40} & \num{43.40(1)} & \num{-2.0(1)} & \num{86(4)} & \num{22(1)} & \num{8.2(9)}\\
		\num{50} & \num{36.40(1)} & \num{-2.0(1)} & \num{73(4)} & \num{18.2(9)} & \num{6.9(7)}\\
		\num{100} & \num{19.50(1)} & \num{-2.0(1)} & \num{40(2)} & \num{9.6(5)} & \num{3.8(4)}\\
		\num{110} & \num{17.90(1)} & \num{-2.0(1)} & \num{37(2)} & \num{8.8(4)} & \num{3.5(4)}\\
		\num{120} & \num{16.60(1)} & \num{-2.0(1)} & \num{34(2)} & \num{8.1(4)} & \num{3.2(3)}\\
		\num{130} & \num{15.30(1)} & \num{-2.0(1)} & \num{31(2)} & \num{7.5(4)} & \num{3.0(3)}\\
		\num{140} & \num{14.20(1)} & \num{-2.0(1)} & \num{29(1)} & \num{7.0(3)} & \num{2.8(3)}\\
		\num{150} & \num{13.20(1)} & \num{-2.0(1)} & \num{27(1)} & \num{6.5(3)} & \num{2.6(3)}\\
		\num{200} & \num{10.10(1)} & \num{-2.0(1)} & \num{21(1)} & \num{5.0(2)} & \num{2.0(2)}\\
		\num{210} & \num{9.60(1)} & \num{-2.0(1)} & \num{19(1)} & \num{4.7(2)} & \num{1.9(2)}\\
		\num{220} & \num{9.10(1)} & \num{-2.0(1)} & \num{18.5(9)} & \num{4.5(2)} & \num{1.8(2)}\\
		\num{230} & \num{8.80(1)} & \num{-2.0(1)} & \num{17.9(9)} & \num{4.3(2)} & \num{1.7(2)}\\
		\num{240} & \num{8.40(1)} & \num{-2.0(1)} & \num{17.1(8)} & \num{4.1(2)} & \num{1.6(2)}\\
		\num{250} & \num{8.10(1)} & \num{-2.0(1)} & \num{16.4(8)} & \num{4.0(2)} & \num{1.6(2)}\\
		\hline
	\end{tabular}
	\end{adjustbox}
	\caption{Messwerte für die Bestimmung der $I$-$U$-Kennlinie der Solarzelle im Abstand $40{,}0$cm \label{tab:Auswertung_Kennlinie_100mA}}
\end{table}

 
		
		
		\includeFigure[scale=0.7]{Grafiken/Kennlinie_30mA.pdf}{Grafische Darstellung der Messwerte der %
		    $I\text{-}U$-Kennlinie im Abstand \SI{81.5}{\cm}}{\label{fig:Auswertung_Kennlinie_30mA}}	
	

	
		\includeFigure[scale=0.7]{Grafiken/Kennlinie_50mA.pdf}{Grafische Darstellung der Messwerte der %
		    $I\text{-}U$-Kennlinie im Abstand \SI{60.5}{\cm}}{\label{fig:Auswertung_Kennlinie_50mA}}			
	

	
		\includeFigure[scale=0.7]{Grafiken/Kennlinie_75mA.pdf}{Grafische Darstellung der Messwerte der %
			$I\text{-}U$-Kennlinie im Abstand \SI{48.0}{\cm}}{\label{fig:Auswertung_Kennlinie_75mA}}	
	

	
 		\includeFigure[scale=0.7]{Grafiken/Kennlinie_100mA.pdf}{Grafische Darstellung der Messwerte der %
		 	$I\text{-}U$-Kennlinie im Abstand \SI{40.0}{\cm}}{\label{fig:Auswertung_Kennlinie_100mA}}	
	
