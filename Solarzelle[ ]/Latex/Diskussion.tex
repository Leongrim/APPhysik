In folgendem Abschnitt werden die erhaltenen Ergebnisse noch einmal abschließend diskutiert und
auf ihre Plausibilität hin überprüft.\\

Die erhaltenen Verläufe von Kurzschlussstrom \cref{fig:Auswertung_Kurzschlussstrom} und Leerlaufspannung 
\cref{fig:Auswertung_Leerlaufspannung} in Abhängigkeit von der Lichtintensität entsprechen den theoretischen 
Prognosen. So besteht zwischen den Kurzschlussstrom und Intensität ein linearer Zusammenhang, während 
für die Leerlaufspannung der anfänglich logarithmische Verlauf erkennbar ist. Für letzteren wurden jedoch zu wenig 
Werte für hohe Intensitäten aufgenommen, um die Annäherung an eine Grenzspannung beobachten zu können.
Die Ergebnisse dieser Versuchsteile sind somit plausibel.\\

Die aufgenommenen Kennlinien für die verschiedenen Lichtintensitäten, weisen alle den in \cref{fig:Theorie_Kennlinie}
gezeigten theoretischen Verlauf für bestrahlte Solarzellen auf. Die weiter aus diesen Messwerten erhaltenen 
Wirkungsgerade sind mit \SI{12}{\percent}-\SI{14}{\percent} ebenfalls plausible und lassen auf eine polykristalline 
Silizium Solarzelle schließen \cite{Gar07} .

Die allgemeine Änderung des Wirkungsgrad lässt sich analog zu der abgegebenen Leistung der Solarzelle beschreiben.
Diese nimmt stark bis zu einem Maximum und fällt dann in weniger rapider Weise wieder ab.


   