In folgendem Abschnitt werden der Aufbau und die Funktionsweise einer Solarzelle erläutert.
Zunächst werden jedoch die, für die Funktionsweise der Solarzelle wichtigen,
Eigenschaften von Halbleitern, sowie die Auswirkung von Dotierung auf Halbleiter betrachtet.\\



\subsection{Eigenschaften von Halbleitern und Dotierung}\label{sec:Theorie_Halbleiter}
Bei Halbleiten (z.B. Silizium) handelt es sich um Elemente, die sich aufgrund ihrer 
Eigenschaften in Bezug auf die elektrische Leitfähigkeit zwischen Leitern und Nichtleitern 
einordnen lassen. 
Diese Eigenschaft lässt sich mit Hilfe des Bändermodells beschreiben, welches 
zwei Elektronenbänder, das Valenzband und das Leiterband unterscheidet.
Das Valenzband ist dabei das oberste vollständig mit Elektronen besetzte Band. 
Ein elektrischer Strom wird von einem Material
jedoch nur im Leitungsband transportiert. Bei Leitern berühren sich diese beiden Bänder,
so dass diese jeder Zeit elektrischen Strom leiten können. Bei Nichtleitern ist das genaue
Gegenteil der Fall hier ist die Bandlücke zwischen Leiter- und Valenzband so groß,
dass diese nicht überwunden werden kann.
Bei Halbleitern berühren sich die beiden Bänder 
zwar nicht, doch ist die Bandlücke so klein, dass sie 
schon durch thermische Energie überwunden werden kann. Geschieht dies, so entsteht im 
Valenzband eine Elektronenfehlstelle, ein sogenanntes Loch, welches ebenfalls zu Leitung beiträgt.

Durch Dotierung mit Fremdatomen kann man die Leitungseigenschaften von Halbleiten noch weiter 
verbessern.
\emph{N}-dotierten Halbleitern, wie beispielsweise Silizium, wurde ein Fremdatom 
eingebracht, welches ein Elektron mehr besitzt als eine Atom des Halbleiters (z.B. Phosphor).
Entsprechend verhält es sich bei \emph{p}-dotierten Halbleitern, mit dem Unterschied,  
dass in diese ein Fremdatom mit einem Elektron weniger eingebracht wurde (z.B. Aluminium). 

Werden nun zwei unterschiedlich dotierte Halbleiter zusammen gebracht, rekombinieren die
Elektronen aus dem \emph{n}-dotierten mit den Löchern aus dem \emph{p}-dotiertem Teil 
an der Kontaktfläche, dem sogenannten \emph{p}-\emph{n}-Übergang und bilden die Raumladungszone. 
Durch diese, Diffusion genannte, Bewegung der Elektronen in die \emph{p}-dotierten Teil 
wird dieser, in der Nähe des \emph{p}-\emph{n}-Übergangs, negativ und der \emph{n}-dotierte Teil 
positiv geladen. 
Durch diese Ladungstrennung wird ein elektrisches Feld $E$ erzeugt, welches wiederum zu einem 
Strom $I_{D}$in Richtung der Feldlinien führt. Dieser lässt sich in Abhängigkeit des 
Sättigungsstroms $I_{0}$ und der Diffusionsspannung $U$ mit der Strom-Spannungscharakteristik
einer Diode
\begin{empheq}{equation}
	I_{D} = I_{0}\del{\Exp{\dfrac{eU}{kT}}- 1}
\end{empheq} 
beschreiben.
% % % %
%
% Skizze
% 
% % % %

\subsection{Aufbau und Funktionsweise einer Solarzelle}
Bei einer Solarzelle handelte es um ein Halbleiterbauelement, dessen grundsätzlicher
Aufbau dem einer Diode gleicht. So besteht die Solarzelle aus einer \emph{n}-dotierten Ober- und 
einem \emph{p}-dotierten Unterseite. Dabei ist die \emph{n}-dotierte Oberseite so dünn, dass sie 
Lichtdurchlässig ist.\\

In einer nicht bestrahlten Solarzelle stellt sich ein, wie in \cref{sec:Theorie_Halbleiter} 
beschriebener Strom in \emph{np}-Richtung ein. Da die \emph{n}-dotierte Schicht 
Lichtdurchlässig ist, werden durch den photoelektrischen Effekt Elektronen-Loch-Paare in 
der Raumladungszone erzeugt. Dies führt zu einem dem Strom $I_{D}$ entgegen gerichtetem 
Photostrom $I_{Ph}$ so dass sich die Strom-Spannungs-Charakteristik einer Solarzelle zu
\begin{empheq}{equation}
I = I_{0}\del{\Exp{\dfrac{eU}{kT}}- 1}-I_{Ph}
\end{empheq} 
ergibt.\\

\includeFigure[scale=0.3]{Grafiken/Kennlinie}{Qualitative Darstellung der $I\text{-}U$-Kennlinie%
 einer bestrahlen und einer unbestrahlten Solarzelle \cite{NHV1}}{\label{fig:Theorie_Kennlinie}}

Die Strom-Spannung-Kennlinie (siehe \cref{fig:Theorie_Kennlinie}) einer Solarzelle
wird durch die drei Größen Kurzschlussstrom $I_{K}$, Leerlaufspannung $U_{L}$ und
den Wirkungsgrad $\eta$ charakterisiert.\\
Der Kurzschlussstrom, welcher bei konstanter Temperatur proportional zu Fläche der Solarzelle
$A_{SZ}$ und der Intensität des Lichtes $J_{Ph}$ verläuft, stellt den maximalen Strom dar,
der bei kurzgeschlossener Solarzelle ($U = 0$) fließt.
Analog dazu ist die Leerlaufspannung die maximale Spannung der Solarzelle, wenn
an diese kein Verbraucher  angeschlossen ist ($I = 0$). Die Leerlaufspannung
hängt, bei ebenfalls konstanter Temperatur logarithmisch von der Lichtintensität $J_{Ph}$ 
ab, bis ein Sättigungswert erreicht ist und ist unabhängig von der Solarzellenfläche $A_{SZ}$.
Der Wirkungsgrad beschreibt das Verhältnis von gelieferter elektrischer Leistung der Solarzelle $P_{SZ}$
und der eingestrahlten Lichtleistung $P_{Ph}$
\begin{empheq}{equation}
	\label{eq:Theorie_Wirkungsgrad}
	\eta = \dfrac{P_{aus}}{P_{ein}}= \dfrac{P_{SZ}}{P_{Ph}} = \dfrac{U \cdot I}{J_{Ph} \cdot A_{SZ}}.
\end{empheq}
Das Maximum der Solarzellenleistung $I$ kann als Flächeninhalt des größtmöglichen Rechtecks bestimmt werden,
welches sich zwischen x- und y-Achse und der $I\text{-}U$-Kennlinie einzeichnen lässt 
(siehe \cref{fig:Theorie_Kennlinie}).