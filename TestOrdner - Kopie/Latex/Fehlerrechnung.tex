\setcounter{equation}{0}
\renewcommand{\theequation}{\Roman{equation}}
In diesem Abschnitt sind die mit Hilfe der gaußschen
Fehlerfortpflanzung berechneten Fehlergleichungen gegeben,
mit denen die Abweichungen der in \cref{sec:Auswertung} 
erhaltenen Messergebnisse berechnet wurden.
 
Der allgemeine Fehler, des Mittelwerts $\mathbb{M}(x_{i})\, \text{mit}\, i \in \{1, \dots, n\}$ durch Gaußsche Fortpflanzung ergibt sich aus der Gleichung:
 \begin{empheq}{equation}
 \sigma_{\mathbb{M}} = \dfrac{1}{n} \sqrt{\sum_{i = 1}^{n} \sigma_{x_{i}}^{2}}
 \label{std:Mittel}
 \end{empheq}  
 
Die Fehler der Wellenlängen in \cref{tab:Auswertung_Wellenlänge} berechnen sich nach:
\begin{empheq}{equation} 
\label{std:Wellenlaenge}
\sigma_{\lambda} = \dfrac{\sigma_{s}}{n}
\end{empheq} 
 
Der Fehler der inversen Wellenlängen in \cref{tab:Auswertung_Wellenlänge} ergibt sich vereinfacht 
aus:
\begin{empheq}{equation}
	 \label{std:inverseWellenlaenge}
	\sigma_{\lambda^{-1}} = \dfrac{\sigma_{\lambda}}{\lambda^{2}}
\end{empheq} 
 
Zur Berechnung des Fehlers der Geschwindigkeiten in
\cref{tab:Auswertung_Geschwindigkeiten} wurde 
 \begin{empheq}{equation}
 \label{std:Geschwindigkeit}
 \sigma_{v} = \sqrt{\frac{l^{2} \sigma_{t}^{2}}{t^{4}} + \frac{\sigma_{l}^{2}}{t^{2}}}
 \end{empheq} 
 verwendet.
 
 Der Fehler der Schallgeschwindigkeit \cref{eq:Auswertung_Schallgeschwindigkeit} ergibt sich nach
 \begin{empheq}{equation}
 \label{std:Schallgeschwinigkeit}
 \sigma_{c} =\sqrt{\lambda^{2} \sigma_{\nu}^{2} + \nu^{2} \sigma_{\lambda}^{2}}.
 \end{empheq} 
 
 Die Frequenzdifferenzen in \cref{tab:Auswertung_Frequenzänderung_Direkt} besitzen 
 den durch 
 \begin{empheq}{equation}
 \label{std:Frequenzänderung}
 \sigma_{\Delta\nu} = \sqrt{\sigma_{\nu}^{2} + \sigma_{\nu_0}^{2}} 
 \end{empheq} 
 berechenbaren Fehler.