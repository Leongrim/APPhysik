\subsection{Der Doppler-Effekt im Medium bzw. ohne Medium}
Da hier der Doppler-Effekt akustisch vermessen wird, muss beachtet werden, dass es einen Unterschied zwischen der Wellenausbreitung im Medium(z.B. Schall) und ohne Medium(elektromagnetische Wellen) gibt. Für elektromagnetische Wellen ist es nicht relevant wie sich Sender und Empfänger aufeinander zu bzw. wegbewegen, deshalb lässt sich dies über \underline{eine} Formel ausdrückenund zwar mit:
\begin{equation}
\nu'=\nu_0 \frac{\sqrt{1-\frac{v^2}{c_0^2}}}{1-\frac{v}{c_0}} 
\end{equation}
$c_0$ ist hierbei die Lichtgeschwindigkeit, $\nu'$ die verschobene Frequenz und $\nu_0$ die Ruhefrequenz. \\
Da für diesen Versuch allerdings der Doppler-Effekt im Medium interessant ist, muss zwischen den Fällen bewegte/ruhende Quelle und bewegter/ruhender Empfänger unterschieden werden.
\subsection{ruhende Quelle, bewegter Empfänger}
Für diesen Fall gilt:
\begin{equation}
\nu_E=\nu_0 \left( 1+\frac{v}{c}\right) 
\end{equation}
und der zugehörigen Frequenzänderung:
\begin{equation}
\Delta\nu=\nu_0 \frac{v}{c}
\label{eq:Theorie_BewegterEmpfänger}
\end{equation}
mit der Ausbreitungsgeschwindigkeit (hier: Schallgeschwindigkeit) $c$.
\newpage
\subsection{bewegte Quelle, ruhender Empfänger}
Der zweite Fall lässt sich formulieren als:
\begin{equation}
\nu_Q=\frac{\nu_0}{(1-\frac{v}{c})}
\label{eq:Theorie_BewegterSender}
\end{equation}
Bei der Entwicklung in eine Potenzreihe von $\frac{v}{c}$, ergibt sich:
\begin{equation}
\nu_Q=\nu_E + \nu_0\left( \left( \frac{v}{c}\right)^2+... \right) 
\end{equation}
