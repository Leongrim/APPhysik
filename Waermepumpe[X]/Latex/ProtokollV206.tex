\documentclass{scrartcl}
\usepackage[ngerman]{babel}
\usepackage[utf8]{inputenc}
\begin{document}

\section{Einleitung}\
\raggedright{In dem Versuch V206 "Die Wärmepumpe" wird der Vorgang, Energie einem
kälterem Reservoir zu entziehen und einem wärmeren hinzuzufügen, mit Hilfe der sog. Wärmepumpe realisiert.}\

\section{Theorie}\
\raggedright{Aus dem ersten Hauptsatz der Thermodynamik geht hervor, dass die innere Energie, die Summe Wärmeänderung und der verrichteten Arbeit ist.}\ \[\]
\centerline{$ $} \raggedleft{(1)} \[\]
\raggedright{Sobald dem kälteren Reservoir die Wärmemenge $Q_2$ abgezogen wird, wird die Wärmemenge $Q_1$ dem wärmeren Reservoir zu geführt, dies geschieht nur mit der verrichteten Arbeit $A$:}\ \[\]
\centerline{$Q_1 = Q_2 + A$}
\raggedleft{(2)} \[\]
\raggedright{Die Güteziffer $\nu$ gibt an wie "gut" der Verlust ist. Sie sagt aus, dass bei kleiner Temperaturdifferenz weniger Arbeitsaufwand geleistet werden muss.}\ \[\]
\centerline{$\nu_{ideal} = \frac {Q_1}{A} = \frac{T_1}{T_1-T_2}$} \raggedleft{(3)} \[\]
\raggedright{Da das System nicht vollständig isoliert ist, gibt es Abweichungen von $\nu_{ideal}$ und der Vorgang ist somit auch nicht reversibel unter realen Voraussetzungen, daher folgt:}\ \[\]
\centerline{$\nu_{real} < \frac{T_1}{T_1-T_2}$}\
\raggedleft{(4)} \[\]




\section{Durchführung}\
\raggedright{Mit Abb.1 wird der prinzipielle Aufbau der Wärmepumpe dargestellt. Der Kompresser K erzeugt einen ständigen Kreislauf des Systems, indem er den Aggregatszustand des Wassers beeinflusst von flüssig zu gasförmig und umgekehrt. \\
Durch das Drosselventil D wird ein Druckunterschied erzeugt, unter dem Druck $p_a$ und bei Temperatur $T_2$ ist das Wasser gasförmig und unter $p_b$ und $T_1$ flüssig. \\
Im kalten Reservoir 2 verdampft der Wasser unter Wärmeaufnahme und wird im Kompressor komprimiert, dadurch erhöhen sich Druck und Temperatur, danach gibt es bei der Kondensation die Wärme ab und so erhöht sich die Temperatur in Reservoir 1, während sie in Reservoir 2 sinkt.\\
Bei jedem Messgang werden Temperaturen und Drücke von $T_1$ und $T_2$, bzw. von $p_a$ und $p_b$ notiert. Abgelesen wird im 90 Sekunden Takt bis Reservoir 2 maximal 50 Grad Celsius erreicht hat.}\ \\

\begin{figure}
%\includegraphics{V206}
\caption{Prinzipieller Aufbau einer Wärmepumpe($p_b > p_a$;$T_1 > T_2$)}
\label{fig:Abb}
\end{figure}




\end{document}