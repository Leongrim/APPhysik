\documentclass[12pt]{article}                                                      % 'Artikel' Dokumentenklasse und Standardschriftgröße  
\usepackage[paper=a4paper,left=2.5cm,right=2.5cm,top=2.5cm,bottom=2.5cm]{geometry}  % Setzt das Papierformat und den Rand auf 2.5cm


\setlength{\parindent}{0.5\parindent}                                                         % Setzt die Einrückung von Absätzen auf gegebenen Abstand
\usepackage[onehalfspacing]{setspace}                                               % Legt Zeilenabstand fest


\usepackage[T1]{fontenc}                                                            % Legt FontKodierung fest
\usepackage[utf8]{inputenc}                                                         % Legt Zeichenkodierung fest

\usepackage[ngerman]{babel}                                                         % Deutsche Rechtschreibung
%\usepackage[ngerman]{varioref}                                                     % Versieht Referenzen mit Bezeichnung des Objektes
\usepackage{lmodern}                                                                % Empfohlener T1-Font für deutsche Texte
\usepackage[backend=biber, style=numeric-verb]{biblatex}							% Ermöglicht das Einbinden von Literaturverzeichnissen 	
\usepackage[babel,german=quotes]{csquotes}


% % % % % % % % % % % % % % % % % % % % % % % % % % % % % % % % % % % % % %
% 
%	Seiten Layout
%
% % % % % % % % % % % % % % % % % % % % % % % % % % % % % % % % % % % % % %

\usepackage{fancyhdr}                                                               % Ermöglicht detailierte Bearbeitung der Kopf- und Fußzeile
% Standard 'fancy' Pagestyle
	\fancyhf{} 	                          % Setzt Kopf und Fußzeile zurück 
	\setlength{\headheight}{28.0 pt}      % Höhe der Kopfzeile
	\setlength{\footskip}{18.0 pt}        % Höhe der Fußzeile

	\renewcommand{\headrulewidth}{.5 pt}  % Dicke des Kopfzeilentrennstrichs
	\renewcommand{\footrulewidth}{.5 pt}  % Dicke des Fußzeilentrennstrichs

	\lhead{\textbf{\VN\,\VNz}}                  % Angabe Links-Oben
	\rhead{\VD}                           % Angabe Rechts-Oben
%	\rfoot{\Namen}                        % Angabe Links-Unten
	\cfoot{\textbf{\thepage\ von \pageref{LastPage}}}  % Angabe Mitte-Unten
                      

% Pagestyle 'bib', mit Emails in der Fußleiste
%\fancypagestyle{bib}{%
%	\fancyhf{}%
%	\setlength{\headheight}{28.0 pt}% 
%	\setlength{\footskip}{18.0 pt}%
%	\renewcommand{\headrulewidth}{.5 pt}% 
%	\renewcommand{\footrulewidth}{.5 pt}% 
%
%	\lhead{\textbf{\VNr\\ \VN}}%
%	\rhead{\VD}% 
%	\rfoot{\Namen}%
%	\cfoot{\textbf{\thepage\ von \pageref{LastPage}}}  % 
%	\lfoot{\Emails}}
%	




% % % % % % % % % % % % % % % % % % % % % % % % % % % % % % % % % % % % % %
% 
%	Mathematik & Naturwissenschaft
%
% % % % % % % % % % % % % % % % % % % % % % % % % % % % % % % % % % % % % %

\usepackage[sumlimits,intlimits,namelimits]{amsmath}                                % Fügt mathematische Symbole hinzu, setzt Grenzen, Limiten und Indizes unter das Symbol und nicht dahinter

\usepackage{amssymb}                                                                % Fügt Symbole wie z.B. Zahlenmengen wie $\mathbb{R}$ hinzu 

\usepackage{amsthm}                                                                 

\usepackage{amsfonts}                                                               % Fügt Font für Mathematikumgebung hinzu

\usepackage{empheq}                                                                 % Stellt verbesserte Gleichungsumgebung bereit \begin{empheq}[<Aussehen>]{<Umgebungstyp>} ... \end{empheq}
%test

\usepackage[version=3]{mhchem}                                                     % Stellt chemische Struktur und Summenformeln bereit \ce{<Summenformel>}
%\usepackage{chemfig}                                                               % Stellt chemische Valenzstrichformeln fürganze Moleküle bereit \chemfig{<Molekül-Aufbau>}

\usepackage{siunitx}                                                                % Stellt eine verbesserte Formatierung von größen mit Einheiten zur Verfügung  
	\sisetup{locale = DE%
	%,prefixes-as-symbols = false %
	}                                   % Setzt das 'Mal'-Zeichen auf \cdot und das Dezimaltrennzeichen auf ',' 
                                                                                    % und ersetzt Prefixe wie '\kilo' mit der entsprechenden Zehnerpotenz
	\sisetup{separate-uncertainty = true}                                           % Ermöglicht vereinfachtes eintragen von Unsicherheiten '42.6(4)' --> '42.6 +/- 0.4'                                                                                 
                                   
\usepackage[b]{esvect}                                                              % Fügt verbesserte Vektorpfeile hinzu \vv{<Vektorname>} 

\usepackage{xfrac}

\usepackage{array}

\usepackage{commath}																% Fügt Differentialoperatoren und -quotienten ein 

 \usepackage{calc}

% % % % % % % % % % % % % % % % % % % % % % % % % % % % % % % % % % % % % %
% 
% Referenzen 
%
% % % % % % % % % % % % % % % % % % % % % % % % % % % % % % % % % % % % % %

\usepackage{scrdate}



\usepackage{lastpage}                                                               % Macht die letzte Seitenzahl referenzierbar mit \pageref{LastPage}

\usepackage{hyperref}																% Ermöglicht (automatische) Verlinkungen im Dokument
	\hypersetup{hidelinks}                                                          % Entfernt farbige Markierung der Links  
	
\usepackage[ngerman]{cleveref}														% Umfangreiche Referenzierungsmöglichkeiten, inkl. Beschriftung mit ref. Typ
\crefformat{equation}{(#2#1#3)}														% \cref{} Einstellung: Gleichungen bekommen
\crefrangeformat{equation}{(#3#1#4) bis~(#5#2#6)}									%  nicht den Text 'Gleichung' vorgesetzt


% % % % % % % % % % % % % % % % % % % % % % % % % % % % % % % % % % % % % %
% 
% Grafiken, Tabellen und Zeichnungen 
%
% % % % % % % % % % % % % % % % % % % % % % % % % % % % % % % % % % % % % %

\usepackage{graphicx}                                                               % Ermöglicht das Einbinden Grafiken '\includegraphics[<Optionen>]{<Grafikpfad>}' und Veränderungen im Text, wie z.B. Schriftfarbe 
                                                        
%\usepackage{subfig}                                                                % Ermöglicht das Hinzufügen von Unterabbildung zu einer Abbildung

\usepackage{wrapfig}

\usepackage{tikz}                                                                   % Ermöglicht Zeichnungen im Dokument \begin{tikzpicture} ... \end{tikzpicture}
	\usetikzlibrary{arrows}                                                         % Fügt zusätzlichen Pfeilspitzen hinzu
%\usepackage{booktabs}
%\usepackage{slashbox}                                                              % Ermöglicht das Einfügen mehrerer Einträge in eine Tabellenzelle, getrennt von einem '\' \backslashbox{<Eintrag unten-links>}{<Eintrag oben-rechts>} TIPP: Leerzeichen

\usepackage[font=small,labelfont=bf]{caption} 										% Ermöglicht Einstellung der caption von floats

% % % % % % % % % % % % % % % % % % % % % % % % % % % % % % % % % % % % % %
%
% Zeichenerweiterung, Textbezogenen Veränderungen
%
% % % % % % % % % % % % % % % % % % % % % % % % % % % % % % % % % % % % % %

\usepackage[normalem]{ulem}                                                         % Fügt verbesserte Unterschtreichungen hinzu, z.B. doppelt, gezackt, gewellt, etc.

\usepackage{xcolor}

\usepackage{enumitem}                                                               % Ermöglicht detailierte Einstellungen an Aufzählungssymbolen


\usepackage{textcomp}  																% Fügt extra Symbole hinzu  
\usepackage{pifont}																    % Fügt neue Symbole und Zeichen ein
\renewcommand\thefootnote{\ding{\numexpr171+\value{footnote}}}                      % Fußnoten mit Zahlen in Kreisen

% % % % % % % % % % % % % % % % % % % % % % % % % % % % % % % % % % % % % %
%
% Steuerelemente 
%
% % % % % % % % % % % % % % % % % % % % % % % % % % % % % % % % % % % % % %

\title{} 
\author{} 
\addbibresource{../_globales/latex/Literatur.bib}
\pagestyle{fancy} 																    % Anwenden des Erweiterten Seitenlayouts
%\renewcommand{\thefootnote}{}
%\setlength{\footnotesep}{2cm}
\setlength{\skip\footins}{2cm}                                                      % Abstand zwischen Text und Fußnoten
%\setlength{\itemsep}{7.5pt}

% % % % % % % % % % % % % % % % % % % % % % % % % % % % % % % % % % % % % %
%
% Eigenen Definitionen
%
% % % % % % % % % % % % % % % % % % % % % % % % % % % % % % % % % % % % % %

                                                                 
% Mathematik
\newcommand{\eqctr}{%
\refstepcounter{equation}\tag{\theequation}}

\renewcommand{\i}{\ensuremath{\textsc{i}}}
\newcommand{\e}{\ensuremath{\textsl{e}}}
\renewcommand{\Im}{\mathrm{Im}\,}
\renewcommand{\Re}{\mathrm{Re}\,}

\newcolumntype{C}{ >{\centering\arraybackslash} m{2.5 cm}}
%\DeclarePairedDelimiter{\abs}{\lvert}{\rvert}
\DeclarePairedDelimiter{\mean}{\langle}{\rangle}

%%%%%%%%%%%%%%%%%%%%%%%%%%%%%%%%%%%%%%%%%%%%%%%%%%%%%%%%%%%%%%%%%%%%%%%%%%

% Verbesserungen hervorheben



