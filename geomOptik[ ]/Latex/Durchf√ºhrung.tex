Der Versuchsaufbau besteht aus einer Lichtquelle (davor das Perl-L), einer Schiene, auf der der abzubildende Gegenstand und die verschiedenen Linsen gesetzt und verschoben werden können und einem Schirm, auf dem das Bild sichtbar gemacht wird. Es werden die Abstände zwischen den beteiligten Komponenten gemessen.\\

Als erstes wird die Linsengleichung an einer Sammellinse
verifiziert, danach die Brennweite einer unbekannten Sammellinse, als drittes die Brennweite für rotes, blaues und weißes Licht
nach der Methode von Bessel und schließlich wird die Brennweite eines Linsensystems nach der Methode von Abbe bestimmt.

\subsection{Verifikation der Linsengleichung}

Zur Verifikation der Linsengleichung wird eine Sammellinse mit bekannter Brennweite auf die Schiene zwischen Perl-L und Schirm gestellt. Jetzt wird
eine Gegenstandsweite $g$ eingestellt, indem die Linse verschoben wird,
und dazu der Schirm so angepasst, dass das Bild scharf ist und damit die Bildweite $b$ bestimmt. Es werden dabei zehn Messungen durchgeführt.

\subsection{Bestimmung der Brennweite einer unbekannten Linse}

Um die Brennweite einer unbekannten Sammellinse zu bestimmen, wird so
vorgegangen wie im Falle der Verifiktation der Linsengleichung. Es wird
eine Sammellinse, deren Brennweite nicht bekannt ist (hier: ist dies durch eine Kunststoff-Linse gegeben, die mit Wasser gefüllt ist
und deren Wölbung sich durch die Menge an Wasser innerhalb variieren
lässt), auf die Schiene gestellt und für verschiedene Gegenstands- und Bildweiten bestimmt.

\subsection{Bestimmung der Brennweite nach Bessel}

Bei der Methode nach Bessel wird der Abstand zwischen Gegenstand und Schirm
konstant gehalten und durch Verschieben der Linse zwei Positionen
gesucht, so dass das Bild scharf auf dem Schirm abgebildet ist. Es ergibt
sich dadurch eine symmetrische Linsenstellung, bei der Gegenstands- und
Bildweite jeweils vertauschen. Wenn die Bildweite $b$ größer als
die Gegenstandsweite $g$ ist, wird das Bild vergrößert. (Umgekehrt verkleinert)\\

Aus dem Abstand von Schirm und Gegenstand $e$
%
\begin{equation}
  e = g + b
\end{equation} 
%
und dem Abstand der beiden Linsenpositionen, an denen der Gegenstand
scharf auf den Schirm abgebildet wird $d$ 
%
\begin{equation}
  d = |g - b|
\end{equation}
%
ergibt sich aus der Linsengleichung die Formel:
%
\begin{equation} 
f = \frac{e^2 - d^2}{4e}
\end{equation}
%
für die Brennweite der Linse.\\
Insgesamt werden für zehn verschiedene
Abstände $e$ jeweils die Gegenstands- und Bildweiten für die
symmetrischen Linsenpositionen bestimmt.
Das Gleiche wird mit einem roten bzw. blauen Lichtfilter vor dem Perl-L wiederholt.

\subsection{Bestimmung der Brennweite nach Abbe}

Die Methode von Abbe erlaubt es, die Brennweite und die Lage der
Hauptebenen eines Linsensystems durch das Setzen eines beliebigen Referenzpunktes $A$ (hier: Kante des Linsensystems bei der konkaven Linse), können die Gegenstands- und Bildweite $g'$ bzw. $b'$ bestimmt werden.\\
Es wird in diesem Versuchsteil ein System aus Sammel- und Zerstreuungslinse
aufgebaut, welche möglichst nahe hintereinander stehen, da die Gegenstands- und Bildweite bezüglich der Hauptebenen gemessen wird, deren Lage aber nicht bekannt ist, deshalb wird der Abstand des Gegenstands $g'$
bzw. Bildes $b'$ gegen einen festen Referenzpunkt gemessen. Die Abstände der Hauptebenen von diesem Referenzpunkt werden mit $h$ und $h'$
bezeichnet.\\
Mit den folgenden Formeln werden die Beziehung zwischen
Brennweite und Abbildungsmaßstab und den gemessenen Längen hergestellt:
%
\begin{align}
  g' &= g + h  = f \left(1 + \frac{1}{V}\right) + h\\
  b' &= b + h' = f (1 + V) + h'
\end{align}

Schirm und Linsensystem werden nun so verschoben, dass das Bild
scharf auf dem Schirm erscheint, wobei jedes Mal die Abstände $g', b'$ zum gewählten Punkt $A$ gemessen werden.
