Die geometrische Optik untersucht die Bewegung von Lichtstrahlen, diese breiten sich geradlinig aus und werden an Übergängen zwischen zwei Medien gebrochen.

In diesem Versuch wird für so einen Übergang zwischen zwei Medien eine Linse eingesetzt. Ein Lichtstrahl, der eine Linse
passiert, wird zweimal gebrochen. Das erste Mal beim Eintritt in
die Linse und das zweite Mal beim Austritt.  Es gibt verschiedene
Linsentypen:  Die Sammellinse(konvex) bündelt parallele Strahlen im Brennpunkt der Linse. Ein weiterer Linsentyp ist die
Zerstreuungslinse(konkav). Parallele Lichtstrahlen werden hier von der optischen Achse weggebrochen, daher der Name.\\

Die Sammellinse bildet einen Gegenstand hinter der Linse auf ein reelles Bild ab, das mit einem Schirm sichtbar gemacht werden kann. Brennweite $f$ und Bildweite $b$. Die Entfernung des reellen Bildes von der Linse, werden für die konvexe Linse durch eine positive Zahl angegeben.\\
Bei einer Zerstreuungslinse entsteht weder vor der Linse, noch dahinter ein
Bild, das von einem Schirm aufgefangen werden kann. Blickt man
allerdings durch die Linse scheint es, als ob der Gegenstand vor der
Linse stünde. Es entsteht ein virtuelles Bild, daher wird die Brenn- und Bildweite negativ gemessen.\\

Bei der Konstruktion des Abbildungsverhaltens geht man hier von einer Vereinfachung in Form einer Brechung an der Mittelebene der Linse aus.\\
Bei dicken Linsen hingegen werden zwei sogenannte
Hauptebenen eingeführt, an denen der Lichtstrahl gebrochen wird.\\
Zur eigentlichen Konstruktion werden drei verschiedene Strahlentypen
verwendet. Der Mittelpunktstrahl geht durch die Mitte der Linse und
wird nicht gebrochen, ein Parallelstrahl verläuft vor seine Brechung
parallel zur optischen Achse und wird an der Mittelebene so gebrochen,
dass er durch den Brennpunkt der Linse verläuft. Hierbei muss beachtet
werden, ob die Linse eine Sammel- oder Zerstreuungslinse ist, bzw. ob
die Brennweite $f$ positiv oder negativ ist.\\
Ein Strahl, der durch den Brennpunkt der Linse verläuft, wird so an der Mittelebene gebrochen, dass sein Strahl danach parallel zur optischen Achse verläuft.\\

Aus der Konstruktion folgt mit den Strahlensätzen der folgende
Zusammenhang für den Abbildungsmaßstab $V$:
%
\begin{equation}
  V = \frac{B}{G} = \frac{b}{g},
  \label{eq:Theorie_Abbildungsgesetz}
\end{equation}
%
mit $B$ und $G$ Bild- bzw. Gegenstandsgröße und $b$ und
$g$ Bild- bzw. Gegenstandsweite.\\

Für dünne Linsen ergibt sich noch die
sogenannte Linsengleichung:
%
\begin{equation}
	\label{eq:Theorie_Linsengln}
  \frac{1}{f} = \frac{1}{b} + \frac{1}{g}.
\end{equation}

Zur Beschreibung des Brechverhaltens von Linsen wird eine neue
Größe, die sogenannte Brechkraft eingeführt:
%
\begin{equation}
  D = \frac{1}{f}
\end{equation}
%
Diese hat die Einheit Dioptrie ($\frac{1}{m}$). Wird ein System
aus mehreren dünnen Linsen zusammengebaut, so addieren sich die
Brechkräfte.
