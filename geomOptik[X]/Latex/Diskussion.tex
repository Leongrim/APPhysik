
Im Folgenden werden die in \cref{sec:Auswertung} erhaltenen Ergebnisse 
noch einmal abschließend diskutiert und dabei auf ihre Plausibilität
hin überprüft. Dabei wird auch Bezug auf Versuchsaufbau und -durchführung 
genommen. 

Die im ersten Versuch erhaltene Brennweite \cref{val:Auswertung_Bekannt} der Linse
mit bekannter Brennweite  $f_{bek} = \SI{10.0}{\cm}$ zeigt mit einer relativen Abweichung 
von ca. $\SI{2}{\percent}$ eine große Übereinstimmung mit 
dem tatsächlichen Wert. Auch die Überprüfung des Abbildungsgesetzes \cref{eq:Theorie_Abbildungsgesetz} lieferte
für die beiden untersuchten Wertepaare eine Übereinstimmung im Rahmen der Messgenauigkeit. \\
Diese Abweichungen sind allgemein durch die Subjektivität des 
Scharfsehens zu begründen. Da es bei der Versuchsdurchführung darauf ankommt ein möglichst
scharf umrissenes Bild des Gegenstands zu erzeugen und der Experimentataor diese Eigenschaft 
des Bildes mit seinen Augen beurteilt, sind der Genauigkeit der Ergebnisse schon durch 
das individuelle Sehvermögen bzw. der Akkommodationsfähigkeit beschränkt. Anders formuliert, 
zeigen die Ergebnisse aus der Beobachtung eines Linsensystems mit einem zweiten Linsensystem(das Auge und eventuelle Sehhilfen), 
schon wegen der Ungenauigkeit des zweiten, 
Abweichungen vom tatsächlichen Wert auf. \\

Auch aus den Daten der Untersuchung der mit Wasser gefüllten Linse, unbekannter Brennweite
erhält man einen Wert \cref{val:Auswertung_Unbekannt} dessen Genauigkeit zwar in Frage zustellen ist,
der jedoch Plausibel scheint.\\
Bei dieser Linse sind die Ungenauigkeiten der Ergebnisse neben dem schon angesprochenen Grund
noch durch die Beschaffenheit der verwendeten Linse zu begründen, die aufgrund ihrer Bauweise 
nicht die Genauigkeit einer Glaslinse erreicht. Dies liegt vor allem an Unreinheiten im eingefüllten
Wasser wie zum Beispiel eingeschlossene Luftbläschen.\\

Die erhalten Werte der Methode von Bessel \cref{val:Auswertung_BesselWeis}, \cref{val:Auswertung_BesselRot} 
und \cref{val:Auswertung_BesselBlau} liefern im Rahmen ihrer Genauigkeit die erwarteten Ergebnisse.
Die Abweichung der gemessenen Brennweite von weißem Licht zum tatsächlichen Wert $f_{bek} = \SI{10.0}{\cm}$ 
ist mit weniger als $\SI{1}{\percent}$ noch geringer als bei der zuvor verwendeten Methode,
was wiederum für die größere Genauigkeit der Methode von Bessel gegen über der Linsengleichungs-Methode
spricht. Auch der Effekt der chromatischen Aberration wird aus den Messwerten deutlich, da sich
die Brennweiten von blauem und rotem Licht um ca. $\SI{1}{\percent}$ unterscheiden.

Da die Steigungen der beiden Regressionsgeraden, die aus den Messwerte der Methode von Abbe bestimmt wurden
beide die nahezu gleiche Steigung aufwiesen, scheint die erhaltene Brennweite \cref{val:Auswertung_Abbe} 
plausibel und auch die erhaltenen Positionen der Hauptachsen weisen keinen unrealistischen 
und damit unplausiblen Abstand zum gewählten Referenzpunkt auf.



 
 
 